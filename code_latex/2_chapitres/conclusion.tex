\chapter{Conclusion}
\section{Résumé des résultats}
\subsection{Synthèse des principaux résultats obtenus}
\subsection{Impact des méthodes d'agrégation sur les portefeuilles de passifs}

\section{Perspectives d'amélioration}
\subsection{Axes d'amélioration pour les générateurs de portefeuilles de passifs}
\subsection{Évolutions possibles des méthodes d'agrégation et de modélisation ALM}  
\subsection{Autres domaines d'application des générateurs de portefeuilles de passifs}
\section{Conclusion générale}
