\chapter{Contraintes techniques et création d'outils}
\section{Nécessité du générateur de portefeuille passif}
\subsection{Besoin de générer des données pour simuler un nouveau produit}
\subsection{Simuler différentes évolutions du business mix pour orienter politique de souscription/politique commerciale.}
\subsection{Simuler un portefeuille représentatif du marché ou composé des principaux concurrents pour se positionner.}
\section{Contraintes techniques associées}
\subsection{Mise à jour sur des outils plus récents}
\subsection{langage open source, permet de s'écarter des problématiques financières (coût de licence).}
\subsection{Travail sur des outils mis à jour fréquemment (Python, Polars).}

\section{Développement du Modèle ALM en Python}
\subsection{Présentation du modèle ALM développé pour Accenture}
\subsection{Fonctionnement du modèle et apprentissage personnel.}
\subsection{Limites du modèle à l’heure actuelle.}

\section{Générateur de portefeuille de passif}
\subsection{Description des contraintes techniques rencontrées}
Description des contraintes techniques rencontrées dans la génération des portefeuilles.

\subsection{Description du générateur de modèle point}
\subsection{Importance pour un cabinet de conseil et rassurance des clients.}