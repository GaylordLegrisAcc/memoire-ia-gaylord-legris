\chapter{Construction d'un Générateur de Portefeuilles de Passifs Réaliste}

\section{Objectifs Stratégiques et Contraintes Techniques}
    \subsection{Besoins métiers : simulation de nouveaux produits et analyse concurrentielle}
    % Votre texte ici...
    \subsection{Défis de la modélisation : réalisme, volumétrie et flexibilité}
    % Votre texte ici...

\section{Méthodologie de Génération et Modélisation Statistique}
    \subsection{Approche stochastique par lois de probabilité}
    % Votre texte ici...
    \subsection{Calibration des distributions marginales à partir des données de marché}
    % Votre texte ici...
    \subsection{Perspective : modélisation des dépendances par la théorie des copules}
    % Votre texte ici...

\section{Présentation de l'Outil et du Portefeuille de Référence Généré}
    \subsection{Architecture de l'application en Python et choix technologiques}
    % Votre texte ici...
    \subsection{Description des paramètres d'entrée et des formats de sortie}
    % Votre texte ici...
    \subsection{Analyse descriptive du portefeuille de référence}
    % Votre texte ici...





% La capacité à tester la robustesse des modèles et la pertinence des analyses de sensibilité repose sur un prérequis fondamental : la disponibilité de données de passif variées et réalistes. Pour un cabinet de conseil, où l'accès aux portefeuilles des clients n'est pas systématique, la faculté de générer des portefeuilles synthétiques, mais représentatifs du marché, constitue un atout stratégique majeur. C'est dans ce contexte qu'un générateur de portefeuilles de passifs a été conçu et développé.

% Ce chapitre a pour vocation de présenter cet outil essentiel. Nous détaillerons les besoins stratégiques et analytiques auxquels il répond, la méthodologie de génération retenue, les contraintes techniques rencontrées, et enfin l'environnement technologique moderne dans lequel il a été implémenté.

% \section{Nécessité du générateur de portefeuille de passifs}

% Un générateur de portefeuille de passifs est un outil permettant de créer des portefeuilles d'assurance vie synthétiques en simulant les caractéristiques des assurés et de leurs contrats. Le développement d'un tel outil s'est imposé comme une nécessité pour plusieurs raisons stratégiques et analytiques, tant pour un cabinet de conseil que pour un organisme d'assurance.

% \subsection{Générer des données pour la simulation d'un nouveau produit}

% Le lancement d'un nouveau produit d'assurance vie est une décision stratégique majeure pour un assureur. Avant sa commercialisation, il est impératif d'évaluer ses impacts potentiels sur le profil de risque et la rentabilité de l'entreprise. Le générateur de portefeuille permet de simuler l'intégration de ce nouveau produit au sein d'un portefeuille existant. En générant des milliers de polices virtuelles conformes aux caractéristiques du nouveau produit, il devient possible de projeter leur comportement dans le temps et d'analyser leur effet sur les indicateurs clés de Solvabilité 2, tels que le \textit{Best Estimate} (BE) et le \textit{Solvency Capital Requirement} (SCR). Cet outil offre ainsi un laboratoire virtuel pour tester et ajuster les caractéristiques d'un produit avant même sa mise sur le marché.

% \subsection{Simuler l'évolution du business mix pour orienter les politiques commerciale et de souscription}

% La stratégie d'un assureur peut évoluer, l'amenant à modifier son \textit{business mix}, c'est-à-dire la répartition de son portefeuille entre différents types de produits (épargne, prévoyance, etc.). Le générateur de portefeuille est un outil précieux pour la prise de décision stratégique. Il permet de simuler divers scénarios d'évolution du portefeuille : Quel serait l'impact d'une politique commerciale plus agressive sur les produits en unités de compte ? Comment une modification de la politique de souscription affecterait-elle la sinistralité et la rentabilité à long terme ? En générant des portefeuilles correspondant à ces différentes hypothèses, la direction peut visualiser les conséquences sur le bilan et le compte de résultat prévisionnels, et ainsi orienter plus efficacement ses politiques commerciale et de souscription.

% \subsection{Simuler un portefeuille représentatif du marché pour l'analyse concurrentielle}

% Pour une compagnie d'assurance, la capacité à se positionner par rapport au marché et à analyser la concurrence est fondamentale. Ne disposant pas des portefeuilles détaillés de ses concurrents, il est peut s'avérer nécessaire de pouvoir reconstituer des portefeuilles représentatifs. Grâce à des données publiques ou sectorielles, le générateur peut créer un portefeuille "moyen" du marché français, ou simuler le portefeuille d'un concurrent principal. Ces portefeuilles synthétiques servent de base pour des analyses comparatives (\textit{benchmarking}), permettant d'évaluer la performance relative d'un concurrent, d'identifier les meilleures pratiques ou d'anticiper les stratégies des autres acteurs du marché.

% \subsection{Description des contraintes techniques rencontrées}

% La conception du générateur a soulevé plusieurs contraintes techniques. La première fut de garantir le réalisme des portefeuilles générés. Il ne s'agit pas simplement de créer des données aléatoires, mais de reproduire les corrélations observées dans la réalité (par exemple, entre l'âge de l'assuré et le montant des primes). Une autre contrainte était liée à la volumétrie : l'outil devait être capable de générer des portefeuilles de plusieurs millions de lignes de manière performante. Enfin, la flexibilité était un critère essentiel ; l'outil devait permettre de paramétrer finement les caractéristiques des produits à simuler et les lois de comportement (rachat, mortalité) associées.

% \subsection{Méthodologie de génération des portefeuilles}

% La méthodologie adoptée repose sur une approche stochastique. Pour chaque caractéristique clé d'un contrat (âge de l'assuré, montant de la provision mathématique, type de support, etc.), une loi de probabilité a été définie et calibrée à partir de données de marché. L'outil génère ensuite, ligne par ligne, des assurés virtuels en tirant aléatoirement des valeurs selon ces distributions. Pour modéliser les dépendances entre les variables, des techniques de copules ont été envisagées afin de garantir la cohérence et le réalisme des profils générés. Le résultat est un portefeuille granulaire, au contrat près, qui peut ensuite être agrégé en \textit{model points} pour être utilisé dans le modèle de projection ALM.

% \subsection{Présentation de l'outil développé}

% L'outil final se présente comme une application développée en Python, dotée d'une interface permettant à l'utilisateur de paramétrer la simulation. Les entrées principales sont :
% \begin{itemize}
% \item Les caractéristiques du ou des produits à simuler (type de garantie, chargements, etc.).
% \item Les paramètres des lois statistiques pour chaque variable (âge, sexe, montant, etc.).
% \item La taille du portefeuille souhaité.
% \item Les lois de comportement (tables de mortalité, formules de rachat).
% \end{itemize}
% En sortie, l'outil produit un fichier standardisé contenant le portefeuille de passifs généré, directement exploitable par le modèle ALM.

% \section{Choix technologiques et environnement de développement}

% Le passage de VBA à un écosystème Python n'est pas anodin ; il reflète une orientation stratégique vers des technologies plus modernes, ouvertes et performantes, mieux adaptées aux défis du "Big Data" et des calculs intensifs en actuariat.

% \subsection{Migration vers des technologies modernes}
% L'environnement Excel/VBA, bien que très répandu, montre ses limites face à la complexité et à la volumétrie des modèles ALM modernes. La migration vers Python a permis de s'affranchir des limitations de mémoire et de performance d'Excel, tout en bénéficiant d'un langage structuré favorisant la qualité du code, la modularité et les tests automatisés, ce qui est un gage de maintenabilité et de fiabilité à long terme.

% \subsection{Avantages d'un langage open-source}
% Le choix de Python, un langage \textit{open-source}, présente des avantages considérables. D'un point de vue financier, il élimine les coûts de licence associés à de nombreux logiciels propriétaires. Plus important encore, il donne accès à une communauté mondiale de développeurs et de scientifiques, qui contribuent à un écosystème de librairies extrêmement riche et en constante évolution. Cette effervescence garantit un accès permanent aux algorithmes et aux techniques les plus récents.

% \subsection{Utilisation d'un écosystème dynamique}
% Le projet s'est appuyé sur des librairies de pointe pour la manipulation de données et le calcul scientifique. En particulier, l'utilisation de la librairie \textit{Polars} (ou alternativement \textit{Pandas}), écrite en Rust, a permis d'atteindre des niveaux de performance très élevés pour le traitement de grands volumes de données, dépassant de loin les capacités des outils traditionnels. Le fait que ces outils soient mis à jour très fréquemment par la communauté garantit que le modèle bénéficie en permanence des dernières optimisations et fonctionnalités.