\chapter{Contexte réglementaire et modélisation en assurance vie}
\label{chap:contexte}
\newpage
\section{Les spécificités des produits d'assurance vie épargne}
\label{sec:spec_av}

\subsection{Principes fondamentaux du contrat d'assurance vie}

L'assurance vie est une convention par laquelle un assureur, en contrepartie du versement de primes, s'engage à verser un capital ou une rente à la survenance d'un événement incertain lié à la durée de la vie humaine. Cet événement, qui constitue l'aléa au cœur du contrat, peut être le décès de l'assuré avant une date donnée ou, à l'inverse, sa survie jusqu'à cette date. Ce mécanisme repose sur un cycle de production inversé : l'assureur perçoit les primes bien avant de devoir potentiellement régler les prestations, ce qui l'amène à investir ces sommes sur des horizons de temps longs pour honorer ses engagements futurs.

\begin{figure}[H]
    \centering
    \includegraphics[width=0.8\textwidth]{images/2_chapitres/chapitre1/cycle-de-production.jpg}
    \caption{Cycle de production inversé en assurance vie}
    \label{fig:cycle_production_inverse}
\end{figure}

La nature de ces engagements répond à des objectifs variés. Les contrats en cas de vie prévoient le versement d'un capital ou d'une rente à une échéance prévue si l'assuré est en vie ; ils sont typiquement utilisés pour se constituer un complément de retraite ou une épargne de précaution. À l'opposé, les contrats en cas de décès garantissent le versement d'un capital ou d'une rente au(x) bénéficiaire(s) désigné(s) au décès de l'assuré, souvent pour protéger des proches ou anticiper des droits de succession. Il existe également des contrats mixtes qui combinent ces deux garanties.


Le fonctionnement de ces contrats repose sur la capitalisation : les primes versées sont investies pour financer la propre couverture future de l'assuré. De par leur nature, ces engagements s'étendent sur de très longues périodes. Une caractéristique fondamentale de l'assurance vie française est sa liquidité. L'assuré dispose de la possibilité de récupérer son épargne à tout moment via un rachat, qui peut être partiel ou total. Cette faculté de rachat constitue une option dont la valeur et le risque doivent être finement gérés par l'assureur, car son exercice a un impact direct sur les besoins de liquidité du portefeuille. La fiscalité joue un rôle incitatif majeur car les plus-values sont imposées plus lourdement si le rachat intervient avant la huitième année du contrat. Ceci encourage alors l'épargne de long terme.

La gestion de ces engagements de long terme amène l'assureur à proposer différentes modalités d'investissement. Celles-ci permettent de répartir le risque financier entre l'assuré et l'assureur, définissant ainsi le profil de rendement potentiel du contrat. Un contrat plus sûr aura des possibilités de rendements plus faible qu'un contrat risqué. L'épargne des assurés peut ainsi être investie sur deux principaux types de supports aux profils de risque distincts, qui peuvent être combinés au sein de différents types de contrats.

\subsection{Les principaux supports d'investissement}

L'épargne des assurés peut être investie sur deux principaux types de supports aux profils de risque distincts, qui peuvent être combinés au sein de différents types de contrats.



Le \textbf{fonds en euros} est le support historique et sécuritaire de l'assurance vie française. Le risque financier y est intégralement porté par l'assureur, qui s'appuie sur une politique d'investissement prudente, majoritairement orientée vers des actifs obligataires. La sécurité de ce support repose sur un ensemble de garanties contractuelles et réglementaires :
\begin{itemize}
    \item \textbf{La garantie du capital :} C'est la garantie la plus fondamentale. L'assureur garantit à tout moment le capital net investi par l'épargnant. Quelle que soit l'évolution des marchés financiers, la somme initialement versée (nette de frais) ne peut pas diminuer.
    \item \textbf{Le taux technique :} Il s'agit d'un taux de revalorisation minimal garanti sur toute la durée du contrat. Fixé à la souscription, il est aujourd'hui très faible, voire nul, en raison des contraintes réglementaires.
    \item \textbf{Le Taux Minimum Garanti (TMG) :} Plus courant aujourd'hui que le taux technique, le TMG est un taux de rendement minimal que l'assureur s'engage à verser pour l'année à venir. Il est fixé annuellement et permet à l'assureur d'ajuster sa politique de rendement.
    \item \textbf{L'effet cliquet :} Ce mécanisme assure que les intérêts générés chaque année sont définitivement acquis. Une fois distribués, ils s'ajoutent au capital garanti et produisent à leur tour des intérêts les années suivantes. Il est impossible de revenir sur les revalorisations passées.
    \item \textbf{La Participation aux Bénéfices (PB) :} L'assureur a l'obligation légale de redistribuer aux assurés au minimum 85\% de ses bénéfices financiers et 90\% de ses bénéfices techniques. Cette participation constitue la majeure partie du rendement annuel, au-delà du TMG. Pour lisser les performances, une partie de cette PB peut être mise en réserve dans une \textit{Provision pour Participation aux Bénéfices} (PPB) que nous appelerons \textit{Provision pour Participations aux Excedents} (PPE) dans la suite de ce mémoire. La PPE doit être reversée aux assurés dans un délai de huit ans au maximum.
\end{itemize}


Les \textbf{unités de compte (UC)} offrent une exposition directe aux marchés financiers. Contrairement au fonds en euros, le risque d'investissement est entièrement porté par l'assuré. L'assureur ne garantit pas la valeur du capital, mais un nombre de parts d'actifs (OPCVM, actions, SCPI, etc.). La valeur de l'épargne fluctue ainsi au gré des marchés, offrant un potentiel de rendement supérieur à long terme, mais exposant également à un risque de perte en capital. Pour l'assureur, ce support est plus simple à gérer car il n'implique pas de garanties financières particulières.


Ces supports sont proposés via deux grandes familles de contrats. Les contrats monosupports permettent d'investir sur un seul type de fonds (soit en euros, soit en UC). Les contrats multisupports sont les plus répandus, quant à eux, combinent au moins un fonds en euros et plusieurs supports en unités de compte, permettant à l'épargnant de répartir son investissement selon son profil de risque. Dans le cadre de cette étude, le portefeuille analysé se compose de contrats multisupports avec une répartition représentative du marché français, soit approximativement 60\% en fonds euros et 40\% en unités de compte.
\begin{figure}[H]
    \centering
    \includegraphics[width=0.8\textwidth]{images/2_chapitres/chapitre1/assurance-vie-stats.png}
    \caption{Le marché de l'assurance vie en France en 2024 \textbf{(graphique temporaire)}}
    \label{fig:marches_assurance_vie}
\end{figure}

Ces contrats sont soumis à un ensemble de réglementations prudentielles visant à garantir la solvabilité des assureurs et la protection des assurés. La directive Solvabilité II encadre ces exigences à l'échelle européenne. La partie suivante détaille le cadre réglementaire de Solvabilité II, ses implications pour les assureurs vie, ainsi que les outils de modélisation stochastique utilisés pour répondre à ces exigences.

\section{Le cadre prudentiel Solvabilité II}
\label{sec:s2}

Entrée en vigueur le 1er janvier 2016, la directive Solvabilité II régit le cadre prudentielle pour la quasi-totalité des assureurs et réassureurs de l'Union Européenne. Son principal objectif est d'harmoniser les pratiques du secteur, d'assurer une protection optimale des assurés et de garantir que les compagnies puissent honorer leurs engagements en toutes circonstances. Pour ce faire, elle instaure une approche économique et prospective, fondée sur une évaluation fine des risques et structurée en trois piliers interdépendants. Seuls les deux premiers piliers seront présentés car ils sont plus pertinents dans le cadre de ce mémoire.


\subsection{Le Pilier 1 : Exigences quantitatives et Bilan Prudentiel}

Le Pilier 1 définit les exigences quantitatives, au cœur desquelles se trouve le Bilan Prudentiel. Il s'agit d'une vision économique du bilan comptable où les actifs et les passifs sont évalués de manière cohérente avec leur valeur de marché (\textit{market-consistent}). La structure de cette section suivra l'équation fondamentale du bilan prudentiel :
\begin{equation}
    \text{Actifs} = \text{Provisions Techniques} + \text{Fonds Propres}
\end{equation}
Chaque terme de cette équation sera détaillé successivement.

\begin{figure}[H]
    \centering
    % --- NOTE : L'image du bilan est placée ici pour introduire visuellement la structure qui sera suivie. ---
    \includegraphics[width=0.65\textwidth]{images/2_chapitres/chapitre1/bilanS2.png}
    \caption{Bilan économique sous Solvabilité II (graphique temporaire)}
    \label{fig:Bilan_economique_S2}
\end{figure}


\subsubsection{L'évaluation des Actifs à la Valeur de Marché}

Le premier terme du bilan, Actifs, sont comptabilisés à leur Valeur de Marché (VM). Cette approche vise à refléter la valeur la plus juste et actuelle des ressources dont dispose l'assureur pour couvrir ses engagements.


\subsubsection{Les Provisions Techniques : Cœur de l'évaluation du passif}
% --- NOTE : Cette sous-section regroupe maintenant tout ce qui concerne le calcul des engagements : BE, RM et les méthodes de valorisation. ---

Les Provisions Techniques (PT) représentent la valeur des engagements de l'assureur envers ses assurés. Elles se décomposent en deux parties : le \textit{Best Estimate} (BE) et la Marge de Risque (\textit{Risk Margin} - RM).
\begin{equation}
    \text{PT} = \text{Best Estimate (BE)} + \text{Marge de Risque (RM)}
\end{equation}
\bigskip


% --- NOTE : Les méthodes de valorisation sont maintenant une sous-partie logique du calcul des provisions techniques. ---
La valorisation des provisions techniques, au cœur du bilan prudentiel, ne peut se contenter d'une vision unique et figée du futur. La présence d'options et de garanties dans les contrats d'épargne impose de distinguer deux approches complémentaires :
\begin{itemize}
    \item \textbf{L'approche déterministe} est un outil de pilotage. Elle repose sur une projection unique des variables économiques. Bien qu'insuffisante pour la valorisation prudentielle, elle demeure un outil fondamental pour l'élaboration du \textit{business plan} et la communication d'un scénario central. Sa limite principale est son incapacité à valoriser les risques asymétriques.

    \item \textbf{L'approche stochastique} est un outil de valorisation. Elle explore un grand nombre de futurs possibles à l'aide d'un \textbf{Générateur de Scénarios Économiques (GSE)}. Cet outil produit des milliers de simulations cohérentes des marchés financiers. La valeur d'un indicateur est alors obtenue en calculant la moyenne des résultats sur l'ensemble de ces scénarios (méthode de Monte-Carlo). Cette exploration est indispensable pour quantifier le coût réel des garanties optionnelles (Taux Minimum Garanti, etc.).
\end{itemize}
La différence de valeur entre ces deux approches est capturée par le concept de \textbf{TVOG (\textit{Time Value of Options and Guarantees})}. En imposant une approche stochastique, Solvabilité II assure une valorisation \textit{market-consistent} des engagements.
\bigskip


Le \textbf{Best Estimate (BE)}, ou \textit{Best Estimate Liability} (BEL), représente la meilleure estimation de la valeur actuelle des flux de trésorerie futurs liés aux engagements d'assurance. Son calcul est réalisé sur un horizon long (40-60 ans) en \textit{run-off} (portefeuille en extinction, pas de nouvelles souscriptions). Il est obtenu par la moyenne des flux actualisés sur un grand nombre de simulations économiques stochastiques en univers risque neutre~:
\begin{equation}
    BEL = \mathbb{E}^{\mathbb{Q}} \left[ \sum_{j=1}^{T} CF(j) \cdot e^{-\int_0^j r(s)ds} \right] \approx \frac{1}{N}\sum_{i=1}^{N}\sum_{j=1}^{T}\frac{CF_{i}(j)}{(1+r_{i,j})^{j}}
\end{equation}
Où $N$ est le nombre de simulations, $T$ l'horizon de projection, $CF_{i}(j)$ le flux de trésorerie net de l'année $j$ pour la simulation $i$, et $r_{i,j}$ le taux d'actualisation sans risque pertinent.
\bigskip


La \textbf{Marge de Risque (RM)} complète le Best Estimate. Elle correspond à la rémunération du capital réglementaire qui doit être immobilisé pour couvrir les risques non-financiers (longévité, rachat, etc.) jusqu'à l'extinction du portefeuille. Son calcul repose sur une approche dite de "Coût du Capital" (\textit{Cost of Capital} - CoC), qui consiste à actualiser le coût futur de détention de ce capital :
\begin{equation}
    RM = \text{CoC}_{\text{rate}} \times \sum_{j=0}^{T} \frac{\text{SCR}_{\text{non-fi}}(j)}{(1+r_{j+1})^{j+1}}
\end{equation}
Où $\text{CoC}_{\text{rate}}$ est le coût du capital (fixé à 6\%), et $\text{SCR}_{\text{non-fi}}(j)$ est la part du SCR couvrant les risques non-financiers à l'année $j$.


\subsubsection{Les Fonds Propres et les Exigences de Capital}
% --- NOTE : Cette sous-section lie logiquement les Fonds Propres aux exigences de capital (SCR/MCR) qui portent sur eux. ---
\bigskip


Les \textbf{Fonds Propres Prudentiels}, aussi appelés \textbf{NAV (\textit{Net Asset Value})}, constituent la richesse de l'assureur. Ils sont définis par la différence entre la valeur des actifs et celle des engagements :
\begin{equation}
    NAV = VM_{Actifs} - (BE + RM)
\end{equation}
Ce sont ces fonds propres qui doivent permettre à l'assureur d'absorber des pertes inattendues. Solvabilité II définit donc deux niveaux d'exigence de capital portant sur la NAV.
\bigskip


Le \textbf{Solvency Capital Requirement (SCR)} est le montant de fonds propres nécessaire pour absorber des pertes exceptionnelles. Il est calibré pour correspondre à la \textbf{Value-at-Risk (VaR) à 99,5\%} de la NAV sur un horizon d'un an. Autrement dit, il s'agit du capital qui doit permettre à l'assureur de faire face à ses engagements sur l'année à venir avec une probabilité de 99,5\%. En cas de non-respect, l'assureur fait l'objet d'un suivi renforcé par le régulateur.

Le calcul du SCR peut se faire via un modèle interne (spécifique à l'assureur) ou, plus communément, via la \textbf{Formule Standard} définie par la réglementation. Cette dernière est une approche modulaire qui décompose le risque total en plusieurs modules et sous-modules (risque de marché, de souscription, de contrepartie, etc.).

Pour chaque risque élémentaire $x$, le capital requis est calculé comme la perte de NAV consécutive à un choc instantané et calibré sur ce risque :
\begin{equation}
    SCR_{x} = \Delta NAV = NAV_{\text{central}} - NAV_{\text{choc } x}
\end{equation}
Les SCR des différents modules sont ensuite agrégés en prenant compte des corrélations prédéfinies entre les risques. L'agrégation de deux modules de risque $i$ et $j$ se fait via la formule :
\begin{equation}
    SCR_{i,j} = \sqrt{SCR_i^2 + SCR_j^2 + 2 \times Corr_{i,j} \times SCR_i \times SCR_j}
\end{equation}
où $Corr_{i,j}$ est le coefficient de corrélation entre les risques $i$ et $j$ fourni par la réglementation. Cette agrégation est appliquée de manière hiérarchique pour obtenir le SCR total, appelé \textit{Basic Solvency Capital Requirement} (BSCR) :
\begin{equation}
    SCR_{total} = \sqrt{\sum_{i}\sum_{j} SCR_i \times SCR_j \times Corr_{i,j}}
\end{equation}

\begin{figure}[H]
    \centering
    \includegraphics[width=0.95\textwidth]{images/2_chapitres/chapitre1/pieuvre_scr.png}
    \caption{Schéma des modules et sous-modules du SCR en Formule Standard}
    \label{fig:pieuvre_scr}
\end{figure}

\subsection{Le Pilier 2 : Exigences qualitatives et gouvernance}

Ce pilier se concentre sur la supervision, la gestion des risques et la gouvernance interne. Il impose aux assureurs de mettre en place un système de gouvernance sain, prudent et proportionné. Cela inclut une structure organisationnelle transparente, des politiques écrites claires, et un système de contrôle interne robuste. La direction doit être assurée par au moins deux dirigeants effectifs (principe des 4 yeux) qui doivent répondre à des exigences de compétence et d'honorabilité (\textit{fit and proper}).

Ce système s'articule autour de quatre fonctions clés indépendantes : la fonction actuarielle, la gestion des risques, l'audit interne et la conformité.

L'élément central du Pilier 2 est l'\textbf{ORSA} (\textit{Own Risk and Solvency Assessment}). Il s'agit d'un processus interne par lequel l'assureur évalue, sur un horizon de 3 à 5 ans, l'adéquation entre son profil de risque spécifique, ses limites de tolérance et ses besoins globaux en solvabilité. C'est un outil de pilotage stratégique qui permet d'aller au-delà des hypothèses standards pour refléter la stratégie propre de l'entreprise.

\section{Les Générateurs de Scénarios Économiques (GSE)}
\label{sec:gse}

Le Générateur de Scénarios Économiques (GSE) est un outil mathématique central dans la modélisation stochastique. Il a pour fonction de simuler de multiples trajectoires futures pour les principales variables financières (taux d'intérêt, performance des actions, inflation, etc.). La qualité des projections ALM dépendant directement de la robustesse du GSE, il est nécessaire de distinguer deux cadres de modélisation qui coexistent.

Bien que ces deux univers soient complémentaires, la réglementation Solvabilité II assigne à chacun un rôle très précis pour le calcul des différents indicateurs prudentiels. Le tableau suivant synthétise cette répartition des tâches.

\begin{table}[H]
\centering
\caption{Répartition des calculs Solvabilité II par univers de projection}
\label{tab:s2_par_univers}
\begin{tabularx}{\textwidth}{>{\raggedright\arraybackslash}X >{\raggedright\arraybackslash}X}
\toprule
\textbf{\texorpdfstring{Univers Risque Neutre (Q)}{Univers Risque Neutre (Q)}} & \textbf{\texorpdfstring{Univers Monde Réel (P)}{Univers Monde Réel (P)}} \\
\midrule
\textbf{Indicateurs du Pilier 1 :}
\begin{itemize}[itemsep=2pt]
\item Best Estimate Liability (BEL)
\item Marge de Risque (RM)
\item Solvency Capital Requirement (SCR)
\item Bilan Prudentiel et NAV
\end{itemize}

\textbf{Exercices du Pilier 2 :}
\begin{itemize}[itemsep=2pt]
\item ORSA (Own Risk and Solvency Assessment)
\end{itemize}
&
\textbf{Exercices du Pilier 2 :}
\begin{itemize}[itemsep=2pt]
\item ORSA (Own Risk and Solvency Assessment)
\item Business Plan et planification stratégique
\item Test de la pérennité du modèle
\end{itemize} \\
\addlinespace
\textbf{Finalité : Valorisation} \textit{Market-Consistent} à un instant t.
&
\textbf{Finalité : Pilotage} stratégique et prospectif. \\
\bottomrule
\end{tabularx}
\end{table}

La distinction entre ces deux approches est donc fondamentale : l'une sert à valoriser, l'autre à piloter. Les sections suivantes détaillent les modèles mathématiques sous-jacents à chaque univers.

\subsection{\texorpdfstring{L'univers Risque Neutre ($\mathbb{Q}$)}{L'univers Risque Neutre (Q)} : un cadre pour la valorisation}

L'\textbf{univers Risque Neutre ($\mathbb{Q}$)} est un cadre de valorisation théorique, requis par Solvabilité II pour les calculs \textit{market-consistent}. Son objectif n'est pas de prédire l'évolution réelle des marchés, mais de calculer la valeur risque neutralisée d'un actif ou d'un passif à la date de calcul, en se fondant sur les prix de marché observés. Dans cet univers, on postule que tous les investisseurs sont indifférents au risque, ce qui implique que le rendement espéré de n'importe quel actif est égal au taux d'intérêt sans risque. Cette construction, fondée sur l'absence d'opportunité d'arbitrage, est indispensable pour valoriser de manière cohérente les options et garanties complexes des contrats d'assurance. La valeur $V_0$ d'un flux de trésorerie futur $CF_T$ est alors son espérance mathématique sous cette probabilité risque neutre, actualisée au taux sans risque $r(s)$ :
\begin{equation}
    V_0 = \mathbb{E}^{\mathbb{Q}} \left[ CF_T \cdot e^{-\int_0^T r(s)ds} \right]
    \label{eq:valeur_risque_neutre}
\end{equation}
Cet univers constitue le fondement du Pilier 1 de Solvabilité II, utilisé pour le calcul du Best Estimate Liability (BEL) et du Solvency Capital Requirement (SCR).

\subsubsection{Modélisation des taux d'intérêt : le modèle de Hull \& White}
Pour les taux d'intérêt, le modèle de \textbf{Hull \& White à un facteur} est une référence dans le cadre réglementaire. Son principal avantage est sa capacité à se calibrer parfaitement à la courbe des taux sans risque initiale, telle que fournie par l'EIOPA.
\begin{figure}[H]
    \centering
    \includegraphics[width=0.65\textwidth]{images/2_chapitres/chapitre1/courbe_EIOPA.png}
    \caption{Courbe des taux sans risque sans \textit{Volatility Adjustment} au 31/12/2024 publiée par l'EIOPA}
    \label{fig:courbe_EIOPA}
\end{figure}

Cette flexibilité est obtenue grâce à un paramètre de retour à la moyenne $\theta(t)$ qui dépend du temps. Son équation différentielle stochastique (EDS) s'écrit :
\begin{equation}
    dr_t = (\theta(t) - ar_t)dt + \sigma dW^{\mathbb{Q}}_t
    \label{eq:hull_white}
\end{equation}
où $r_t$ est le taux d’intérêt court, $a$ la vitesse de retour à la moyenne, $\sigma$ la volatilité et $W^{\mathbb{Q}}_t$ un mouvement brownien sous la mesure risque neutre. Pour des raisons de calcul, nous utilisons la solution discrète de cette EDS :
\begin{equation}
    r_{t+h} = r_t e^{-ah} + \theta(t+h) - \theta(t)e^{-ah} + \sigma \sqrt{\frac{1 - e^{-2ah}}{2a}} Z
    \label{eq:hull_white_discrete}
\end{equation}

\subsubsection{Modélisation des actions et de l'immobilier : le modèle de Black \& Scholes}
Pour les actifs risqués comme les actions ou l'immobilier, le modèle de \textbf{Black \& Scholes} est couramment utilisé. Conformément à la logique risque neutre, le rendement espéré (la dérive du processus) est le taux sans risque $r_t$. L'EDS du prix de l'actif $S_t$ est :
\begin{equation}
    dS_t = r_t S_t dt + \sigma S_t dW^{\mathbb{Q}}_t
    \label{eq:black_scholes_q}
\end{equation}
où $S_t$ est le prix de l'actif, $r_t$ le taux sans risque et $\sigma$ la volatilité de l'actif. La solution de cette EDS est donnée par :
\begin{equation}
    S_t = S_0 \exp\left( \int_0^t \left(r_s - \frac{\sigma^2}{2}\right)ds + \int_0^t \sigma dW^{\mathbb{Q}}_s \right)
\end{equation}
En pratique, on utilise sa solution discrétisée pour simuler les trajectoires de prix sur un pas de temps $h$ :
\begin{equation}
    S_{t+h} = S_t \exp\left( \left(r_t - \frac{\sigma^2}{2}\right)h + \sigma\sqrt{h}Z \right)
    \label{eq:black_scholes_q_discrete}
\end{equation}
où $Z$ est une variable aléatoire suivant une loi normale centrée réduite $\mathcal{N}(0,1)$.

\subsection{\texorpdfstring{L'univers Monde Réel ($\mathbb{P}$)}{L'univers Monde Réel (P)} : un outil de pilotage stratégique}

À l'inverse de l'univers risque neutre, l'\textbf{univers Monde Réel ($\mathbb{P}$)} vise à générer des scénarios réalistes pour refléter une évolution plausible des marchés. Son objectif est la projection et la planification stratégique, notamment pour l'exercice ORSA (Pilier 2). 

La différence fondamentale réside dans l'introduction d'une \textbf{prime de risque} pour rémunérer la volatilité supportée par les investisseurs. Le rendement espéré d'un actif risqué est donc supérieur au taux sans risque, calibré sur des données historiques et des anticipations d'experts :
\begin{equation}
    \mathbb{E}^{\mathbb{P}}[\text{Rendement de l'actif}] = \text{Taux sans risque} + \text{Prime de risque}
\end{equation}

Les modèles utilisés, bien que similaires dans leur forme à ceux de l'univers $\mathbb{Q}$ (par exemple, Vasicek pour les taux ou Black \& Scholes pour les actions), sont modifiés pour intégrer cette prime. La dérive du processus stochastique n'est plus le taux sans risque $r_t$, mais un rendement espéré monde réel $\mu$.En somme, si l'univers $\mathbb{Q}$ valorise les engagements à un instant $t$, l'univers $\mathbb{P}$ permet d'exprimer la situation financière de l'entreprise dans le futur, ce qui le rend indispensable pour le pilotage stratégique.

\subsection{Synthèse des deux univers}

Le tableau suivant résume les caractéristiques et les usages des deux univers de projection. Si l'univers risque neutre $\mathbb{Q}$ répond à la question « \textit{Combien vaut cet engagement aujourd'hui ?} », l'univers monde réel $\mathbb{P}$ répond à « \textit{Quelle sera ma situation financière demain ?} ».

\begin{table}[H]
    \centering
    \caption{Synthèse comparative des univers de projection}
    \label{tab:univers_s2_comp}
    \begin{tabularx}{\textwidth}{l >{\raggedright\arraybackslash}X >{\raggedright\arraybackslash}X}
        \toprule
        \textbf{Critère} & \textbf{\texorpdfstring{Univers Risque Neutre ($\mathbb{Q}$)}{Univers Risque Neutre (Q)}} & \textbf{\texorpdfstring{Univers Monde Réel ($\mathbb{P}$)}{Univers Monde Réel (P)}} \\
        \midrule
        \textbf{Objectif}
        &
        \textbf{Valorisation} \textit{Market-Consistent} (Pilier 1 : BEL, SCR). Calculer une valeur juste à $t=0$.
        &
        \textbf{Projection} stratégique (Pilier 2 : ORSA, Business Plan). Simuler des futurs plausibles. \\
        \addlinespace
        \textbf{Rendement Espéré (Actifs risqués)}
        &
        Taux sans risque ($r_t$). Aucune prime de risque.
        &
        Taux sans risque + Prime de risque ($\mu = r + \text{prime}$). \\
        \addlinespace
        \textbf{Modèle de Taux Typique}
        &
        \textbf{Hull \& White}. Flexible, calibré à la courbe des taux initiale.
        &
        \textbf{Vasicek}. Économique, retour à une moyenne de long terme. \\
        \addlinespace
        \textbf{Calibration}
        &
        Calibré sur les prix des instruments financiers \textbf{actuels} (courbe des taux, volatilités implicites).
        &
        Calibré sur des \textbf{données historiques} et des \textbf{anticipations d'experts} (primes de risque). \\
        \bottomrule
    \end{tabularx}
\end{table}

Pour la suite de ce mémoire, je vais me concentrer sur l'univers risque neutre $\mathbb{Q}$, car il est le plus pertinent pour les calculs prudentiels et la gestion Actif-Passif dans le cadre des calculs liés au pilier 1 de Solvabilité II.


\section{La représentation du passif : le concept de \textit{Model Point}}
\label{sec:mp}

\subsection{Enjeux de l'agrégation et réduction de la dimensionnalité}

Les portefeuilles d'assurance vie comptent souvent des centaines de milliers, voire des millions de polices. Une modélisation "police à police" est techniquement possible mais informatiquement très chronophage, voire irréalisable pour des calculs stochastiques complexes comme ceux requis par les modèles ALM, où chaque contrat doit être projeté sur des milliers de scénarios économiques.

La charge de calcul deviendrait excessive. La simplification du portefeuille de passif n'est donc pas un choix, mais une contrainte opérationnelle majeure. La réponse standard à cette contrainte est la création de \textbf{\textit{Model Points}} (MP). Un MP est un contrat synthétique représentant un agrégat de polices partageant des caractéristiques homogènes (âge, ancienneté, type de produit, TMG). L'objectif est de réduire drastiquement le volume de données à traiter tout en préservant les propriétés actuarielles et financières essentielles du portefeuille complet.

\subsection{Cadre réglementaire et exigences de validation}

Si l'agrégation est une nécessité opérationnelle, elle constitue également un risque de modèle. Le remplacement d'un portefeuille réel par une image simplifiée introduit inévitablement un biais d'estimation. Dans le cadre prudentiel de Solvabilité II, ces simplifications sont encadrées strictement.

\subsubsection{Le principe de matérialité et justification des approximations}

La directive Solvabilité II et les textes de l'ACPR imposent que les méthodes de simplification soient \og appropriées \fg{} et ne conduisent pas à une sous-estimation significative des risques.
En particulier, la \textit{Notice ACPR sur les modèles internes} (2023) \cite{notice_acpr} rappelle dans son paragraphe 21 que l'entreprise doit \og identifier, documenter et justifier la pertinence des simplifications, des approximations et des hypothèses \fg  utilisées. Bien que ce texte vise les modèles internes, il établit un standard de place : l'erreur introduite par l'agrégation doit rester immatérielle au regard du Best Estimate Liability (BEL) et du Capital de Solvabilité Requis (SCR).

Les autorités de contrôle (ACPR, EIOPA) insistent sur deux points majeurs :
\begin{itemize}
    \item \textbf{La granularité adaptée (Segmentation) :} Comme précisé dans le paragraphe 38 de la notice, la granularité des facteurs de risque et la segmentation doivent être justifiées en analysant les avantages et les inconvénients. Une segmentation trop grossière (ex: moyenne des âges sur tout le portefeuille) écraserait la convexité des engagements et sous-estimerait la valeur des options et garanties.
    \item \textbf{La validation quantitative :} L'entreprise doit démontrer, chiffres à l'appui, que le passage du "réel" au "modèle" n'altère pas significativement les indicateurs clés (BEL, SCR).
\end{itemize}

\subsection{Dispositif de contrôle de la qualité de l'agrégation}

Pour répondre à ces exigences, la construction des \textit{Model Points} doit s'accompagner d'un dispositif de validation robuste, intégré à la gouvernance des risques et à l'exercice ORSA. Ce dispositif repose généralement sur trois niveaux de contrôle.

\subsubsection{Tests de reproduction des métriques statiques (Reproduction)}
Il s'agit de vérifier, à la date d'arrêté, que le portefeuille agrégé (MP) reproduit fidèlement les stocks du portefeuille réel. Les écarts relatifs sont mesurés sur des indicateurs clés tels que :
\begin{itemize}
    \item La Provision Mathématique (PM) totale et par segment ;
    \item Les encours moyens, l'âge moyen pondéré des assurés, l'ancienneté moyenne ;
    \item La répartition par Taux Minimum Garanti (TMG).
\end{itemize}
Des seuils de tolérance stricts (par exemple $<0.1\%$ d'écart sur la PM) sont définis par l'assureur pour valider cette étape.

\subsubsection{Contrôles du profil de risque et des queues de distribution}
L'agrégation ne doit pas "lisser" les risques extrêmes. Un point d'attention particulier, souligné par les contrôles prudentiels (paragraphe 39 de la notice ACPR), est la bonne représentation des "queues de distribution".
Par exemple, un portefeuille contenant quelques très gros contrats ou des contrats très anciens avec des TMG très élevés (4.5\%) ne doit pas voir ces risques dilués dans une moyenne à 0\% comme c'est le cas pour les contrats récents. Ces profils atypiques nécessitent souvent la création de \textit{Model Points} dédiés pour isoler le risque.

\subsubsection{Tests de sensibilité et stabilité dynamique}
La validation la plus critique concerne le comportement du portefeuille agrégé sous contrainte (chocs). Un regroupement peut sembler correct en vision statique (t=0) mais diverger lors des projections.
Des tests de sensibilité sont réalisés en comparant les résultats du modèle "Police à Police" (sur un scénario déterministe) et du modèle agrégé :
\begin{itemize}
    \item \textbf{Sensibilité aux taux :} Comportement du BEL suite des chocs de taux d'intérêt (hausse, baisse) ;
    \item \textbf{Sensibilité aux rachats :} Impact des chocs sur les taux de rachat (hausse ou baisse) ;
    \item \textbf{Sensibilité à la mortalité :} Impact des chocs sur les tables de mortalité.
\end{itemize}
L'objectif est de s'assurer que l'agrégation conserve la convexité du passif et la sensibilité aux variables économiques, garantissant ainsi un calcul correct du SCR.

\subsection{Problématique du mémoire}

C'est dans ce cadre réglementaire et technique que s'inscrit ce mémoire. La méthode d'agrégation choisie a un impact direct sur la qualité des indicateurs S2 produits.
En partant du portefeuille granulaire, nous chercherons à définir et comparer différentes méthodes de regroupement (notamment par \textit{Clustering}) pour construire des \textit{Model Points}. L'objectif est d'identifier la méthodologie qui offre le meilleur compromis entre temps de calcul (réduction dimensionnelle) et fidélité prudentielle (stabilité du SCR et des sensibilités), conformément aux attentes du régulateur.


\section{Gestion Actif-Passif et Architecture du Modèle de Projection}
\label{sec:alm_modele}

La Gestion Actif-Passif (ALM) est la réponse opérationnelle à l'inadéquation structurelle inhérente au cycle de production inversé de l'assurance vie. Cette discipline vise à piloter les risques nés de l'interdépendance entre :
\begin{itemize}
    \item \textbf{Le Passif :} Des engagements longs, sensibles aux taux (TMG) et aux comportements (rachats).
    \item \textbf{L'Actif :} Des placements volatils dont la performance finance la revalorisation des assurés.
\end{itemize}

Pour quantifier ces risques et répondre aux exigences de Solvabilité II, il est nécessaire de projeter sur un long horizon de projection le bilan de l'assureur sur un grand nombre de scénarios économiques. Cette section détaille les mécanismes de cette projection et l'architecture du modèle développé en Python pour ce mémoire.

\subsection{Mécanismes de projection et interactions Actif-Passif}

La projection repose sur la modélisation des flux financiers et des règles de gestion qui lient l'actif et le passif.

\subsubsection*{L'approche Stochastique vs Déterministe}
Si une approche déterministe (un seul scénario central) suffit pour construire un \textit{Business Plan}, elle est incapable de capturer le coût des options et garanties financières (TMG, option de rachat).
L'approche stochastique est donc indispensable. Elle consiste à simuler des milliers de trajectoires économiques (via le GSE présenté en section \ref{sec:gse}) et à calculer la moyenne des résultats (méthode de Monte-Carlo). C'est cette distribution des résultats qui permet d'estimer la \textit{Value-at-Risk} et donc le SCR.

\subsubsection*{Le mécanisme de la Participation aux Bénéfices (PB)}
L'interaction centrale du modèle est la distribution de la performance financière. Le Code des assurances impose une redistribution minimale, mais le modèle doit simuler la stratégie discrétionnaire de l'assureur.
La règle de partage modélisée suit l'algorithme suivant :
\begin{enumerate}
    \item Calcul du résultat financier (produits financiers nets de charges).
    \item Détermination de l'assiette de PB réglementaire.
    \item Application de la politique commerciale (Taux cible concurrentiel).
    \item Dotation ou reprise à la Provision pour Participation aux Excédents (PPE) pour lisser les rendements dans le temps.
\end{enumerate}

\subsection{Algorithmes de projection du passif et valorisation du Best Estimate}
\label{subsec:algo_passif_be}

Contrairement aux modèles traditionnels procédant par itération sur chaque contrat ("boucle ligne à ligne"), ce modèle exploite la vectorisation pour projeter simultanément l'ensemble des \textit{Model Points} sur tous les scénarios.

Cette approche permet de traiter des volumes de données massifs en définissant l'espace de projection comme le produit cartésien :
\[ \Omega = \{MP\} \times \{Annees\} \times \{Scenarios\} \times \{Chocs_{S2}\} \]

Les paragraphes suivants détaillent la mécanique actuarielle de projection sur un pas de temps annuel $[t, t+1]$.

\subsubsection*{Algorithme de projection des flux de passif}

La projection des engagements s'effectue selon une logique de flux en fin de période, adaptée pour simuler la survenance continue des événements. Pour chaque année projetée $t$, l'évolution de la Provision Mathématique ($PM$) suit la séquence d'opérations suivante :

\begin{enumerate}
    \item \textbf{Vieillissement et mise à jour des statuts :}
    L'âge des assurés $x$ devient $x+1$ et l'ancienneté des contrats est incrémentée. Les contrats arrivant à terme sont sortis du stock et le capital constitutif est versé.

    \item \textbf{Calcul des taux de sortie (Décréments) :}
    Le modèle applique les tables biométriques et comportementales pour déterminer les probabilités de sortie :
    \begin{itemize}
        \item \textbf{Mortalité ($q_{x+t}$) :} Application des tables réglementaires ou d'expérience (par génération ou par âge atteint).
        \item \textbf{Rachats ($\alpha_t$) :} Application du taux de rachat total lié à l'ancienneté du contrat. \textcolor{blue}{(Les rachats conjoncturels ne sont pas modélisés dans la version actuelle du modèle, mais pourraient être intégrés dans une version future pour simuler des comportements plus réalistes en fonction des conditions de marché).}
    \end{itemize}

    \item \textbf{Valorisation intermédiaire et calcul des prestations :}
    Afin de simuler une distribution uniforme des décès et des rachats au cours de l'année, le modèle calcule une PM revalorisée à mi-période ($t+0.5$) sur laquelle sont appliqués les taux de sortie.
    \[ PM_{t+0.5} = PM_{t} \times (1 + TMG)^{0.5} \]
    Les flux de prestations ($P_{MP,t}$) versées sur un Model Point l'année t sont alors déduits :
    \begin{equation}
        P_{MP,t} = \underbrace{PM_{MP,t+0.5} \times q_{MP,x+t}}_{\text{Prestations Décès}} + \underbrace{PM_{MP,t+0.5} \times (1-q_{MP,x+t}) \times \alpha_{MP,t}}_{\text{Prestations Rachat}}
    \end{equation}
    Cette approche assure la cohérence entre la valeur de l'épargne acquise au moment du départ et les montants effectivement versés.

    \item \textbf{Prélèvement des frais et capitalisation du stock restant :}
    Les chargements ($Ch_{MP,t}$) et frais de gestion ($Fg_{MP,t}$) sont prélevés sur l'encours. Le stock restant continue de capitaliser au TMG sur la seconde moitié de l'année. La PM d'un Model Point avant participation aux bénéfices s'écrit alors :
    \begin{equation}
        PM_{MP,t+1}^{\text{avant PB}} = PM_{MP,t} \times (1 + TMG_{MP}) - P_{MP,t} \times (1 + TMG_{MP})^{0.5} - (Ch_{MP,t} + Fg_{MP,t})
    \end{equation}

    \item \textbf{Revalorisation finale (Participation aux Bénéfices) :}
    En fin d'année, le module ALM détermine le taux de revalorisation final ($Tx_{servi}$). Si la performance de l'actif le permet et que la stratégie l'exige, une Participation aux Bénéfices additionnelle ($PB_{MP,t}$) est injectée sur le Model Point :
    \[ PB_{MP,t} = PM_{MP,moy} \times \max(0, Tx_{servi} - TMG_{MP}) \]
    La Provision Mathématique finale du Model Point pour l'année $t+1$ est donc :
    \begin{equation}
        PM_{MP,t+1} = PM_{MP,t+1}^{\text{avant PB}} + PB_{MP,t}
    \end{equation}
\end{enumerate}

\subsubsection*{Méthodologie de calcul du Best Estimate (BE)}

Conformément aux exigences de Solvabilité II, le \textit{Best Estimate} correspond à la moyenne des valeurs actuelles des flux de trésorerie futurs, pondérée par leur probabilité de réalisation.

\paragraph{Définition des Flux Nets de Trésorerie ($CF_t$)}
Pour chaque pas de temps $t$ et chaque scénario, le modèle agrège les flux entrants et sortants du point de vue de l'assureur :
\begin{equation}
    CF_t = \text{Prestations}_{t} + \text{Frais de Gestion}_{t} - \text{Primes}_{t}
\end{equation}
À noter que les chargements prélevés sur l'épargne ne constituent pas des flux de trésorerie (ils restent au sein de l'entreprise), mais viennent diminuer la dette envers les assurés.

\paragraph{Actualisation et Agrégation}
L'actualisation repose sur l'utilisation de \textbf{déflateurs stochastiques} $D(0,t)$, fournis par le GSE en cohérence avec la courbe des taux sans risque. Le déflateur intègre intrinsèquement le facteur d'actualisation et la mesure de probabilité risque-neutre $\mathbb{Q}$.

Le Best Estimate est obtenu par la moyenne arithmétique (méthode de Monte-Carlo) sur l'ensemble des $N$ scénarios simulés :

\begin{equation}
    BE = \frac{1}{N} \sum_{s=1}^{N} \left( \sum_{t=1}^{T} CF_{s,t} \times D_{s}(0,t) \right)
\end{equation}

Cette approche par déflateurs permet de valoriser de manière cohérente les options et garanties financières (l'effet "cliquet" des Taux Minimums Garantis) qui ne seraient pas capturées par une simple actualisation déterministe.

\subsection{Limites actuelles du modèle}
À date, bien que le modèle projette finement les provisions techniques et les actifs, la dynamique des \textbf{Fonds Propres} n'est pas entièrement bouclée. Cette limitation empêche le calcul direct d'un ratio de solvabilité projeté, mais n'impacte pas la qualité des sensibilités sur le Best Estimate et le SCR, qui constituent le cœur de notre étude.

% \chapter{La Gestion Actif-Passif et développement d'un modèle ALM en Python}
% \section{La gestion Actif-Passif (ALM) : définitions et enjeux}
% \label{sec:alm}

% \subsection{Le cycle de production inversé, origine de l'inadéquation Actif-Passif}

% La gestion Actif-Passif (ALM) trouve son origine dans une particularité fondamentale du secteur de l'assurance que l'on a énoncé dans le début de ce chapitre : le \textbf{cycle de production inversé}. Contrairement à une entreprise classique qui vend un produit avant d'en percevoir le revenu, un assureur collecte des primes aujourd'hui en échange de la promesse de verser des prestations dans un futur lointain et incertain. Ce décalage temporel est au cœur du modèle économique de l'assurance vie.

% Ce mécanisme engendre une inadéquation structurelle (\textit{mismatch}) entre les deux côtés du bilan. D'une part, le passif est constitué d'engagements de longue durée, dont l'échéance et le montant sont soumis à des aléas (mortalité, comportement de rachat des assurés). D'autre part, pour couvrir ces engagements, l'assureur investit les primes sur les marchés financiers, constituant un actif dont la valeur et les flux sont, par nature, volatiles et dépendants du contexte économique.

% Cette inadéquation est renforcée par une interdépendance dynamique et complexe entre l'actif et le passif.

% \begin{itemize}
%     \item \textbf{Le passif influe sur l'actif :} Le versement des prestations (rachats, décès) contraint l'assureur à liquider une partie de ses actifs, parfois dans des conditions de marché défavorables.
%         \item \textbf{L'actif influe sur le passif :} La performance des actifs financiers a un impact direct sur le niveau des engagements. C'est notamment le cas de la \textbf{Participation aux Bénéfices (PB)}, qui dépend des résultats financiers générés par l'assureur. Le Code des assurances impose une redistribution minimale aux assurés, calculée comme suit :
%         \begin{equation}
%             \label{eq:pb_minReg}
%             \text{PB}_{\text{minReg}} = 85\% \times \max(\text{RésFi}, 0) + 
%             \begin{cases}
%                 90\% \times \text{RésTech} & \text{si RésTech} \ge 0 \\
%                 100\% \times \text{RésTech} & \text{si RésTech} < 0
%             \end{cases}
%         \end{equation}
%         Où le Résultat Financier (RésFi) est directement issu de la performance des actifs et le Résultat Technique (RésTech) des risques de mortalité et de rachat.
% \end{itemize}
% La gestion Actif-Passif est donc la discipline qui vise à piloter les risques nés de cette interdépendance afin d'assurer la solvabilité et d'optimiser la rentabilité de l'acteur.

% \subsection{Objectifs et périmètre de l'ALM}

% % À COMPLÉTER :
% % Expliquer ici les trois grands objectifs de l'ALM :
% % 1. Maîtriser les risques (taux, marché, liquidité...).
% % 2. Assurer la solvabilité réglementaire (respect du SCR).
% % 3. Optimiser la rentabilité (maximiser le rendement pour un risque donné).


% \subsection{Le modèle ALM : un outil de simulation prospective}

% % À COMPLÉTER :
% % Décrire ici le rôle du modèle ALM comme outil de prise de décision.
% % - Principe : simulation conjointe de l'actif et du passif sur le long terme.
% % - Inputs principaux : GSE, lois de comportement, portefeuille, règles de gestion.
% % - Mentionner que vous utilisez un tel outil dans le cadre de votre mémoire, sans pour autant faire une étude ALM complète.


% \subsection{Principaux indicateurs de pilotage ALM}

% % À COMPLÉTER :
% % Lister et décrire brièvement quelques indicateurs clés issus des modèles ALM.
% % - Projection de la NAV (fonds propres prudentiels).
% % - Évolution du ratio de couverture SCR.
% % - Analyses de sensibilité et stress tests (impact d'un choc de taux, actions, etc.).

% La réalisation des analyses de sensibilités, cœur de ce mémoire, s'appuie sur un moteur de projection robuste : le modèle ALM (Asset Liability Management). Historiquement, le modèle utilisé au sein d'Accenture reposait sur une architecture en VBA (Visual Basic for Applications). Pour des impératifs de performance, de maintenabilité et de flexibilité, la décision a été prise de développer un nouveau modèle en Python. Cette migration a été l'occasion de repenser l'architecture du modèle pour mieux répondre aux exigences de calcul complexes de Solvabilité 2.

% Ce chapitre décrit ce nouveau modèle ALM. Nous présenterons son architecture globale, son fonctionnement détaillé à travers la projection des actifs, des passifs et la mise en œuvre des stratégies de gestion, et nous conclurons sur ses limites actuelles qui ouvrent des perspectives d'amélioration.

% \section{Développement du Modèle ALM en Python}

% La transition d'un modèle ALM de VBA vers Python a été motivée par la recherche de performance, de maintenabilité et de modularité. Le langage Python, avec son écosystème de librairies scientifiques comme \textit{NumPy}, \textit{Pandas} ou \textit{Polars}, offre un cadre de développement beaucoup plus robuste et performant pour des calculs actuariels intensifs.

% \subsection{Présentation du modèle ALM développé pour Accenture}

% Le modèle ALM a pour objectif principal de projeter l'ensemble des actifs et des passifs d'un assureur sur un horizon de long terme (typiquement 40 à 60 ans), sous un grand nombre de scénarios économiques stochastiques. Ces projections permettent de calculer les indicateurs prudentiels requis par la directive Solvabilité 2, notamment le BE et le SCR. Le modèle se veut un outil de pilotage stratégique, capable de simuler l'impact de différentes stratégies de gestion d'actifs, de politiques de souscription ou de participation aux bénéfices.

% \subsection{Fonctionnement du modèle ALM}

% \subsubsection{Fonctionnement général du modèle}
% Le modèle opère de manière itérative, année par année, pour chaque scénario économique. À chaque pas de temps, il simule l'ensemble des flux financiers et des opérations de bilan. Le processus global peut être résumé comme suit :
% \begin{enumerate}
% \item \textbf{Entrées :} Le modèle prend en entrée le portefeuille de passifs (issu du générateur ou d'un client), le portefeuille d'actifs, un set de scénarios économiques (ESG), et les règles de gestion (stratégie d'investissement, politique de PB, etc.).
% \item \textbf{Moteur de projection :} Pour chaque année et chaque scénario, le moteur calcule les flux de passifs, les flux d'actifs, et applique les décisions de gestion.
% \item \textbf{Sorties :} En fin de projection, le modèle génère des comptes de résultat et des bilans prévisionnels pour chaque scénario, qui sont ensuite utilisés pour calculer les indicateurs S2 par agrégation et analyse statistique.
% \end{enumerate}

% \subsubsection{Fonctionnement du passif}
% La projection du passif consiste à simuler l'évolution du portefeuille de contrats. À chaque pas de temps, le modèle calcule :
% \begin{itemize}
% \item Les primes encaissées.
% \item Les prestations versées (décès, rachats, rentes).
% \item Les chargements et frais prélevés.
% \item L'évolution des provisions mathématiques, en tenant compte de la revalorisation issue de la participation aux bénéfices.
% \end{itemize}
% Les flux de prestations sont déterminés par l'application des lois de comportement (mortalité, rachat) sur le portefeuille des assurés survivants.

% \subsubsection{Fonctionnement de l'actif}
% Simultanément, le modèle projette la valeur du portefeuille d'actifs. Pour chaque classe d'actifs (actions, obligations, immobilier, etc.), il calcule :
% \begin{itemize}
% \item L'évolution de la valeur de marché, en fonction des indices fournis par le scénario économique.
% \item Les revenus générés (dividendes, coupons, loyers).
% \end{itemize}
% Le modèle gère également le réinvestissement des flux de trésorerie et les opérations d'achat/vente décidées par la stratégie d'investissement.

% \subsubsection{Fonctionnement de la stratégie d'investissement}
% La stratégie d'investissement est un ensemble de règles qui dictent la manière dont l'actif est géré. Le modèle implémente une allocation stratégique cible (\textit{Strategic Asset Allocation} - SAA). Chaque année, il compare l'allocation réelle du portefeuille à l'allocation cible et déclenche des opérations d'achat ou de vente pour réduire l'écart, dans le respect des contraintes de liquidité et de transaction.

% \subsubsection{Fonctionnement de la stratégie ALM et de la politique de PB}
% Le cœur du modèle ALM est l'interaction entre l'actif et le passif. Le résultat financier généré par l'actif est utilisé pour déterminer la revalorisation servie aux assurés. La politique de Participation aux Bénéfices (PB) est une fonction clé qui répartit la performance financière entre l'assureur et les assurés, dans le respect des engagements contractuels et réglementaires. Le modèle simule la constitution et la reprise de la Provision pour Participation aux Bénéfices (PPB), qui permet de lisser les taux servis dans le temps.

% \subsection{Limites actuelles du modèle}
% Malgré sa robustesse, le modèle actuel présente certaines limites. Les règles de gestion, notamment la stratégie d'investissement, sont encore modélisées de manière relativement statique et ne réagissent pas toujours de façon dynamique aux conditions de marché extrêmes. De plus, la granularité de certains modules pourrait être affinée, notamment en ce qui concerne la modélisation des frais ou des impôts. Enfin, bien que les performances aient été grandement améliorées par rapport à la version VBA, les temps de calcul pour des portefeuilles très volumineux sur des milliers de scénarios restent un défi et une piste d'optimisation continue.