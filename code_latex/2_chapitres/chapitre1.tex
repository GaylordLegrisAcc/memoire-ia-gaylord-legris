\chapter{Contexte réglementaire et modélisation en assurance vie}
\label{chap:contexte}
\newpage
\section{Les spécificités des produits d'assurance vie épargne}
\label{sec:spec_av}

\subsection{Principes fondamentaux du contrat d'assurance vie}

L'assurance vie est une convention par laquelle un assureur, en contrepartie du versement de primes, s'engage à verser un capital ou une rente à la survenance d'un événement incertain lié à la durée de la vie humaine. Cet événement, qui constitue l'aléa au cœur du contrat, peut être le décès de l'assuré avant une date donnée ou, à l'inverse, sa survie jusqu'à cette date. Ce mécanisme repose sur un cycle de production inversé : l'assureur perçoit les primes bien avant de devoir potentiellement régler les prestations, ce qui l'amène à investir ces sommes sur des horizons de temps longs pour honorer ses engagements futurs.

\begin{figure}[H]
    \centering
    \includegraphics[width=0.8\textwidth]{images/2_chapitres/chapitre1/cycle-de-production.jpg}
    \caption{Cycle de production inversé en assurance vie}
    \label{fig:cycle_production_inverse}
\end{figure}

La nature de ces engagements répond à des objectifs variés. Les contrats en cas de vie prévoient le versement d'un capital ou d'une rente à une échéance prévue si l'assuré est en vie ; ils sont typiquement utilisés pour se constituer un complément de retraite ou une épargne de précaution. À l'opposé, les contrats en cas de décès garantissent le versement d'un capital ou d'une rente au(x) bénéficiaire(s) désigné(s) au décès de l'assuré, souvent pour protéger des proches ou anticiper des droits de succession. Il existe également des contrats mixtes qui combinent ces deux garanties.


Le fonctionnement de ces contrats repose sur la capitalisation : les primes versées sont investies pour financer la propre couverture future de l'assuré. De par leur nature, ces engagements s'étendent sur de très longues périodes. Une caractéristique fondamentale de l'assurance vie française est sa liquidité. L'assuré dispose de la possibilité de récupérer son épargne à tout moment via un rachat, qui peut être partiel ou total. Cette faculté de rachat constitue une option dont la valeur et le risque doivent être finement gérés par l'assureur, car son exercice a un impact direct sur les besoins de liquidité du portefeuille. La fiscalité joue un rôle incitatif majeur car les plus-values sont imposées plus lourdement si le rachat intervient avant la huitième année du contrat. Ceci encourage alors l'épargne de long terme.

La gestion de ces engagements de long terme amène l'assureur à proposer différentes modalités d'investissement. Celles-ci permettent de répartir le risque financier entre l'assuré et l'assureur, définissant ainsi le profil de rendement potentiel du contrat. Un contrat plus sûr aura des possibilités de rendements plus faible qu'un contrat risqué. L'épargne des assurés peut ainsi être investie sur deux principaux types de supports aux profils de risque distincts, qui peuvent être combinés au sein de différents types de contrats.

\subsection{Les principaux supports d'investissement}

L'épargne des assurés peut être investie sur deux principaux types de supports aux profils de risque distincts, qui peuvent être combinés au sein de différents types de contrats.



Le \textbf{fonds en euros} est le support historique et sécuritaire de l'assurance vie française. Le risque financier y est intégralement porté par l'assureur, qui s'appuie sur une politique d'investissement prudente, majoritairement orientée vers des actifs obligataires. La sécurité de ce support repose sur un ensemble de garanties contractuelles et réglementaires :
\begin{itemize}
    \item \textbf{La garantie du capital :} C'est la garantie la plus fondamentale. L'assureur garantit à tout moment le capital net investi par l'épargnant. Quelle que soit l'évolution des marchés financiers, la somme initialement versée (nette de frais) ne peut pas diminuer.
    \item \textbf{Le taux technique :} Il s'agit d'un taux de revalorisation minimal garanti sur toute la durée du contrat. Fixé à la souscription, il est aujourd'hui très faible, voire nul, en raison des contraintes réglementaires.
    \item \textbf{Le Taux Minimum Garanti (TMG) :} Plus courant aujourd'hui que le taux technique, le TMG est un taux de rendement minimal que l'assureur s'engage à verser pour l'année à venir. Il est fixé annuellement et permet à l'assureur d'ajuster sa politique de rendement.
    \item \textbf{L'effet cliquet :} Ce mécanisme assure que les intérêts générés chaque année sont définitivement acquis. Une fois distribués, ils s'ajoutent au capital garanti et produisent à leur tour des intérêts les années suivantes. Il est impossible de revenir sur les revalorisations passées.
    \item \textbf{La Participation aux Bénéfices (PB) :} L'assureur a l'obligation légale de redistribuer aux assurés au minimum 85\% de ses bénéfices financiers et 90\% de ses bénéfices techniques. Cette participation constitue la majeure partie du rendement annuel, au-delà du TMG. Pour lisser les performances, une partie de cette PB peut être mise en réserve dans une \textit{Provision pour Participation aux Bénéfices} (PPB) que nous appelerons \textit{Provision pour Participations aux Excedents} (PPE) dans la suite de ce mémoire. La PPE doit être reversée aux assurés dans un délai de huit ans au maximum.
\end{itemize}


Les \textbf{unités de compte (UC)} offrent une exposition directe aux marchés financiers. Contrairement au fonds en euros, le risque d'investissement est entièrement porté par l'assuré. L'assureur ne garantit pas la valeur du capital, mais un nombre de parts d'actifs (OPCVM, actions, SCPI, etc.). La valeur de l'épargne fluctue ainsi au gré des marchés, offrant un potentiel de rendement supérieur à long terme, mais exposant également à un risque de perte en capital. Pour l'assureur, ce support est plus simple à gérer car il n'implique pas de garanties financières particulières.


Ces supports sont proposés via deux grandes familles de contrats. Les contrats monosupports permettent d'investir sur un seul type de fonds (soit en euros, soit en UC). Les contrats multisupports sont les plus répandus, quant à eux, combinent au moins un fonds en euros et plusieurs supports en unités de compte, permettant à l'épargnant de répartir son investissement selon son profil de risque. Dans le cadre de cette étude, le portefeuille analysé se compose de contrats multisupports avec une répartition représentative du marché français, soit approximativement 60\% en fonds euros et 40\% en unités de compte.
\begin{figure}[H]
    \centering
    \includegraphics[width=0.8\textwidth]{images/2_chapitres/chapitre1/assurance-vie-stats.png}
    \caption{Le marché de l'assurance vie en France en 2024 \textbf{(graphique temporaire)}}
    \label{fig:marches_assurance_vie}
\end{figure}

Ces contrats sont soumis à un ensemble de réglementations prudentielles visant à garantir la solvabilité des assureurs et la protection des assurés. La directive Solvabilité II encadre ces exigences à l'échelle européenne. La partie suivante détaille le cadre réglementaire de Solvabilité II, ses implications pour les assureurs vie, ainsi que les outils de modélisation stochastique utilisés pour répondre à ces exigences.

\section{Le cadre prudentiel Solvabilité II}
\label{sec:s2}

Entrée en vigueur le 1er janvier 2016, la directive Solvabilité II régit le cadre prudentielle pour la quasi-totalité des assureurs et réassureurs de l'Union Européenne. Son principal objectif est d'harmoniser les pratiques du secteur, d'assurer une protection optimale des assurés et de garantir que les compagnies puissent honorer leurs engagements en toutes circonstances. Pour ce faire, elle instaure une approche économique et prospective, fondée sur une évaluation fine des risques et structurée en trois piliers interdépendants. Seuls les deux premiers piliers seront présentés car ils sont plus pertinents dans le cadre de ce mémoire.


\subsection{Le Pilier 1 : Exigences quantitatives et Bilan Prudentiel}

Le Pilier 1 définit les exigences quantitatives, au cœur desquelles se trouve le Bilan Prudentiel. Il s'agit d'une vision économique du bilan comptable où les actifs et les passifs sont évalués de manière cohérente avec leur valeur de marché (\textit{market-consistent}). La structure de cette section suivra l'équation fondamentale du bilan prudentiel :
\begin{equation}
    \text{Actifs} = \text{Provisions Techniques} + \text{Fonds Propres}
\end{equation}
Chaque terme de cette équation sera détaillé successivement.

\begin{figure}[H]
    \centering
    % --- NOTE : L'image du bilan est placée ici pour introduire visuellement la structure qui sera suivie. ---
    \includegraphics[width=0.65\textwidth]{images/2_chapitres/chapitre1/bilanS2.png}
    \caption{Bilan économique sous Solvabilité II (graphique temporaire)}
    \label{fig:Bilan_economique_S2}
\end{figure}


\subsubsection{L'évaluation des Actifs à la Valeur de Marché}

Le premier terme du bilan, Actifs, sont comptabilisés à leur Valeur de Marché (VM). Cette approche vise à refléter la valeur la plus juste et actuelle des ressources dont dispose l'assureur pour couvrir ses engagements.


\subsubsection{Les Provisions Techniques : Cœur de l'évaluation du passif}
% --- NOTE : Cette sous-section regroupe maintenant tout ce qui concerne le calcul des engagements : BE, RM et les méthodes de valorisation. ---

Les Provisions Techniques (PT) représentent la valeur des engagements de l'assureur envers ses assurés. Elles se décomposent en deux parties : le \textit{Best Estimate} (BE) et la Marge de Risque (\textit{Risk Margin} - RM).
\begin{equation}
    \text{PT} = \text{Best Estimate (BE)} + \text{Marge de Risque (RM)}
\end{equation}
\bigskip


% --- NOTE : Les méthodes de valorisation sont maintenant une sous-partie logique du calcul des provisions techniques. ---
La valorisation des provisions techniques, au cœur du bilan prudentiel, ne peut se contenter d'une vision unique et figée du futur. La présence d'options et de garanties dans les contrats d'épargne impose de distinguer deux approches complémentaires :
\begin{itemize}
    \item \textbf{L'approche déterministe} est un outil de pilotage. Elle repose sur une projection unique des variables économiques. Bien qu'insuffisante pour la valorisation prudentielle, elle demeure un outil fondamental pour l'élaboration du \textit{business plan} et la communication d'un scénario central. Sa limite principale est son incapacité à valoriser les risques asymétriques.

    \item \textbf{L'approche stochastique} est un outil de valorisation. Elle explore un grand nombre de futurs possibles à l'aide d'un \textbf{Générateur de Scénarios Économiques (GSE)}. Cet outil produit des milliers de simulations cohérentes des marchés financiers. La valeur d'un indicateur est alors obtenue en calculant la moyenne des résultats sur l'ensemble de ces scénarios (méthode de Monte-Carlo). Cette exploration est indispensable pour quantifier le coût réel des garanties optionnelles (Taux Minimum Garanti, etc.).
\end{itemize}
La différence de valeur entre ces deux approches est capturée par le concept de \textbf{TVOG (\textit{Time Value of Options and Guarantees})}. En imposant une approche stochastique, Solvabilité II assure une valorisation \textit{market-consistent} des engagements.
\bigskip


Le \textbf{Best Estimate (BE)}, ou \textit{Best Estimate Liability} (BEL), représente la meilleure estimation de la valeur actuelle des flux de trésorerie futurs liés aux engagements d'assurance. Son calcul est réalisé sur un horizon long (40-60 ans) en \textit{run-off} (portefeuille en extinction, pas de nouvelles souscriptions). Il est obtenu par la moyenne des flux actualisés sur un grand nombre de simulations économiques stochastiques en univers risque neutre~:
\begin{equation}
    BEL = \mathbb{E}^{\mathbb{Q}} \left[ \sum_{j=1}^{T} CF(j) \cdot e^{-\int_0^j r(s)ds} \right] \approx \frac{1}{N}\sum_{i=1}^{N}\sum_{j=1}^{T}\frac{CF_{i}(j)}{(1+r_{i,j})^{j}}
\end{equation}
Où $N$ est le nombre de simulations, $T$ l'horizon de projection, $CF_{i}(j)$ le flux de trésorerie net de l'année $j$ pour la simulation $i$, et $r_{i,j}$ le taux d'actualisation sans risque pertinent.
\bigskip


La \textbf{Marge de Risque (RM)} complète le Best Estimate. Elle correspond à la rémunération du capital réglementaire qui doit être immobilisé pour couvrir les risques non-financiers (longévité, rachat, etc.) jusqu'à l'extinction du portefeuille. Son calcul repose sur une approche dite de "Coût du Capital" (\textit{Cost of Capital} - CoC), qui consiste à actualiser le coût futur de détention de ce capital :
\begin{equation}
    RM = \text{CoC}_{\text{rate}} \times \sum_{j=0}^{T} \frac{\text{SCR}_{\text{non-fi}}(j)}{(1+r_{j+1})^{j+1}}
\end{equation}
Où $\text{CoC}_{\text{rate}}$ est le coût du capital (fixé à 6\%), et $\text{SCR}_{\text{non-fi}}(j)$ est la part du SCR couvrant les risques non-financiers à l'année $j$.


\subsubsection{Les Fonds Propres et les Exigences de Capital}
% --- NOTE : Cette sous-section lie logiquement les Fonds Propres aux exigences de capital (SCR/MCR) qui portent sur eux. ---
\bigskip


Les \textbf{Fonds Propres Prudentiels}, aussi appelés \textbf{NAV (\textit{Net Asset Value})}, constituent la richesse de l'assureur. Ils sont définis par la différence entre la valeur des actifs et celle des engagements :
\begin{equation}
    NAV = VM_{Actifs} - (BE + RM)
\end{equation}
Ce sont ces fonds propres qui doivent permettre à l'assureur d'absorber des pertes inattendues. Solvabilité II définit donc deux niveaux d'exigence de capital portant sur la NAV.
\bigskip


Le \textbf{Solvency Capital Requirement (SCR)} est le montant de fonds propres nécessaire pour absorber des pertes exceptionnelles. Il est calibré pour correspondre à la \textbf{Value-at-Risk (VaR) à 99,5\%} de la NAV sur un horizon d'un an. Autrement dit, il s'agit du capital qui doit permettre à l'assureur de faire face à ses engagements sur l'année à venir avec une probabilité de 99,5\%. En cas de non-respect, l'assureur fait l'objet d'un suivi renforcé par le régulateur.

Le calcul du SCR peut se faire via un modèle interne (spécifique à l'assureur) ou, plus communément, via la \textbf{Formule Standard} définie par la réglementation. Cette dernière est une approche modulaire qui décompose le risque total en plusieurs modules et sous-modules (risque de marché, de souscription, de contrepartie, etc.).

Pour chaque risque élémentaire $x$, le capital requis est calculé comme la perte de NAV consécutive à un choc instantané et calibré sur ce risque :
\begin{equation}
    SCR_{x} = \Delta NAV = NAV_{\text{central}} - NAV_{\text{choc } x}
\end{equation}
Les SCR des différents modules sont ensuite agrégés en prenant compte des corrélations prédéfinies entre les risques. L'agrégation de deux modules de risque $i$ et $j$ se fait via la formule :
\begin{equation}
    SCR_{i,j} = \sqrt{SCR_i^2 + SCR_j^2 + 2 \times Corr_{i,j} \times SCR_i \times SCR_j}
\end{equation}
où $Corr_{i,j}$ est le coefficient de corrélation entre les risques $i$ et $j$ fourni par la réglementation. Cette agrégation est appliquée de manière hiérarchique pour obtenir le SCR total, appelé \textit{Basic Solvency Capital Requirement} (BSCR) :
\begin{equation}
    SCR_{total} = \sqrt{\sum_{i}\sum_{j} SCR_i \times SCR_j \times Corr_{i,j}}
\end{equation}

\begin{figure}[H]
    \centering
    \includegraphics[width=0.95\textwidth]{images/2_chapitres/chapitre1/pieuvre_scr.png}
    \caption{Schéma des modules et sous-modules du SCR en Formule Standard}
    \label{fig:pieuvre_scr}
\end{figure}

\subsection{Le Pilier 2 : Exigences qualitatives et gouvernance}

Ce pilier se concentre sur la supervision, la gestion des risques et la gouvernance interne. Il impose aux assureurs de mettre en place un système de gouvernance sain, prudent et proportionné. Cela inclut une structure organisationnelle transparente, des politiques écrites claires, et un système de contrôle interne robuste. La direction doit être assurée par au moins deux dirigeants effectifs (principe des 4 yeux) qui doivent répondre à des exigences de compétence et d'honorabilité (\textit{fit and proper}).

Ce système s'articule autour de quatre fonctions clés indépendantes : la fonction actuarielle, la gestion des risques, l'audit interne et la conformité.

L'élément central du Pilier 2 est l'\textbf{ORSA} (\textit{Own Risk and Solvency Assessment}). Il s'agit d'un processus interne par lequel l'assureur évalue, sur un horizon de 3 à 5 ans, l'adéquation entre son profil de risque spécifique, ses limites de tolérance et ses besoins globaux en solvabilité. C'est un outil de pilotage stratégique qui permet d'aller au-delà des hypothèses standards pour refléter la stratégie propre de l'entreprise.

\section{Les Générateurs de Scénarios Économiques (GSE)}
\label{sec:gse}

Le Générateur de Scénarios Économiques (GSE) est un outil mathématique central dans la modélisation stochastique. Il a pour fonction de simuler de multiples trajectoires futures pour les principales variables financières (taux d'intérêt, performance des actions, inflation, etc.). La qualité des projections ALM dépendant directement de la robustesse du GSE, il est nécessaire de distinguer deux cadres de modélisation qui coexistent.

Bien que ces deux univers soient complémentaires, la réglementation Solvabilité II assigne à chacun un rôle très précis pour le calcul des différents indicateurs prudentiels. Le tableau suivant synthétise cette répartition des tâches.

\begin{table}[H]
\centering
\caption{Répartition des calculs Solvabilité II par univers de projection}
\label{tab:s2_par_univers}
\begin{tabularx}{\textwidth}{>{\raggedright\arraybackslash}X >{\raggedright\arraybackslash}X}
\toprule
\textbf{\texorpdfstring{Univers Risque Neutre (Q)}{Univers Risque Neutre (Q)}} & \textbf{\texorpdfstring{Univers Monde Réel (P)}{Univers Monde Réel (P)}} \\
\midrule
\textbf{Indicateurs du Pilier 1 :}
\begin{itemize}[itemsep=2pt]
\item Best Estimate Liability (BEL)
\item Marge de Risque (RM)
\item Solvency Capital Requirement (SCR)
\item Bilan Prudentiel et NAV
\end{itemize}

\textbf{Exercices du Pilier 2 :}
\begin{itemize}[itemsep=2pt]
\item ORSA (Own Risk and Solvency Assessment)
\end{itemize}
&
\textbf{Exercices du Pilier 2 :}
\begin{itemize}[itemsep=2pt]
\item ORSA (Own Risk and Solvency Assessment)
\item Business Plan et planification stratégique
\item Test de la pérennité du modèle
\end{itemize} \\
\addlinespace
\textbf{Finalité : Valorisation} \textit{Market-Consistent} à un instant t.
&
\textbf{Finalité : Pilotage} stratégique et prospectif. \\
\bottomrule
\end{tabularx}
\end{table}

La distinction entre ces deux approches est donc fondamentale : l'une sert à valoriser, l'autre à piloter. Les sections suivantes détaillent les modèles mathématiques sous-jacents à chaque univers.

\subsection{\texorpdfstring{L'univers Risque Neutre ($\mathbb{Q}$)}{L'univers Risque Neutre (Q)} : un cadre pour la valorisation}

L'\textbf{univers Risque Neutre ($\mathbb{Q}$)} est un cadre de valorisation théorique, requis par Solvabilité II pour les calculs \textit{market-consistent}. Son objectif n'est pas de prédire l'évolution réelle des marchés, mais de calculer la valeur risque neutralisée d'un actif ou d'un passif à la date de calcul, en se fondant sur les prix de marché observés. Dans cet univers, on postule que tous les investisseurs sont indifférents au risque, ce qui implique que le rendement espéré de n'importe quel actif est égal au taux d'intérêt sans risque. Cette construction, fondée sur l'absence d'opportunité d'arbitrage, est indispensable pour valoriser de manière cohérente les options et garanties complexes des contrats d'assurance. La valeur $V_0$ d'un flux de trésorerie futur $CF_T$ est alors son espérance mathématique sous cette probabilité risque neutre, actualisée au taux sans risque $r(s)$ :
\begin{equation}
    V_0 = \mathbb{E}^{\mathbb{Q}} \left[ CF_T \cdot e^{-\int_0^T r(s)ds} \right]
    \label{eq:valeur_risque_neutre}
\end{equation}
Cet univers constitue le fondement du Pilier 1 de Solvabilité II, utilisé pour le calcul du Best Estimate Liability (BEL) et du Solvency Capital Requirement (SCR).

\subsubsection{Modélisation des taux d'intérêt : le modèle de Hull \& White}
Pour les taux d'intérêt, le modèle de \textbf{Hull \& White à un facteur} est une référence dans le cadre réglementaire. Son principal avantage est sa capacité à se calibrer parfaitement à la courbe des taux sans risque initiale, telle que fournie par l'EIOPA.
\begin{figure}[H]
    \centering
    \includegraphics[width=0.65\textwidth]{images/2_chapitres/chapitre1/courbe_EIOPA.png}
    \caption{Courbe des taux sans risque sans \textit{Volatility Adjustment} au 31/12/2024 publiée par l'EIOPA}
    \label{fig:courbe_EIOPA}
\end{figure}

Cette flexibilité est obtenue grâce à un paramètre de retour à la moyenne $\theta(t)$ qui dépend du temps. Son équation différentielle stochastique (EDS) s'écrit :
\begin{equation}
    dr_t = (\theta(t) - ar_t)dt + \sigma dW^{\mathbb{Q}}_t
    \label{eq:hull_white}
\end{equation}
où $r_t$ est le taux d’intérêt court, $a$ la vitesse de retour à la moyenne, $\sigma$ la volatilité et $W^{\mathbb{Q}}_t$ un mouvement brownien sous la mesure risque neutre. Pour des raisons de calcul, nous utilisons la solution discrète de cette EDS :
\begin{equation}
    r_{t+h} = r_t e^{-ah} + \theta(t+h) - \theta(t)e^{-ah} + \sigma \sqrt{\frac{1 - e^{-2ah}}{2a}} Z
    \label{eq:hull_white_discrete}
\end{equation}

\subsubsection{Modélisation des actions et de l'immobilier : le modèle de Black \& Scholes}
Pour les actifs risqués comme les actions ou l'immobilier, le modèle de \textbf{Black \& Scholes} est couramment utilisé. Conformément à la logique risque neutre, le rendement espéré (la dérive du processus) est le taux sans risque $r_t$. L'EDS du prix de l'actif $S_t$ est :
\begin{equation}
    dS_t = r_t S_t dt + \sigma S_t dW^{\mathbb{Q}}_t
    \label{eq:black_scholes_q}
\end{equation}
où $S_t$ est le prix de l'actif, $r_t$ le taux sans risque et $\sigma$ la volatilité de l'actif. La solution de cette EDS est donnée par :
\begin{equation}
    S_t = S_0 \exp\left( \int_0^t \left(r_s - \frac{\sigma^2}{2}\right)ds + \int_0^t \sigma dW^{\mathbb{Q}}_s \right)
\end{equation}
En pratique, on utilise sa solution discrétisée pour simuler les trajectoires de prix sur un pas de temps $h$ :
\begin{equation}
    S_{t+h} = S_t \exp\left( \left(r_t - \frac{\sigma^2}{2}\right)h + \sigma\sqrt{h}Z \right)
    \label{eq:black_scholes_q_discrete}
\end{equation}
où $Z$ est une variable aléatoire suivant une loi normale centrée réduite $\mathcal{N}(0,1)$.

\subsection{\texorpdfstring{L'univers Monde Réel ($\mathbb{P}$)}{L'univers Monde Réel (P)} : un outil de pilotage stratégique}

À l'inverse de l'univers risque neutre, l'\textbf{univers Monde Réel ($\mathbb{P}$)} vise à générer des scénarios réalistes pour refléter une évolution plausible des marchés. Son objectif est la projection et la planification stratégique, notamment pour l'exercice ORSA (Pilier 2). 

La différence fondamentale réside dans l'introduction d'une \textbf{prime de risque} pour rémunérer la volatilité supportée par les investisseurs. Le rendement espéré d'un actif risqué est donc supérieur au taux sans risque, calibré sur des données historiques et des anticipations d'experts :
\begin{equation}
    \mathbb{E}^{\mathbb{P}}[\text{Rendement de l'actif}] = \text{Taux sans risque} + \text{Prime de risque}
\end{equation}

Les modèles utilisés, bien que similaires dans leur forme à ceux de l'univers $\mathbb{Q}$ (par exemple, Vasicek pour les taux ou Black \& Scholes pour les actions), sont modifiés pour intégrer cette prime. La dérive du processus stochastique n'est plus le taux sans risque $r_t$, mais un rendement espéré monde réel $\mu$.En somme, si l'univers $\mathbb{Q}$ valorise les engagements à un instant $t$, l'univers $\mathbb{P}$ permet d'exprimer la situation financière de l'entreprise dans le futur, ce qui le rend indispensable pour le pilotage stratégique.

\subsection{Synthèse des deux univers}

Le tableau suivant résume les caractéristiques et les usages des deux univers de projection. Si l'univers risque neutre $\mathbb{Q}$ répond à la question « \textit{Combien vaut cet engagement aujourd'hui ?} », l'univers monde réel $\mathbb{P}$ répond à « \textit{Quelle sera ma situation financière demain ?} ».

\begin{table}[H]
    \centering
    \caption{Synthèse comparative des univers de projection}
    \label{tab:univers_s2_comp}
    \begin{tabularx}{\textwidth}{l >{\raggedright\arraybackslash}X >{\raggedright\arraybackslash}X}
        \toprule
        \textbf{Critère} & \textbf{\texorpdfstring{Univers Risque Neutre ($\mathbb{Q}$)}{Univers Risque Neutre (Q)}} & \textbf{\texorpdfstring{Univers Monde Réel ($\mathbb{P}$)}{Univers Monde Réel (P)}} \\
        \midrule
        \textbf{Objectif}
        &
        \textbf{Valorisation} \textit{Market-Consistent} (Pilier 1 : BEL, SCR). Calculer une valeur juste à $t=0$.
        &
        \textbf{Projection} stratégique (Pilier 2 : ORSA, Business Plan). Simuler des futurs plausibles. \\
        \addlinespace
        \textbf{Rendement Espéré (Actifs risqués)}
        &
        Taux sans risque ($r_t$). Aucune prime de risque.
        &
        Taux sans risque + Prime de risque ($\mu = r + \text{prime}$). \\
        \addlinespace
        \textbf{Modèle de Taux Typique}
        &
        \textbf{Hull \& White}. Flexible, calibré à la courbe des taux initiale.
        &
        \textbf{Vasicek}. Économique, retour à une moyenne de long terme. \\
        \addlinespace
        \textbf{Calibration}
        &
        Calibré sur les prix des instruments financiers \textbf{actuels} (courbe des taux, volatilités implicites).
        &
        Calibré sur des \textbf{données historiques} et des \textbf{anticipations d'experts} (primes de risque). \\
        \bottomrule
    \end{tabularx}
\end{table}

Pour la suite de ce mémoire, je vais me concentrer sur l'univers risque neutre $\mathbb{Q}$, car il est le plus pertinent pour les calculs prudentiels et la gestion Actif-Passif dans le cadre des calculs liés au pilier 1 de Solvabilité II.


\section{La représentation du passif : le concept de \textit{Model Point}}
\label{sec:mp}

\subsection{La nécessité de l'agrégation}

Les portefeuilles d'assurance vie comptent souvent des centaines de milliers, voire des millions de polices. Une modélisation "police à police" est techniquement possible mais informatiquement très chronophage voire irréalisable pour des calculs stochastiques complexes comme ceux requis par les modèles ALM. La charge de calcul deviendrait prohibitive. La simplification du portefeuille de passif n'est donc pas un choix, mais une contrainte opérationnelle majeure.



La réponse standard à cette contrainte est la création de \textbf{\textit{Model Points}} (MP). Un MP est un contrat synthétique représentant un agrégat de polices partageant des caractéristiques homogènes. L'objectif est de réduire drastiquement le volume de données à traiter tout en préservant les propriétés actuarielles et financières essentielles du portefeuille complet. La qualité de la représentation dépend directement de la pertinence des critères de regroupement (caractéristiques du produit, de l'assuré, du contrat), souvent optimisés par des techniques de classification statistique (\textit{clustering}).

\subsection{Les impacts de l'agrégation en MP sur les indicateurs S2}

L'utilisation des \textit{Model Points} constitue la méthode standard pour agréger le passif et rendre les modèles ALM opérationnels. Cependant, la manière dont ces groupes de contrats sont formés à partir des polices individuelles a un impact direct et significatif sur la valorisation des indicateurs réglementaires et leur sensibilité aux chocs.



La problématique centrale de ce mémoire est donc d'optimiser cette étape fondamentale de l'agrégation. En partant du portefeuille granulaire "police à police", nous cherchons à définir, comparer et tester différentes méthodes de regroupement pour créer des \textit{Model Points}. L'objectif est de construire une méthodologie de simplification optimale, c'est-à-dire celle qui minimise l'erreur d'agrégation tout en garantissant la plus grande stabilité des indicateurs Solvabilité II (SCR, Marge de Risque, NAV) lors du calcul des différentes sensibilités. Cette section introductive pose donc les fondations de notre analyse : la simplification étant une nécessité, comment s'assurer que la méthode de regroupement choisie est la plus fidèle et la plus robuste possible ?


\section{Principes et Enjeux de la Gestion Actif-Passif (ALM)}
La gestion Actif-Passif (ALM) trouve son origine dans une particularité fondamentale du secteur de l'assurance énoncé précédemment dans le mémoire : le \textbf{cycle de production inversé}. Contrairement à une entreprise classique qui vend un produit avant d'en percevoir le revenu, un assureur collecte des primes aujourd'hui en échange de la promesse de verser des prestations dans un futur lointain et incertain. Ce décalage temporel est au cœur du modèle économique de l'assurance vie.

Ce mécanisme engendre une inadéquation structurelle (\textit{mismatch}) entre les deux côtés du bilan. D'une part, le passif est constitué d'engagements de longue durée, dont l'échéance et le montant sont soumis à des aléas (mortalité, comportement de rachat des assurés). D'autre part, pour couvrir ces engagements, l'assureur investit les primes sur les marchés financiers, constituant un actif dont la valeur et les flux sont, par nature, volatiles et dépendants du contexte économique.

Cette inadéquation est renforcée par une interdépendance dynamique et complexe entre l'actif et le passif.

\begin{itemize}
    \item \textbf{Le passif influe sur l'actif :} Le versement des prestations (rachats, décès) contraint l'assureur à liquider une partie de ses actifs, parfois dans des conditions de marché défavorables.
    \item \textbf{L'actif influe sur le passif :} La performance des actifs financiers a un impact direct sur le niveau des engagements. C'est notamment le cas de la \textbf{Participation aux Bénéfices (PB)}, qui dépend des résultats financiers générés par l'assureur. Le Code des assurances impose une redistribution minimale aux assurés, calculée comme suit :
    \begin{equation}
        \label{eq:pb_minReg}
        \text{PB}_{\text{minReg}} = 85\% \times \max(\text{RésFi}, 0) + 
        \begin{cases}
            90\% \times \text{RésTech} & \text{si RésTech} \ge 0 \\
            100\% \times \text{RésTech} & \text{si RésTech} < 0
        \end{cases}
    \end{equation}
    Où le Résultat Financier (RésFi) est directement issu de la performance des actifs et le Résultat Technique (RésTech) des risques de mortalité et de rachat.
\end{itemize}
La gestion Actif-Passif est donc la discipline qui vise à piloter les risques nés de cette interdépendance afin d'assurer la solvabilité et d'optimiser la rentabilité de l'acteur.

\subsection{La Modélisation ALM : un Outil de Projection Essentiel}

Pour quantifier et piloter les risques complexes découlant de l'inadéquation actif-passif, les assureurs ont recours à des modèles de projection actuariels sophistiqués, communément appelés modèles de Gestion Actif-Passif ou modèles ALM. L'utilité principale de ces modèles est de projeter le bilan d'un assureur, qu'il s'agisse d'un bilan prudentiel sous Solvabilité 2 ou d'un bilan comptable sous les normes IFRS17, afin d'évaluer la santé financière future de l'organisme sur un horizon de long terme. Leur fonctionnement repose sur la combinaison de deux piliers fondamentaux : des scénarios prospectifs sur l'environnement économique et financier, et des hypothèses sur le comportement futur des assurés (lois de mortalité, de rachat, etc.).

L'approche de modélisation peut être déterministe ou stochastique, chaque approche répondant à des objectifs d'analyse distincts.

L'approche \textbf{déterministe} consiste à projeter le bilan de l'assureur selon une trajectoire unique et prédéfinie de l'environnement économique. Cette trajectoire, qualifiée de \textbf{scénario central}, est généralement construite à partir de la courbe des taux sans risque fournie par l'EIOPA. Elle permet d'obtenir le (\textit{Best Estimate}) central, c'est à dire la somme des flux futurs actualisés du portefeuille dans un contexte économique considéré comme le plus probable, servant de base pour le plan d'affaires et la valorisation prudentielle.

Cependant, une approche déterministe ne peut à elle seule capturer l'éventail des risques, notamment ceux liés aux options et garanties financières (par exemple, les Taux Minimum Garantis ou les options de rachat) dont le coût ne se matérialise que dans des conditions de marché adverses. Pour pallier cette limite, une approche \textbf{stochastique} est nécessaire. Celle-ci s'appuie sur un \textbf{Générateur de Scénarios Économiques (GSE)} pour simuler un grand nombre (souvent plusieurs milliers) de trajectoires économiques futures possibles, chacune représentant une évolution plausible des marchés financiers.

Chaque scénario économique généré sert alors d'input pour une projection complète du modèle ALM. En agrégeant les résultats de ces multiples projections via la \textbf{méthode de Monte Carlo}, l'assureur obtient non plus une seule valeur, mais une distribution des résultats possibles. L'objectif final est de disposer, pour chaque trajectoire, du détail des flux financiers à la maille la plus fine. Cette granularité permet une analyse statistique approfondie des risques, comme le calcul de quantiles (Value at Risk à 99.5\%) pour déterminer le capital de solvabilité requis (SCR) sous Solvabilité 2. Ces modèles stochastiques sont donc au cœur de l'évaluation des risques et de la prise de décision stratégique, et constituent le fondement du modèle de simulation qui sera détaillé dans la suite de ce chapitre.

\begin{figure}[H]
    \centering
    \includegraphics[width=0.8\textwidth]{images/2_chapitres/chapitre2/alm_det.png}
    \caption{Fonctionnement d'un modèle ALM \textbf{(graphique temporaire)}}
    \label{fig:alm_det}
\end{figure}



\section{Architecture et Fonctionnement du Modèle de Projection}
\label{sec:architecture_modele}

L'objectif de cette section est de détailler l'architecture et le séquencement des opérations du modèle ALM développé pour les besoins de cette étude. Le modèle a été conçu pour simuler de manière dynamique et séquentielle le bilan d'un assureur vie sur un horizon de projection pluriannuel.

Son fonctionnement global peut être décomposé en trois phases principales, comme illustré dans la figure \ref{fig:modele_alm_sequence} :
\begin{enumerate}
\item \textbf{Phase d'Initialisation} : Préparation et validation des données d'entrée, et application des chocs réglementaires à la date de départ.
\item \textbf{Boucle de Projection Annuelle} : Cœur du modèle qui simule, année après année, l'évolution du bilan selon une séquence d'événements prédéfinis.
\item \textbf{Phase de Finalisation} : Calcul des indicateurs prudentiels et génération des résultats en fin de projection.
\end{enumerate}

\begin{figure}[h!]
\centering
% Note : Pensez à remplacer le chemin vers votre image
\includegraphics[width=0.98\textwidth]{images/2_chapitres/chapitre2/modele_alm_sequence.png}
\caption{Architecture générale et séquencement des événements du modèle ALM. \textbf{(graphique temporaire)}}
\label{fig:modele_alm_sequence}
\end{figure}

\subsection{Phase 1 : Initialisation du Modèle}

Cette première phase prépare l'environnement de projection. Elle est elle-même divisée en trois sous-étapes.

\subsubsection{Chargement et Validation des Données d'Entrée}

Le modèle est alimenté par un ensemble de données exhaustif, regroupées en quatre catégories :
\begin{itemize}
\item \textbf{Données Économiques et Financières} : Issues du Générateur de Scénarios Économiques (GSE), elles comprennent les courbes de taux, les taux d'inflation et les performances des différentes classes d'actifs pour chaque scénario stochastique.
\item \textbf{Portefeuille d'Actifs} : Les \textit{model points} d'actifs représentant l'ensemble des placements de l'assureur (obligations à taux fixe et variable, actions et immobilier).
\item \textbf{Portefeuille de Passifs} : Les \textit{model points} d'épargne décrivant les engagements envers les assurés.
\item \textbf{Hypothèses de Modélisation} : Un ensemble de tables paramétrant le comportement futur (stratégie d'investissement, stratégie ALM, règles de participation aux bénéfices, tables de mortalité, de rachat) et les chocs prudentiels Solvabilité 2.
\end{itemize}
Une étape de validation est systématiquement réalisée pour assurer la cohérence et la qualité des données chargées. Par exemple, nous vérifions qu'il n'y a pas de Provisions mathématiques négatives ou que la somme des taux d'affectation des différentes classes d'actifs est bien égale à 1.

\subsubsection{Application des Chocs Solvabilité 2 en t=0}

L'application des chocs instantanés en t=0 est une étape fondamentale du calcul du SCR (Solvency Capital Requirement) selon la formule standard de Solvabilité 2. L'objectif est de mesurer la résilience de l'assureur face à une série de scénarios de crise prédéfinis, calibrés pour représenter un événement se produisant une fois tous les 200 ans.

Pour chaque choc, le modèle crée un "axe analytique" : les données d'entrée sont dupliquées, puis les paramètres pertinents sont modifiés conformément au scénario de choc. La différence entre la valeur des fonds propres dans le scénario central (sans choc) et leur valeur dans le scénario choqué détermine le besoin en capital pour ce risque spécifique.

Les principaux chocs, ou modules de risque, se répartissent en plusieurs catégories :

\begin{itemize} 
    \item \textbf{Le risque de marché :} Il regroupe les risques liés aux fluctuations des marchés financiers. 
    \begin{itemize} 
        \item \textbf{Choc de taux d'intérêt :} Simule une variation soudaine, à la hausse ou à la baisse, de la courbe des taux d'intérêt. Ce choc a un impact majeur sur la valeur des actifs (notamment les obligations) et des passifs (qui sont actualisés avec cette courbe des taux). 
        \item \textbf{Choc actions :} Modélise une baisse brutale des marchés actions. La formule standard distingue généralement deux types de chocs actions (Type 1 et Type 2) en fonction de la nature et de la diversification des investissements. Par exemple, un choc de -33.84\% pourrait s'appliquer à un portefeuille d'actions diversifié. 
        \item \textbf{Choc immobilier :} Représente une chute de la valeur du marché immobilier. Le paramètre de -25\% est une valeur standard pour ce type de risque. 
        \item \textbf{Choc de spread :} Concerne le risque de crédit sur les obligations et les prêts. Il simule un élargissement des spreads de crédit, ce qui diminue la valeur de marché de ces actifs. 
    \end{itemize}

\item \textbf{Le risque de souscription vie :} Il est lié aux aléas inhérents à l'activité d'assurance vie.
\begin{itemize}
\item \textbf{Choc de mortalité :} Simule une augmentation soudaine et permanente du taux de mortalité (par exemple, +15\%). Ce choc est particulièrement impactant pour les contrats de prévoyance où l'assureur doit verser un capital en cas de décès.
\item \textbf{Choc de longévité :} À l'inverse, il modélise une baisse permanente du taux de mortalité (les assurés vivent plus longtemps que prévu). Ce risque affecte principalement les rentes viagères, pour lesquelles l'assureur doit verser des prestations plus longtemps.
\item \textbf{Choc de rachat :} Simule une variation massive et soudaine du comportement des assurés en matière de rachat de leurs contrats. Il se décline en trois sous-modules : une hausse des taux de rachat, une baisse, et un scénario de rachat de masse (catastrophique).
\item \textbf{Choc de dépenses :} Modélise une augmentation imprévue des frais de gestion de l'assureur, combinée à une hausse de l'inflation de ces frais.
\item \textbf{Choc de catastrophe :} Concerne les événements extrêmes, comme une pandémie, entraînant une augmentation brutale et temporaire de la mortalité (par exemple, une hausse de 0.15 points de pourcentage du taux de mortalité sur une année).
\end{itemize}

\end{itemize}

\subsubsection{Préparation à la Boucle de Projection}

Enfin, les tables de données sont préparées pour la projection en y ajoutant les dimensions d'analyse fondamentales : le \textbf{scénario économique}, la \textbf{période de projection} (l'année) et l'\textbf{événement} intra-annuel.

\subsection{Phase 2 : La Boucle de Projection Annuelle}

Le cœur du modèle est une boucle itérative qui projette le bilan année par année. Pour chaque pas de temps, trois événements clés sont modélisés séquentiellement.

\subsubsection{Événement 1 : Performance}
Cette première étape simule le "passage du temps" sans intervention active du management.
\begin{itemize}
\item \textbf{Côté Passif} : Les engagements évoluent sous l'effet des flux biométriques (décès), comportementaux (rachats), et contractuels (arrérages de rente). La provision mathématique est revalorisée en appliquant les Taux Minimums Garantis (TMG) pour les fonds euros et la performance des marchés pour les Unités de Compte (UC).
\item \textbf{Côté Actif} : La valeur des actifs est mise à jour pour refléter la performance des marchés. Les revenus (coupons, dividendes) sont encaissés et alimentent la trésorerie.
\end{itemize}

\subsubsection{Événement 2 : Stratégie d'Investissement}
Cette étape modélise les décisions de gestion financière. L'algorithme simule la politique d'investissement en cherchant à maintenir une allocation d'actifs cible. En fonction de la trésorerie disponible, le modèle arbitre entre les différentes classes d'actifs, générant des ordres d'achats et de ventes.

\subsubsection{Événement 3 : Stratégie ALM et Clôture du Compte de Résultat}
C'est l'étape finale de l'exercice annuel, qui vise à équilibrer le bilan.
\begin{enumerate}
\item \textbf{Détermination du Résultat Financier} : Le modèle agrège l'ensemble des produits financiers générés.
\item \textbf{Application de la Stratégie de Participation aux Bénéfices (PB)} : Le modèle calcule la PB à distribuer, en respectant les contraintes réglementaires et en visant un taux cible. Il peut activer des leviers comme la reprise de la Provision pour Participation aux Excédents (PPE) ou la réalisation de plus-values latentes.
\item \textbf{Clôture des Comptes} : Le résultat technique et financier sont finalisés pour établir le compte de résultat et s'assurer de l'équilibre du bilan de clôture.
\end{enumerate}

\subsection{Phase 3 : Finalisation et Génération des Outputs}

Une fois la boucle achevée pour toutes les années et tous les scénarios, le modèle entre dans sa phase finale.

\subsubsection{Calcul des Indicateurs Prudentiels}
Le principal objectif est de calculer le \textbf{Best Estimate (BE)} des passifs. Pour chaque scénario, le modèle agrège les flux de passifs futurs et les actualise en utilisant les courbes de taux sans risque correspondantes. La moyenne des BE sur l'ensemble des scénarios fournit la valeur centrale, tandis que la distribution permet de calculer le capital de solvabilité requis (SCR).

\subsubsection{Génération des Rapports de Sortie}
Enfin, le modèle produit un ensemble de rapports détaillés (similaires aux QRTs réglementaires) présentant les bilans projetés, les comptes de résultat, la décomposition du BE et du SCR, et d'autres indicateurs clés pour l'analyse actuarielle.

\section{Limites Actuelles du modèle}
A date, les fonds propres ne sont pas modélisés dans le modèle ALM. En effet, le modèle se concentre sur la projection de l'actif et du passif, ainsi que sur les interactions entre les deux, mais n'intègre pas encore la dynamique des fonds propres. Cette limitation est importante car les fonds propres jouent un rôle crucial dans la solvabilité et la résilience financière de l'assureur. Leur modélisation permettrait de mieux évaluer l'impact des stratégies de gestion sur la solidité financière globale de l'entreprise. L'absence des fonds propres nous empêche de calculer des indicateurs cruciaux tels que le ratio de solvabilité.

% \chapter{La Gestion Actif-Passif et développement d'un modèle ALM en Python}
% \section{La gestion Actif-Passif (ALM) : définitions et enjeux}
% \label{sec:alm}

% \subsection{Le cycle de production inversé, origine de l'inadéquation Actif-Passif}

% La gestion Actif-Passif (ALM) trouve son origine dans une particularité fondamentale du secteur de l'assurance que l'on a énoncé dans le début de ce chapitre : le \textbf{cycle de production inversé}. Contrairement à une entreprise classique qui vend un produit avant d'en percevoir le revenu, un assureur collecte des primes aujourd'hui en échange de la promesse de verser des prestations dans un futur lointain et incertain. Ce décalage temporel est au cœur du modèle économique de l'assurance vie.

% Ce mécanisme engendre une inadéquation structurelle (\textit{mismatch}) entre les deux côtés du bilan. D'une part, le passif est constitué d'engagements de longue durée, dont l'échéance et le montant sont soumis à des aléas (mortalité, comportement de rachat des assurés). D'autre part, pour couvrir ces engagements, l'assureur investit les primes sur les marchés financiers, constituant un actif dont la valeur et les flux sont, par nature, volatiles et dépendants du contexte économique.

% Cette inadéquation est renforcée par une interdépendance dynamique et complexe entre l'actif et le passif.

% \begin{itemize}
%     \item \textbf{Le passif influe sur l'actif :} Le versement des prestations (rachats, décès) contraint l'assureur à liquider une partie de ses actifs, parfois dans des conditions de marché défavorables.
%         \item \textbf{L'actif influe sur le passif :} La performance des actifs financiers a un impact direct sur le niveau des engagements. C'est notamment le cas de la \textbf{Participation aux Bénéfices (PB)}, qui dépend des résultats financiers générés par l'assureur. Le Code des assurances impose une redistribution minimale aux assurés, calculée comme suit :
%         \begin{equation}
%             \label{eq:pb_minReg}
%             \text{PB}_{\text{minReg}} = 85\% \times \max(\text{RésFi}, 0) + 
%             \begin{cases}
%                 90\% \times \text{RésTech} & \text{si RésTech} \ge 0 \\
%                 100\% \times \text{RésTech} & \text{si RésTech} < 0
%             \end{cases}
%         \end{equation}
%         Où le Résultat Financier (RésFi) est directement issu de la performance des actifs et le Résultat Technique (RésTech) des risques de mortalité et de rachat.
% \end{itemize}
% La gestion Actif-Passif est donc la discipline qui vise à piloter les risques nés de cette interdépendance afin d'assurer la solvabilité et d'optimiser la rentabilité de l'acteur.

% \subsection{Objectifs et périmètre de l'ALM}

% % À COMPLÉTER :
% % Expliquer ici les trois grands objectifs de l'ALM :
% % 1. Maîtriser les risques (taux, marché, liquidité...).
% % 2. Assurer la solvabilité réglementaire (respect du SCR).
% % 3. Optimiser la rentabilité (maximiser le rendement pour un risque donné).


% \subsection{Le modèle ALM : un outil de simulation prospective}

% % À COMPLÉTER :
% % Décrire ici le rôle du modèle ALM comme outil de prise de décision.
% % - Principe : simulation conjointe de l'actif et du passif sur le long terme.
% % - Inputs principaux : GSE, lois de comportement, portefeuille, règles de gestion.
% % - Mentionner que vous utilisez un tel outil dans le cadre de votre mémoire, sans pour autant faire une étude ALM complète.


% \subsection{Principaux indicateurs de pilotage ALM}

% % À COMPLÉTER :
% % Lister et décrire brièvement quelques indicateurs clés issus des modèles ALM.
% % - Projection de la NAV (fonds propres prudentiels).
% % - Évolution du ratio de couverture SCR.
% % - Analyses de sensibilité et stress tests (impact d'un choc de taux, actions, etc.).

% La réalisation des analyses de sensibilités, cœur de ce mémoire, s'appuie sur un moteur de projection robuste : le modèle ALM (Asset Liability Management). Historiquement, le modèle utilisé au sein d'Accenture reposait sur une architecture en VBA (Visual Basic for Applications). Pour des impératifs de performance, de maintenabilité et de flexibilité, la décision a été prise de développer un nouveau modèle en Python. Cette migration a été l'occasion de repenser l'architecture du modèle pour mieux répondre aux exigences de calcul complexes de Solvabilité 2.

% Ce chapitre décrit ce nouveau modèle ALM. Nous présenterons son architecture globale, son fonctionnement détaillé à travers la projection des actifs, des passifs et la mise en œuvre des stratégies de gestion, et nous conclurons sur ses limites actuelles qui ouvrent des perspectives d'amélioration.

% \section{Développement du Modèle ALM en Python}

% La transition d'un modèle ALM de VBA vers Python a été motivée par la recherche de performance, de maintenabilité et de modularité. Le langage Python, avec son écosystème de librairies scientifiques comme \textit{NumPy}, \textit{Pandas} ou \textit{Polars}, offre un cadre de développement beaucoup plus robuste et performant pour des calculs actuariels intensifs.

% \subsection{Présentation du modèle ALM développé pour Accenture}

% Le modèle ALM a pour objectif principal de projeter l'ensemble des actifs et des passifs d'un assureur sur un horizon de long terme (typiquement 40 à 60 ans), sous un grand nombre de scénarios économiques stochastiques. Ces projections permettent de calculer les indicateurs prudentiels requis par la directive Solvabilité 2, notamment le BE et le SCR. Le modèle se veut un outil de pilotage stratégique, capable de simuler l'impact de différentes stratégies de gestion d'actifs, de politiques de souscription ou de participation aux bénéfices.

% \subsection{Fonctionnement du modèle ALM}

% \subsubsection{Fonctionnement général du modèle}
% Le modèle opère de manière itérative, année par année, pour chaque scénario économique. À chaque pas de temps, il simule l'ensemble des flux financiers et des opérations de bilan. Le processus global peut être résumé comme suit :
% \begin{enumerate}
% \item \textbf{Entrées :} Le modèle prend en entrée le portefeuille de passifs (issu du générateur ou d'un client), le portefeuille d'actifs, un set de scénarios économiques (ESG), et les règles de gestion (stratégie d'investissement, politique de PB, etc.).
% \item \textbf{Moteur de projection :} Pour chaque année et chaque scénario, le moteur calcule les flux de passifs, les flux d'actifs, et applique les décisions de gestion.
% \item \textbf{Sorties :} En fin de projection, le modèle génère des comptes de résultat et des bilans prévisionnels pour chaque scénario, qui sont ensuite utilisés pour calculer les indicateurs S2 par agrégation et analyse statistique.
% \end{enumerate}

% \subsubsection{Fonctionnement du passif}
% La projection du passif consiste à simuler l'évolution du portefeuille de contrats. À chaque pas de temps, le modèle calcule :
% \begin{itemize}
% \item Les primes encaissées.
% \item Les prestations versées (décès, rachats, rentes).
% \item Les chargements et frais prélevés.
% \item L'évolution des provisions mathématiques, en tenant compte de la revalorisation issue de la participation aux bénéfices.
% \end{itemize}
% Les flux de prestations sont déterminés par l'application des lois de comportement (mortalité, rachat) sur le portefeuille des assurés survivants.

% \subsubsection{Fonctionnement de l'actif}
% Simultanément, le modèle projette la valeur du portefeuille d'actifs. Pour chaque classe d'actifs (actions, obligations, immobilier, etc.), il calcule :
% \begin{itemize}
% \item L'évolution de la valeur de marché, en fonction des indices fournis par le scénario économique.
% \item Les revenus générés (dividendes, coupons, loyers).
% \end{itemize}
% Le modèle gère également le réinvestissement des flux de trésorerie et les opérations d'achat/vente décidées par la stratégie d'investissement.

% \subsubsection{Fonctionnement de la stratégie d'investissement}
% La stratégie d'investissement est un ensemble de règles qui dictent la manière dont l'actif est géré. Le modèle implémente une allocation stratégique cible (\textit{Strategic Asset Allocation} - SAA). Chaque année, il compare l'allocation réelle du portefeuille à l'allocation cible et déclenche des opérations d'achat ou de vente pour réduire l'écart, dans le respect des contraintes de liquidité et de transaction.

% \subsubsection{Fonctionnement de la stratégie ALM et de la politique de PB}
% Le cœur du modèle ALM est l'interaction entre l'actif et le passif. Le résultat financier généré par l'actif est utilisé pour déterminer la revalorisation servie aux assurés. La politique de Participation aux Bénéfices (PB) est une fonction clé qui répartit la performance financière entre l'assureur et les assurés, dans le respect des engagements contractuels et réglementaires. Le modèle simule la constitution et la reprise de la Provision pour Participation aux Bénéfices (PPB), qui permet de lisser les taux servis dans le temps.

% \subsection{Limites actuelles du modèle}
% Malgré sa robustesse, le modèle actuel présente certaines limites. Les règles de gestion, notamment la stratégie d'investissement, sont encore modélisées de manière relativement statique et ne réagissent pas toujours de façon dynamique aux conditions de marché extrêmes. De plus, la granularité de certains modules pourrait être affinée, notamment en ce qui concerne la modélisation des frais ou des impôts. Enfin, bien que les performances aient été grandement améliorées par rapport à la version VBA, les temps de calcul pour des portefeuilles très volumineux sur des milliers de scénarios restent un défi et une piste d'optimisation continue.