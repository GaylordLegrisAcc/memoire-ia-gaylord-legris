\newpage % Pour commencer la bibliographie sur une nouvelle page

% Ajoute une entrée "Bibliographie" dans la table des matières
\addcontentsline{toc}{chapter}{Bibliographie} 

\begin{thebibliography}{9}

% \bibitem{auteur_livre}
% NOM Prénom. \textit{Titre du livre en italique}. Lieu d'édition : Nom de l'éditeur, Année.

% \bibitem{auteur_article}
% NOM Prénom. « Titre de l'article entre guillemets ». \textit{Titre de la revue en italique}, volume X, numéro Y, saison Année, p. 123-456.

\bibitem{clustering_book}
DORNAIKA Fadi, HAMAD Denis, CONSTANTIN Joseph, TRONG HOANG Vinh. \textit{Advances in Data Clustering}. Lieu d'édition : Springer, 2024.

\bibitem{goffard_guerrault}
GOFFARD Pierre-Olivier, GUERRAULT Xavier. « Is it optimal to group policyholders by age, gender, and seniority for BEL computations based on model points? ». \textit{European Actuariel Journal}, volume 5, 17 Avril 2015, p. 165-180.


\bibitem{memoire_ben_fadhel}
BEN FADHEL Amine. « Accéleration de l'évaluation de la solvabilité prospective d'un assureur épargne ». \textit{Mémoire pour l'Institut des Actuaires }, 2022.

\bibitem{france_assureurs}
FRANCE ASSUREURS. \textit{L'assurance vie en 2023}. (20 septembre 2024). Consulté le 1er Septembre 2025, sur \url{https://www.franceassureurs.fr/nos-chiffres-cles/assurance-vie/etude-statistique-assurance-vie-2023/}

% \bibitem{rapport_technique}
% NOM Prénom et NOM Prénom. \textit{Titre du rapport}. Type de rapport (e.g. Rapport de recherche), Organisme/Université, Année.

\end{thebibliography}