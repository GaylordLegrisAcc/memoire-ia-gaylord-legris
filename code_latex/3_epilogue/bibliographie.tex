\newpage % Pour commencer la bibliographie sur une nouvelle page

% Ajoute une entrée "Bibliographie" dans la table des matières
\addcontentsline{toc}{chapter}{Bibliographie} 

\begin{thebibliography}{9}

% \bibitem{auteur_livre}
% NOM Prénom. \textit{Titre du livre en italique}. Lieu d'édition : Nom de l'éditeur, Année.

% \bibitem{auteur_article}
% NOM Prénom. « Titre de l'article entre guillemets ». \textit{Titre de la revue en italique}, volume X, numéro Y, saison Année, p. 123-456.
\bibitem{france_assureurs}
FRANCE ASSUREURS. \textit{L'assurance vie en 2024}. (23 septembre 2025). Consulté le 28 Octobre 2025, sur \url{https://www.franceassureurs.fr/nos-chiffres-cles/assurance-vie/lassurance-vie-en-2024/}


\bibitem{clustering_book}
DORNAIKA Fadi, HAMAD Denis, CONSTANTIN Joseph, TRONG HOANG Vinh. \textit{Advances in Data Clustering}. Lieu d'édition : Springer, 2024.

\bibitem{goffard_guerrault}
GOFFARD Pierre-Olivier, GUERRAULT Xavier. « Is it optimal to group policyholders by age, gender, and seniority for BEL computations based on model points? ». \textit{European Actuariel Journal}, volume 5, 17 Avril 2015, p. 165-180.


\bibitem{memoire_ben_fadhel}
BEN FADHEL Amine. « Accéleration de l'évaluation de la solvabilité prospective d'un assureur épargne ». \textit{Mémoire pour l'Institut des Actuaires }, 2022.

\bibitem{insee_prop_av_age}
INSEE. \textit{Taux de détention des produits de patrimoine selon l'âge de la personne de référence du ménage en 2021}. (2021). Consulté le 18 Septembre 2025, sur \url{https://www.insee.fr/fr/outil-interactif/5367857/details/30_RPC/34_PAT/34A_Figure1}

\bibitem{pyramide_age}
INSEE. \textit{Pyramide des âges de la population française en 2024}. (2024). Consulté le 1er Septembre 2025, sur \url{https://www.insee.fr/fr/outil-interactif/5014911/pyramide.htm}

\bibitem{insee_patrimoine_age}
INSEE. \textit{Enquête Histoire de vie et Patrimoine 2020-2021}. 2021. Consulté le 24 Janvier 2026, sur \url{https://www.insee.fr/fr/statistiques/7941419?sommaire=7941491#tableau-figure1}

\bibitem{repartition_hommes_femmes}
IFOP. \textit{Enquête sur les 60 ans de l'indépendance financière des femmes : chiffres clés sur la répartition hommes-femmes en assurance vie}. (16 Juillet 2025). Consulté le 16 Novembre 2025, sur \url{https://www.ifop.com/article/enquete-sur-les-60-ans-de-lindependance-financiere-des-femmes/}

\bibitem{repartition_hommes_femmes_euro_uc}
IFOP. \textit{Baromètre - les femmes et l'argent : chiffres clés sur la répartition hommes-femmes sur les fonds EURO et UC en assurance vie}. (Janvier 2024). Consulté le 16 Novembre 2025, sur \url{https://www.ifop.com/wp-content/uploads/2024/03/barometre-vives-2024ok-1.pdf}

\bibitem{kmeans}
geeksforgeeks. \textit{K-Means vs K-Means++ Clustering Algorithm}. Consulté le 1 Février 2026, sur \url{https://www.geeksforgeeks.org/machine-learning/k-means-vs-k-means-clustering-algorithm/}

% \bibitem{rapport_technique}
% NOM Prénom et NOM Prénom. \textit{Titre du rapport}. Type de rapport (e.g. Rapport de recherche), Organisme/Université, Année.
\bibitem{dbscan_hdbscan}
Daily Dose of Data Science. \textit{HDBSCAN vs. DBSCAN}. Consulté le 1 Février 2026, sur \url{https://blog.dailydoseofds.com/p/hdbscan-vs-dbscan}

\bibitem{hdbscan}
HDBSCAN Developers. \textit{How HDBSCAN Works}. Consulté le 3 Février 2026, sur \url{https://hdbscan.readthedocs.io/en/latest/how_hdbscan_works.html}


\bibitem{acpr66}
ACPR. \textit{Le taux technique en assurance vie (Code des assurances)}. Consulté le 14 Février 2026, sur \url{https://acpr.banque-france.fr/sites/default/files/medias/documents/201606_as66_le_taux_technique_en_assurance_vie.pdf}

\bibitem{acprTauxBas}
ACPR. \textit{Assurance vie en France et environnement de taux bas}. Consulté le 14 Février 2026, sur \url{https://acpr.banque-france.fr/sites/default/files/medias/documents/201705-as78-taux-bas-version-3_0.pdf}

\bibitem{acprRevalo2024}
ACPR. \textit{Revalorisation 2024 des contrats d’assurance-vie et de capitalisation}. Consulté le 14 Février 2026, sur \url{https://acpr.banque-france.fr/fr/publications-et-statistiques/publications/ndeg-175-revalorisation-2024-des-contrats-dassurance-vie-et-de-capitalisation}

\bibitem{acprMarche2024}
ACPR. \textit{Le marché de l’assurance-vie en 2024}. Consulté le 14 Février 2026, sur \url{https://acpr.banque-france.fr/fr/publications-et-statistiques/publications/ndeg-170-le-marche-de-lassurance-vie-en-2024}

\bibitem{gvmTMG}
Good Value for Money. \textit{Taux Minimum Garanti (TMG)}. Consulté le 14 Février 2026, sur \url{https://www.goodvalueformoney.eu/documentation/taux-minimum-garanti-tag}

\bibitem{faUC2024}
France Assureurs. \textit{L’assurance vie en unités de compte en 2024}. Consulté le 14 Février 2026, sur \url{https://www.franceassureurs.fr/nos-chiffres-cles/assurance-vie/lassurance-vie-en-unites-de-compte-en-2024/}

\end{thebibliography}