\chapter{Introduction au contexte réglementaire et à la modélisation ALM}
Ceci est le contenu du premier chapitre. Vous pouvez écrire ici votre texte ou ajouter des sections et sous-sections.

\section{Spécificités de l'assurance vie}

\section{La réglementation Solvabilité 2}
\subsection{Différents piliers}
\subsubsection{Pilier 1 : Exigences quantitatives}
\subsubsection{Pilier 2 : Exigences qualitatives}
\subsubsection{Pilier 3 : Transparence et reporting}

\subsection{Formule interne vs formule standard}
\subsubsection{Formule standard}
\subsubsection{Formule interne}
\subsubsection{Comparaison des deux approches}

\section{Définitions et enjeux de l’ALM}
\subsection{Définition de l’ALM}
\subsection{Enjeux pour les assureurs}

\section{Présentation du modèle ALM (pour plus tard)}

\section{Les générateurs de scénarios économiques}
\subsection{Définition et rôle}
\subsection{Exemples de générateurs utilisés}

\section{Qu’est ce qu’un model point et pourquoi on les utilise ?}
\subsection{Définition des model points}
\subsection{Utilisation dans la modélisation ALM}

\section{Impact des Réglementations sur les Portefeuilles (A mettre plus tard)}
\subsection{Analyse de l'impact des réglementations sur les structures de portefeuilles de passifs}
\subsection{Quelles sont les réglementations existantes concernant l’agrégation en MP}
\subsection{Études de cas illustrant les contraintes réglementaires}


