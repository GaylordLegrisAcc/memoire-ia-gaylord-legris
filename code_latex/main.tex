\documentclass[a4paper]{report} % Changé de 'article' à 'report'

%% Language and font encodings
\usepackage[english, french]{babel}
\usepackage[utf8]{inputenc}
\usepackage[T1]{fontenc}

%% Sets page size and margins
\usepackage{geometry}
\usepackage{lipsum}              % Pour du texte d'exemple (facultatif)

%% Useful packages
\usepackage{amsmath, amssymb}
\usepackage{amssymb} % Redundant if amsmath is used, but kept as in original
\usepackage{graphicx}
\usepackage[colorlinks=true, allcolors=black]{hyperref}
\usepackage{float}
\usepackage{tikz}
\usepackage{calc}
\usepackage{pgfplots}
\usepackage{setspace}
\usepackage{titlesec}
\usepackage{parskip}
\usepackage{tabularx}
\usepackage{booktabs}
\setcounter{secnumdepth}{3} %pour que le niveau 3 (subsubsection soit numeroté)
\setcounter{tocdepth}{2} %pour que le niveau 2 (subsection apparaisse dans la table des matieres)
% Personnalisation des titres de chapitres
% Met le mot "Chapitre" et son numéro en gras et très grand (\Huge)
% et le titre du chapitre également en \Huge
\titleformat{\chapter}[display]
  {\normalfont\Huge\bfseries}
  {\chaptertitlename\ \thechapter}
  {25pt}
  {\Huge}

% Personnalisation des titres de sections
% Met le numéro et le titre en gras et grand (\LARGE)
\titleformat{\section}
  {\normalfont\LARGE\bfseries}
  {\thesection}
  {1em}
  {}

% Personnalisation des titres de sous-sections
% Met le numéro et le titre en gras et un peu plus grand que la normale (\large)
\titleformat{\subsection}
  {\normalfont\large\bfseries}
  {\thesubsection}
  {1em}
  {}

\pgfplotsset{compat=1.14}

%% Added packages for new layout
\usepackage{fancyhdr} % For custom headers and footers

\newcommand\fillin[1]{\makebox[#1]{\dotfill}}


\begin{document}
\include{0_page_de_garde/page_de_garde.tex}
\restoregeometry % Restaurer la géométrie précédente avant de la redéfinir pour le reste du document.

% Redéfinir la géométrie pour le reste du document après la page de garde
\newgeometry{left=2.54cm,right=2.54cm,top=3.54cm,bottom=3.54cm}
\onehalfspacing
% Augmenter headheight pour accommoder des logos plus grands
\setlength{\headheight}{50pt} % Ajusté pour des logos de 1.6cm de hauteur

% Configurer fancyhdr pour les en-têtes et pieds de page personnalisés
\pagestyle{fancy}
\fancyhf{} % Effacer tous les champs d'en-tête et de pied de page

% En-têtes des pages suivantes avec les logos demandés (taille doublée)
% IMPORTANT: Assurez-vous que images/accenture_logo.png existe !
\fancyhead[L]{\includegraphics[height=1.6cm]{images/accenture_logo.png}} % Logo Accenture à gauche
\fancyhead[R]{\includegraphics[height=1.6cm]{images/isfa_logo.jpg}}   % Logo ISFA à droite

\fancyfoot[C]{\sffamily \thepage}

\renewcommand{\headrulewidth}{0pt}
\renewcommand{\footrulewidth}{0pt}

% Redéfinir le style 'plain' pour qu'il soit identique à 'fancy' afin que les logos apparaissent sur les premières pages de chapitre (taille doublée)
\fancypagestyle{plain}{%
  \fancyhf{}%
  \fancyhead[L]{\includegraphics[height=1.6cm]{images/accenture_logo.png}}% Logo Accenture à gauche
  \fancyhead[R]{\includegraphics[height=1.6cm]{images/isfa_logo.jpg}}%   Logo ISFA à droite
  \fancyfoot[C]{\sffamily \thepage}%
  \renewcommand{\headrulewidth}{0pt}%
  \renewcommand{\footrulewidth}{0pt}%
}


% Contenu principal en une seule colonne
\tableofcontents

% IMPORTANT: Ensure all included .tex files exist in the specified paths
% and do not contain LaTeX errors themselves!
\chapter*{Résumé}
% Ajouter manuellement cette section à la table des matières
\addcontentsline{toc}{chapter}{Résumé}
\include{1_preambule/1_abstract}
\include{1_preambule/2_remerciements}
\chapter*{Synthèse}
% Ajouter manuellement cette section à la table des matières
\addcontentsline{toc}{chapter}{Synthèse}
\include{1_preambule/4_synthesis}
\chapter*{Introduction}
% Ajouter manuellement cette section à la table des matières
\addcontentsline{toc}{chapter}{Introduction}

Placement privilégié des épargnants français, l'assurance vie a atteint un encours record de 1 908 milliards d'euros à fin 2023 (France Assureurs) \cite{site_web}, confirmant ainsi son rôle prépondérant dans le patrimoine financier national. Toutefois, ce secteur fait face à une rupture structurelle marquée par la fin du cycle de taux bas et la remontée brutale des taux d'intérêt observée depuis 2022. Le taux de revalorisation moyen des fonds en euros a ainsi atteint 2,6 \% pour l'année 2023 (estimation ACPR). Cette mutation rend d'autres produits d'épargne plus attractifs et exerce une pression concurrentielle inédite sur les contrats d'assurance vie, notamment sur les fonds en euros qui deviennent de moins en moins attractifs. Pour les assureurs, le défi est de taille : leurs portefeuilles d'actifs, majoritairement constitués d'obligations acquises durant la longue période de taux bas, présentent une forte inertie. Cet héritage obligataire freine leur capacité à servir des rendements compétitifs et place la gestion actif-passif (ALM) au cœur des enjeux stratégiques.

Pour piloter leur bilan, les assureurs s'appuient sur des modèles ALM sophistiqués, essentiels pour simuler l'impact de différentes stratégies dans le cadre réglementaire de Solvabilité II. Cependant, la complexité de ces modèles et la nécessité de réaliser un grand nombre de simulations se heurtent à une contrainte opérationnelle majeure : le temps de calcul. Cette contrainte limite la capacité des assureurs à explorer en profondeur l'ensemble des risques et des opportunités. Face à cette réalité, une question centrale émerge : dans quelle mesure peut-on réduire le temps de calcul des modèles tout en préservant la fidélité des indicateurs de risque ? Ce mémoire se propose d'investiguer cette problématique en étudiant l'impact d'une méthode de réduction de lignes dans des portefeuilles de passifs. L'enjeu est de déterminer si une représentation plus grossière du passif peut suffire pour représenter correctement le portefeuille et gérer le pilotage stratégique et sous quelles conditions une telle simplification est valide, sans masquer des dynamiques de risque essentielles.

Pour répondre à cette problématique, ce mémoire adoptera une double approche. Premièrement, il s'agira de développer un générateur de portefeuilles de passif puis le reste de l'analyse portera sur les effets de l'agrégation sur un portefeuille représentatif du marché français. L'objectif est de comprendre comment les risques évoluent à travers une agrégation. Ce mémoire ne se contentera pas d'analyser l'impact d'une seule méthode d'agrégation ; au contraire, plusieurs méthodes et approches seront testées. Le critère de sélection de la méthode la plus pertinente reposera sur un triple objectif : minimiser l'écart des indicateurs clés (notamment le Best Estimate et le SCR), optimiser la rapidité des calculs et atteindre le plus haut niveau d'agrégation possible qui garderait une significaitivité économique pour l'assureur suffisante.

L'axe principal de ce mémoire consistera donc à mener une analyse de sensibilités approfondie sur ces portefeuilles, qu'ils soient granulaires ou agrégés. Notre étude s'appuiera sur des indicateurs quantitatifs clés issus de la norme Solvabilité II, en évaluant notamment l'impact des chocs économiques sur le Best Estimate, le Solvency Capital Requirement (SCR) et la Present Value of Future Profits (PVFP). Ces métriques permettront de mesurer rigoureusement comment l'agrégation modifie la perception du risque et la valeur économique du portefeuille.

Ce mémoire s'articulera en quatre parties distinctes, chacune conçue pour apporter une réponse progressive et rigoureuse à notre problématique.


La \textbf{première partie} sera consacrée au cadre conceptuel de notre étude. Nous y détaillerons le contexte réglementaire de Solvabilité II, qui définit les exigences de capital et les métriques de risque, ainsi que les principes fondamentaux de la modélisation actif-passif (ALM). L'avantage de cette section est de fournir au lecteur les clés de compréhension essentielles pour appréhender les enjeux techniques et stratégiques du pilotage d'un bilan assurantiel.


La \textbf{deuxième partie} adoptera une approche pratique en se concentrant sur la mise en œuvre de notre environnement de simulation. Nous y décrirons la méthodologie de création des portefeuilles de passifs synthétiques, représentatifs de différentes générations de contrats, ainsi que les outils développés pour leur projection. Cette étape est cruciale car elle garantit la robustesse et la pertinence des analyses qui suivront, en créant un laboratoire d'expérimentation fiable.


La \textbf{troisième partie} constituera le cœur méthodologique de ce mémoire. Elle explorera et comparera de manière systématique plusieurs techniques d'agrégation des engagements de passif. L'objectif sera d'identifier les approches les plus prometteuses, en évaluant leur capacité à simplifier la structure du portefeuille sans dénaturer ses caractéristiques fondamentales. Cette analyse comparative permettra de mettre en lumière les forces et faiblesses de chaque méthode.


Enfin, la \textbf{quatrième partie} présentera et analysera les résultats de nos simulations. À travers des tests de sensibilité approfondis sur les indicateurs clés (SCR, PVFP), nous quantifierons l'impact de chaque méthode d'agrégation sur la perception du risque et la valeur économique. Cette analyse empirique nous permettra de conclure sur la validité des approches testées et de formuler des recommandations concrètes sur les conditions d'utilisation d'un passif agrégé pour un pilotage ALM à la fois efficace et optimisé.

Le secteur de l'assurance vie en France est confronté à une mutation structurelle induite par la remontée des taux d'intérêt, qui met sous pression la compétitivité des fonds en euros et complexifie la gestion actif-passif (ALM). Dans ce contexte, les assureurs s'appuient sur des modèles de projection sophistiqués pour piloter leur bilan et respecter les exigences du cadre réglementaire Solvabilité II. C'est de la rencontre entre ces exigences et les contraintes opérationnelles observées en entreprise qu'est née la problématique de ce mémoire.

\section{Contexte professionnel et genèse de la problématique chez Accenture}

Mon année en alternance au sein des équipes actuarielles d'Accenture m'a plongé au cœur de ces enjeux stratégiques. Ma mission principale a été de participer activement au développement et à l'industrialisation d'un modèle ALM propriétaire, entièrement programmé en Python. Cet outil, destiné à réaliser des projections et des simulations de bilans assurantiels, est conçu pour être à la fois robuste, flexible et performant.

Au cours de ce développement, nous avons été directement confrontés à énormément de contraintes opérationnelles dont le temps de calcul. L'exploration de stratégies multiples, la réalisation de sensibilités aux chocs réglementaires ou l'évaluation de nouvelles offres produits nécessitent un grand nombre de simulations. Or, la granularité fine des portefeuilles de passifs, bien que garantissant une précision maximale, engendre des temps de calcul prohibitifs qui peuvent freiner l'agilité et la prise de décision stratégique. Ce défi technique, vécu au quotidien, a été le point de départ de ma réflexion et constitue le fondement de ce mémoire. Les outils développés durant cette expérience, notamment le modèle ALM et un générateur de portefeuilles synthétiques, seront le socle technique sur lequel reposeront toutes les analyses de cette étude.

\section{Problématique et objectifs du mémoire}

L'optimisation des temps de calcul des modèles ALM est un enjeu de premier plan pour les compagnies d'assurance. Une des pistes explorées pour parvenir à améliorer les temps de calcul est la simplification ou l'agrégation des portefeuilles de passifs. Cependant, une telle démarche soulève une question fondamentale, qui constitue la problématique centrale de ce mémoire :

\begin{center}
\textit{Dans quelle mesure est-il possible d'agréger des portefeuilles de passifs en assurance vie pour optimiser les temps de simulation, sans dénaturer la mesure des indicateurs de risque et de valeur prudentiels issus du cadre Solvabilité II ?}
\end{center}

Pour répondre à cette problématique, ce mémoire poursuit un double objectif. D'une part, il s'agira de tester et de comparer rigoureusement plusieurs méthodes d'agrégation, en partant de portefeuilles de contrats générés grâce à un générateur de contrats développé dans le cadre de ce mémoire. D'autre part, nous chercherons à quantifier précisément l'impact de ces agrégations sur des indicateurs clés tels que le Best Estimate (BE), le Solvency Capital Requirement (SCR) et la Present Value of Future Profits (PVFP). L'enjeu est de déterminer s'il existe un niveau d'agrégation optimal qui préserve la signification économique des résultats tout en offrant un gain de performance substantiel.

Pour mener à bien cette analyse, ce mémoire s'articulera en quatre parties. La \textbf{première} posera le cadre conceptuel de Solvabilité II et de la modélisation ALM. La \textbf{deuxième} décrira l'environnement de simulation et les outils développés. La \textbf{troisième} se concentrera sur l'étude des différentes techniques d'agrégation. Enfin, la \textbf{quatrième} partie présentera et analysera les résultats des sensibilités menées sur les portefeuilles agrégés afin de répondre à notre problématique.

\clearpage
\pagenumbering{arabic}

\chapter{Introduction}
Ceci est le contenu du premier chapitre. Vous pouvez écrire ici votre texte ou ajouter des sections et sous-sections.

\section{Première section}
\lipsum[1-5] % Exemple de texte généré
dsdvv
\chapter{Construction d'un Générateur de Portefeuilles de Passifs}

\section{Objectifs Stratégiques et Contraintes Techniques}
La capacité à tester la robustesse des modèles et la pertinence des analyses de sensibilité repose sur un prérequis fondamental : la disponibilité de données de passif réalistes. Pour un cabinet de conseil, où l'accès aux portefeuilles des clients n'est pas systématique, la faculté de générer des portefeuilles synthétiques, mais représentatifs du marché, constitue un atout stratégique majeur. C'est dans ce contexte qu'un générateur de portefeuilles de passifs a été conçu et développé dans le cadre de ce mémoire.

Ce chapitre a pour vocation de présenter cet outil et la manière dont il a été construits. Il sera également détaillé les besoins stratégiques et analytiques auxquels ce générateur répond, la méthodologie de génération retenue, les contraintes techniques rencontrées et les données qui ont été utilisées pour rendre le portefeuille le plus réaliste possible.

 
\subsection{Définition du générateur de portefeuilles de passifs}
Le générateur de portefeuille de passifs développé dans le cadre de ce mémoire est un outil conçu pour créer, de manière algorithmique, des ensembles de données synthétiques qui imitent avec réalisme des portefeuilles de contrats d'assurance-vie. Plutôt que de s'appuyer sur des données réelles, souvent confidentielles ou indisponibles, cet outil simule les caractéristiques fondamentales des assurés (âge, sexe, etc.) et de leurs contrats (type de produit, montant de la provision mathématique, date de souscription, etc.).

L'objectif n'est pas de produire des données aléatoires, mais de générer un portefeuille dont les propriétés statistiques, distributions, corrélations, tendances, sont indiscernables de celles d'un portefeuille réel. Pour cela, toutes les lois et hypothèses ont été calibrées sur des données publiques de marché. Avoir des données fiables est très important pour produire des résultats de qualité dans le cadre d'analyses ou de décisions stratégiques, c'est pourquoi le développement d'un tel outil s'est imposé comme une nécessité.


\subsection{Besoins métiers : simulation de nouveaux produits et analyse concurrentielle}
Pour un acteur du secteur de l'assurance, qu'il s'agisse d'un assureur ou d'un cabinet de conseil, la capacité à modéliser et à anticiper les dynamiques de marché est un avantage concurrentiel décisif. Le générateur de portefeuilles de passifs répond directement à ce besoin en fournissant un support quantitatif pour la prise de décision stratégique. Il permet par exemple à un cabinet de conseil de tester ses modèles sans dépendre des données clients, et à un assureur d'explorer des scénarios prospectifs ou d'évaluer l'impact de nouvelles offres. Son utilité se manifeste dans trois domaines clés pour les assureurs : le lancement de nouveaux produits, l'orientation du \textit{business mix} et l'analyse concurrentielle.

Premièrement, le lancement d'un nouveau produit d'assurance-vie représente un investissement et un risque significatifs. Avant toute commercialisation, il est impératif d'en évaluer rigoureusement les impacts sur le profil de risque et la rentabilité de l'entreprise. Le générateur offre un véritable laboratoire virtuel pour effectuer ces tests. En simulant l'intégration de milliers de polices conformes aux caractéristiques du nouveau produit (garanties, frais, options), il permet de projeter leur comportement dans le temps. Il devient alors possible d'analyser leur effet sur les indicateurs prudentiels de Solvabilité II, tels que le \textit{Best Estimate} (BE) et le \textit{Solvency Capital Requirement} (SCR), mais aussi d'évaluer leur sensibilité à divers chocs de marché (hausse des taux, krach boursier) ou de comportement (vagues de rachats). Cet outil permet ainsi de tester, d'ajuster et d'optimiser les caractéristiques d'un produit pour atteindre le couple rendement/risque désiré avant même sa mise sur le marché.

Deuxièmement, le générateur est un outil précieux pour piloter la stratégie à long terme de l'entreprise. La direction peut être amenée à vouloir faire évoluer son \textit{business mix}, c'est-à-dire la répartition de son portefeuille entre différents types de produits (fonds en euros, unités de compte, prévoyance...). Par exemple, dans un contexte de taux bas persistants, un assureur pourrait vouloir accélérer sa transition vers les produits en unités de compte. Le générateur permet de quantifier les implications d'une telle stratégie. En simulant des portefeuilles futurs correspondant à ces nouvelles orientations commerciales, la direction peut visualiser les conséquences sur le bilan, la rentabilité prévisionnelle, mais aussi sur la consommation de capital et l'exposition aux risques. Ces simulations éclairent les décisions stratégiques et s'intègrent naturellement dans des exercices prospectifs comme l'ORSA (\textit{Own Risk and Solvency Assessment}).

Enfin, la capacité à se positionner par rapport à ses concurrents est fondamentale. Faute d'accès aux portefeuilles détaillés des autres acteurs, un assureur doit s'appuyer sur des reconstitutions. En se basant sur des données publiques (rapports annuels ou publications réglementaires comme les Rapport sur la Solvabilité et la Situation Financière) ou des statistiques sectorielles, le générateur peut permettre la création d'un portefeuille représentatif du marché, ou simuler le portefeuille probable d'un concurrent spécifique. Ces portefeuilles synthétiques deviennent alors une base solide pour des analyses comparatives (\textit{benchmarking}). Ils permettent non seulement d'évaluer la performance relative, mais aussi de comparer les profils de risque, d'anticiper les stratégies concurrentes et d'identifier les meilleures pratiques du marché. Il convient toutefois de souligner les limites d'une telle démarche. Une analyse ALM complète et réaliste d'un concurrent ne peut se contenter de la seule modélisation du passif. Elle exigerait également de simuler son portefeuille d'actifs et de disposer d'informations précises sur ses ressources financières et ses fonds propres. Or, ces données, qui relèvent du secret des affaires, sont rarement publiques. Par conséquent, l'analyse comparative reste nécessairement partielle, se concentrant sur les caractéristiques intrinsèques du portefeuille de passifs reconstitué.
\bigskip

\begin{figure}[h!]
    \centering
    \begin{tikzpicture}[
        node distance=1ex,
        box/.style={
            rectangle,
            rounded corners=4pt,
            draw=gray,
            fill=white,
            very thin,
            inner sep=15pt,
            minimum width=3cm,
            minimum height=2cm,
            align=center,
            font=\sffamily\bfseries\color{black}
        },
        arrow/.style={
            -Latex,
            very thin,
            color=accenture,
            line width=2pt
        }
    ]

    % Nodes
    \node[box] (input) {Données d'entrée :\\- Rapports d'assureurs\\- Marché français};
    \node[box, right=4ex of input] (process) {Modélisation et calibrage :\\- Distributions\\- Corrélations};
    \node[box, right=4ex of process] (output) {Portefeuille synthétique};

    % Arrows
    \draw [arrow] (input) -- (process);
    \draw [arrow] (process) -- (output);

    \end{tikzpicture}
    \caption{Schéma de la méthodologie de génération d'un portefeuille de passifs synthétique.}
    \label{fig:methodologie_horizontale}
\end{figure}

\subsection{Défis de la modélisation : réalisme, volumétrie et flexibilité}

La conception et la mise en œuvre d'un générateur de portefeuilles de passifs efficace soulèvent trois défis majeurs et interdépendants :

\begin{itemize}
\item \textbf{Le réalisme des données générées :} Il s'agit du défi le plus complexe. L'objectif n'est pas de produire des données aléatoires, mais de créer un portefeuille synthétique dont les propriétés statistiques sont indiscernables de celles d'un portefeuille réel. Cela implique non seulement de reproduire fidèlement les distributions de chaque caractéristique individuelle (âge, montant, etc.), mais aussi, et surtout, de capturer les corrélations complexes qui les lient. Par exemple, l'âge d'un assuré est souvent corrélé au type de produit souscrit et au montant de sa provision mathématique. Ignorer ces dépendances conduirait à un portefeuille incohérent, dont le comportement sous différents scénarios de risque serait erroné, invalidant ainsi les analyses prudentielles ou stratégiques qui en découlent.

\item \textbf{La gestion de la volumétrie :} Les portefeuilles d'assurance-vie des grands acteurs du marché se comptent en centaines de milliers, voire en millions de contrats. Le générateur doit être capable de produire des ensembles de données de cette ampleur de manière performante, c'est-à-dire dans un temps de calcul raisonnable et sans consommer une quantité excessive de ressources mémoire. Cette contrainte de performance est d'autant plus forte que la gestion des corrélations, nécessaire au réalisme du portefeuille, ajoute une complexité de calcul significative. Il faut donc trouver un équilibre entre la complexité statistique et la performance, ce qui a des implications directes sur les choix technologiques et algorithmiques.

\item \textbf{La flexibilité de l'outil :} Un générateur ne serait que d'une utilité limitée s'il ne produisait qu'un seul type de portefeuille statique. Pour répondre aux besoins métiers variés, l'outil doit être hautement paramétrable. L'utilisateur doit pouvoir ajuster finement les caractéristiques du portefeuille à générer : définir les spécificités d'un nouveau produit, modifier les distributions statistiques pour simuler un segment de marché différent, ou encore changer les lois de comportement (rachat, mortalité) pour tester de nouvelles hypothèses. Cette section concernant la flexibilité ne sera pas développée dans le cadre de ce mémoire car l'objectif est d'utiliser un portefeuille calibré sur le marché français. 
\end{itemize}

La section suivante présente en détail la méthodologie de modélisation probabiliste qui a été développée pour construire un portefeuille à la fois réaliste et volumineux.

\section{Méthodologie de Génération et Modélisation Statistique}

La méthodologie de génération du portefeuille de passifs synthétique repose sur une approche probabiliste. L'objectif est de construire un ensemble de contrats d'assurance dont les propriétés statistiques sont entièrement maîtrisées. Pour ce faire, chaque caractéristique d'un contrat (âge de l'assuré, montant de la provision, etc.) est modélisée comme une variable aléatoire, tirée d'une loi de probabilité préalablement calibrée sur des données de marché quand elles sont disponibles. Sinon, des hypothèses ont été formulées pour recréer ces variables de la manière la plus réaliste possible.


\subsection{Approche stochastique par lois de probabilité}

La génération du portefeuille synthétique s'appuie sur une modélisation stochastique où chaque attribut d'un contrat est représenté par une variable aléatoire. Dans un premier temps, l'objectif est de générer un portefeuille complet, représentatif du marché. Pour ce faire, une loi de probabilité marginale est définie pour chaque caractéristique, avec des paramètres rigoureusement calibrés sur des données de marché. Certaines variables sont ensuite liées entre elles pour refléter les dépendances observées dans les portefeuilles réels. Ce même cadre peut ensuite être adapté pour simuler un produit spécifique ; il suffirait alors de générer des variables aléatoires de manière conditionnelle aux caractéristiques de ce produit. Cette méthode garantit à la fois le réalisme statistique du portefeuille et la flexibilité nécessaire aux analyses prospectives.

Les sections suivantes détailleront la méthodologie de calibration pour les variables fondamentales qui structurent le portefeuille :
\begin{itemize}
    \item L'âge de l'assuré ;
    \item L'âge à la souscription, qui détermine l'ancienneté du contrat ;
    \item Le montant de la Provision Mathématique (PM).
\end{itemize}
D'autres variables, telles que le sexe ou la répartition des supports, seront également modélisées pour compléter le profil de chaque contrat.


\subsubsection{Modélisation de l'âge des assurés}

La première étape a consisté à construire une distribution de probabilité réaliste à partir de deux sources de données :
\begin{enumerate}
\item \textbf{La pyramide des âges de la population française} pour l'année 2024 \cite{pyramide_age}, fournissant la structure démographique de base entre 0 et 100 ans.

\item \textbf{Une étude statistique de l'INSEE} sur la détention d'assurance-vie par tranche d'âge en France \cite{insee_prop_av_age}
\begin{figure}[H]
\centering
\includegraphics[width=0.7\textwidth]{images/2_chapitres/chapitre3/insee_prop_av_age.png}
\caption{Proportion de détention d'assurance-vie par tranche d'âge en France \cite{insee_prop_av_age}.}
\label{fig:insee_prop_av_age}
\end{figure}
\end{enumerate}
Afin de transformer les données discrètes de l'INSEE, présentées par tranches d'âge, en une distribution continue du taux de détention par âge, une méthodologie d'interpolation a été mise en œuvre. Cette étape est cruciale pour pouvoir ensuite simuler l'âge des assurés de manière réaliste.

L'approche a consisté à définir d'abord des points de données représentatifs pour chaque tranche d'âge fournie. Les choix suivants ont été faits :
\begin{itemize}
    \item Pour la tranche des moins de 30 ans, plusieurs points ont été positionnés entre 18 et 30 ans afin de modéliser la croissance progressive de la détention en début de vie active.
    \item Pour les tranches intermédiaires (par exemple, 30-39 ans), le point central de l'intervalle a été retenu.
    \item Pour la tranche des "70 ans et plus", un âge représentatif de 80 ans a été choisi car à partir de cet âge, la souscription/résiliation du contrat d'assurance est rare.
\end{itemize}

Une fois ces points définis, une interpolation PCHIP (\textit{Piecewise Cubic Hermite Interpolating Polynomial} : interpolation par morceaux concervant la monotonie et utilisant des polynômes de degré trois) a été appliquée sur la partie de la courbe de 0 à 80 ans. Cette méthode a été privilégiée car elle évite les oscillations artificielles qu'une interpolation cubique aurait généré et garantit que la proportion de détention reste croissante comme ce que suggérent les données . Pour les âges plus avancés, une interpolation linéaire a été utilisée pour assurer une transition douce, suivie d'un plateau constant après 80 ans. Cette dernière hypothèse modélise une stabilisation du comportement de détention chez les assurés les plus âgés avec une absence de rachats.

Le résultat de ce processus est une fonction continue et lisse qui estime le taux de détention d'assurance-vie pour chaque âge entre 18 et 100 ans, comme illustré par la figure \ref{fig:interpolation_prop_age}.

\begin{figure}[H]
\centering
\includegraphics[width=0.7\textwidth]{images/2_chapitres/chapitre3/interpolation_prop_age.png}
\caption{Proportion de détention d'assurance-vie par âge en France par diverses méthodes d'interpolation. TODO : modifier légende et titre pour plus de lisibilité + refaire la modélisation avec 80 ans et non 82.5 ans}
\label{fig:interpolation_prop_age}
\end{figure}


En multipliant la population de chaque âge par le taux de détention estimé, il a été possible d'obtenir une estimation du nombre d'assurés pour chaque âge et chaque sexe. Après standartisation, on peut alors construire une loi de probabilité empirique.  : $$P(Age = x) = \frac{N_{assures}(x)}{Total_{assures}}$$ 




\paragraph{Calibration d'une loi usuelle sur la loi empirique}

\bigskip

Deux lois de probabilité continues ont été sélectionnées comme candidates pour modéliser la distribution empirique : la \textbf{loi Gamma} et la \textbf{loi Beta}. Les paramètres de ces deux lois ont été estimés par la méthode du maximum de vraisemblance sur un échantillon de 200 000 individus tirés de la loi empirique. Pour déterminer la loi la plus adéquate, des critères visuels et statistiques (Test de Kolmogorov-Smirnov\footnote{Le test de Kolmogorov-Smirnov est un test d'adéquation non paramétrique comparant la fonction de répartition empirique d'un échantillon à celle d'une loi théorique. La statistique $D$ représente l'écart maximal absolu entre ces deux fonctions : une valeur faible indique un bon ajustement.}, AIC\footnote{Le critère d'information d'Akaike (AIC) est une mesure de la qualité d'un modèle statistique qui arbitre entre la qualité de l'ajustement et la complexité du modèle. Il pénalise l'ajout de paramètres pour éviter le surapprentissage. Un AIC plus faible indique un meilleur modèle.}, BIC\footnote{Le critère d'information bayésien (BIC) est similaire à l'AIC mais impose une pénalité plus forte pour le nombre de paramètres, dépendant de la taille de l'échantillon. Il tend à favoriser des modèles plus parcimonieux. Comme pour l'AIC, une valeur plus faible est préférable.}\footnote{L'AIC et le BIC servent à comparer deux ajustements, pour savoir quel modèle est le mieux ajusté on va donc regarder le plus faible AIC/BIC}) ont été utilisés. La figure \ref{fig:beta} montre l'ajustement de la loi Beta qui a des meilleurs résultats statistiques et qui épouse mieux la distribution empirique que la loi Gamma (figure \ref{fig:gamma}). 

\begin{figure}[H]
\centering
\includegraphics[width=0.8 \textwidth]{images/2_chapitres/chapitre3/estimation_loi_beta.png}
\caption{Ajustement de la loi Beta sur la distribution empirique. TODO : Séparer une version homme et une femme et mettre en annexe}
\label{fig:beta}
\end{figure}

Le tableau \ref{tab:stats} confirme cette observation. La loi Beta présente une statistique K-S inférieure (traduisant une distance maximale plus faible entre la distribution théorique et empirique) ainsi que des scores AIC et BIC plus bas, indiquant un meilleur ajustement global. C'est donc cette loi qui a été retenue pour modéliser l'âge des assurés dans le portefeuille synthétique.

\begin{table}[H]
\centering
\begin{tabular}{@{}lccc@{}}
\toprule
\textbf{Métrique} & \textbf{Loi Gamma} & \textbf{Loi Beta} & \textbf{Meilleur Modèle} \\
\midrule
Statistique K-S (D) & 0.0396 & 0.0271 & Beta \\
AIC (Akaike) & 1 674 497 & 1 668 357 & Beta \\
BIC (Bayésien) & 1 674 527 & 1 668 378 & Beta \\
\bottomrule
\end{tabular}
\caption{Tableau comparatif des métriques d'ajustement (population masculine). TODO : mettre aussi les résultats pour la population féminine, mettre également des valeurs en gras}
\label{tab:stats}
\end{table}

\subsubsection{Modélisation de l'âge à la souscription}
Une fois l'âge des assurés modélisé, il est indispensable de déterminer l'âge à la souscription. Cette variable est fondamentale, car elle permet de calculer l'ancienneté du contrat, un paramètre clé qui influence directement les lois de comportement, notamment les taux de rachat, dans les modèles de projection.

La distribution de l'âge à la souscription n'a pas été calibrée sur des données directes, mais a été dérivée de la courbe de taux de détention par âge, établie dans la section précédente. L'hypothèse sous-jacente est que la densité de probabilité de souscrire à un âge donné est proportionnelle à la vitesse à laquelle le taux de détention augmente à cet âge. Autrement dit, la distribution de l'âge à la souscription peut être approximée par la dérivée discrète de la fonction du taux de détention. Par exemple, une forte pente de la courbe de détention entre 25 et 35 ans signale une intense activité de souscription dans cette tranche d'âge. En calculant la différence finie entre les points de la courbe interpolée, il est donc possible de construire une distribution empirique de l'âge à la souscription.

\begin{figure}[H]
\centering
\includegraphics[width=0.9\textwidth]{images/2_chapitres/chapitre3/derivation_dist_souscription.png}
\caption{Passage de la distribution empirique de l'âge à la souscription à la densité de probabilité du taux de souscription.}

\end{figure}

Plusieurs lois de probabilité ont été testées pour modéliser cette distribution empirique de l'âge à la souscription :
\begin{itemize}
    \item La \textbf{loi Gamma} : définie sur $\mathbb{R}^+$, elle est souvent utilisée pour modéliser des durées. Sa densité est $f(x; k, \theta) = \frac{x^{k-1}e^{-\frac{x}{\theta}}}{\theta^k\Gamma(k)}$.
    \item La \textbf{loi Beta} : définie sur un intervalle borné $[a,b]$, elle offre une grande flexibilité de forme. Sa densité standard sur $[0,1]$ est $f(x; \alpha, \beta) = \frac{x^{\alpha-1}(1-x)^{\beta-1}}{B(\alpha, \beta)}$.
    \item La \textbf{loi de Weibull} : courante en analyse de survie, sa densité est donnée par $f(x; \lambda, k) = \frac{k}{\lambda}\left(\frac{x}{\lambda}\right)^{k-1}e^{-(x/\lambda)^k}$.
    \item Le \textbf{Modèle de Mélange Gaussien (GMM)} : il s'agit d'une combinaison linéaire de plusieurs lois normales, permettant de s'ajuster à des distributions multimodales. Sa densité est $f(x) = \sum_{i=1}^{K} w_i \frac{1}{\sigma_i\sqrt{2\pi}} e^{-\frac{1}{2}\left(\frac{x-\mu_i}{\sigma_i}\right)^2}$.
\end{itemize}
Le tableau \ref{tab:stats_souscription} présente les résultats.

\begin{table}[H]
\centering
\begin{tabular}{@{}lcccc@{}}
\toprule
\textbf{Distribution} & \textbf{AIC} & \textbf{BIC} & \textbf{K-S (D)} \\
\midrule
GMM (n=2) & 1 125 603 & 1 125 654 & 0.0485 \\
Beta & 1 385 454 & 1 385 495 & 0.0845 \\
Gamma & 1 400 178 & 1 400 208 & 0.1110 \\
Weibull & 1 437 176 & 1 437 207 & 0.1544 \\
\bottomrule
\end{tabular}
\caption{Tableau comparatif des métriques d'ajustement pour l'âge à la souscription. TODO : mettre également des valeurs en gras}
\label{tab:stats_souscription}
\end{table}

Le \textbf{Mélange Gaussien à deux composantes (GMM)} s'est avéré être le modèle le plus performant, avec des scores AIC et BIC nettement inférieurs. Cela s'explique car 2 pics de souscription sont observables sur la distribution. Cette bimodalité, illustrée par la calibration de la loi GMM en figure \ref{fig:gmm_souscription}, est une caractéristique de marché que les lois de probabilité plus simples ne peuvent capturer.

\begin{figure}[H]
\centering
\includegraphics[width=0.9\textwidth]{images/2_chapitres/chapitre3/calibration_loi_GMM_souscription.png}
\caption{Ajustement d'une GMM sur la distribution de l'âge à la souscription.}
\label{fig:gmm_souscription}
\end{figure}


\subsubsection{Modélisation de la Provision Mathématique (PM)}

Pour calculer la Provision Mathématique à partir des données marché, une méthode plus complexe a été mise en place. Une approche directe consistant à ajuster une loi sur des données de PM n'est pas possible car des données publiques sur ce sujet n'existent pas. Une méthodologie de modélisation conditionnelle a donc été mise en place.

L'hypothèse fondamentale est que la PM d'un individu est principalement fonction de son patrimoine, qui lui-même est fortement corrélé à son âge. La modélisation s'est donc déroulée en plusieurs étapes.


La première étape a consisté à modéliser la distribution du patrimoine brut en fonction de l'âge, en s'appuyant sur les données de l'INSEE \cite{insee_patrimoine_age}. Plutôt que de calibrer une seule loi pour toute la population, une \textbf{loi Lognormale} a été ajustée pour chaque tranche d'âge. Les paramètres de cette loi, $\mu$ et $\sigma$, ont été estimés par la méthode de \textbf{correspondance des quantiles}. Cette méthode est particulièrement adaptée lorsque les données individuelles ne sont pas disponibles et que seules des statistiques agrégées sont fournies. Elle consiste à déterminer les paramètres qui minimisent l'écart quadratique entre les quantiles théoriques de la loi et les quantiles empiriques observés (ici les déciles) :
\begin{equation}
    (\hat{\mu}, \hat{\sigma}) = \arg\min_{\mu, \sigma} \sum_{i=1}^{9} \left( F^{-1}(p_i; \mu, \sigma) - Q_{emp}(p_i) \right)^2
\end{equation}
où $p_i \in \{0.1, \dots, 0.9\}$, $F^{-1}$ est la fonction quantile de la loi Lognormale et $Q_{emp}$ les valeurs fournies par l'INSEE. Cette approche garantit que la distribution ajustée reproduit fidèlement la dispersion de la population réelle. Le tableau \ref{tab:params_patrimoine_age} synthétise les paramètres obtenus.

\begin{table}[H]
\centering
\begin{tabular}{@{}lccc@{}}
\toprule
\textbf{Tranche d'âge} & $\mu$ & $\sigma$ \\
\midrule
Moins de 30 ans & 9.9233 & 1.7713 \\
30 à 39 ans & 11.6750 & 1.8374 \\
40 à 49 ans & 12.1756 & 2.1221 \\
50 à 59 ans & 12.3216 & 2.0551 \\
60 à 69 ans & 12.3579 & 2.0772 \\
70 ans ou plus & 12.2620 & 1.8199 \\
\bottomrule
\end{tabular}
\caption{Paramètres de la loi Lognormale du patrimoine brut, calibrés par tranche d'âge.}
\label{tab:params_patrimoine_age}
\end{table}




Sur la base des calibrations précédentes, une population synthétique de 500 000 individus a été générée. Pour chaque individu, un âge a été tiré selon la loi Bêta déterminée précédemment, puis un patrimoine a été tiré selon la loi Lognormale conditionnelle correspondant à son âge. On obtient ainsi un échantillon de paires (Âge, Patrimoine) respectant la corrélation observée dans la réalité (TODO : rajouter un graphe 3D si c'est possible/lisible).

La deuxième étape consiste à estimer la part du patrimoine de chaque individu allouée à l'assurance-vie. Pour ce faire, les données de l'INSEE sur la composition du patrimoine par décile \cite{insee_patrimoine_age} sont utilisées. La figure \ref{fig:composition_patrimoine} illustre la part du patrimoine financier en fonction du patrimoine brut moyen. Cette relation n'est pas linéaire. Pour les patrimoines les plus modestes, qui n'ont pas encore la capacité d'investir dans l'immobilier, la part des actifs financiers est relativement élevée. À mesure que le patrimoine augmente, une part significative est allouée à l'immobilier, ce qui diminue mécaniquement la part relative du patrimoine financier. Ce phénomène s'inverse cependant pour les patrimoines les plus élevés, où la diversification vers des actifs financiers redevient prépondérante, entraînant une remontée de leur part dans le patrimoine total, ce n'est pas observable sur le graphique \ref{fig:composition_patrimoine} car les données sont agrégées par décile.

\begin{figure}[H]
    \centering
    \begin{tikzpicture}
        \begin{axis}[
            xlabel={Patrimoine brut moyen (€)},
            ylabel={Patrimoine financier (\%)},
            xmode=log,
            log ticks with fixed point,
            xticklabel style={
                /pgf/number format/fixed,
                /pgf/number format/precision=0,
                /pgf/number format/1000 sep={\,}
            },
            yticklabel style={
                /pgf/number format/fixed,
                /pgf/number format/precision=0
            },
            grid=major,
            width=\textwidth,
            height=8cm,
            legend pos=outer north east
        ]
        \addplot[
            smooth,
            mark=*,
            accenture,
        ] coordinates {
            (1900, 31.4)
            (8300, 33.9)
            (21500, 41.7)
            (64300, 43.0)
            (142100, 20.0)
            (211500, 14.4)
            (285900, 14.7)
            (383300, 16.4)
            (559800, 20.2)
            (1487700, 23.2)
        };
        \legend{Part du patrimoine financier}
        \end{axis}
    \end{tikzpicture}
    \caption{Part du patrimoine financier en fonction du patrimoine brut moyen, par décile \cite{insee_patrimoine_age}.}
    \label{fig:composition_patrimoine}
\end{figure}

La troisième étape consiste à lier le patrimoine financier à la Provision Mathématique. Il est en effet plus réaliste de considérer que l'épargne en assurance-vie (ou la provision mathématique du point de vue de l'assureur) constitue une part du patrimoine financier plutôt que du patrimoine brut. Une hypothèse centrale est donc formulée : la PM d'un individu est estimée comme une fraction de son patrimoine financier. Sur la base de données INSEE \cite{insee_patrimoine_age}, cette fraction est fixée à \textbf{40\%}. Ainsi, pour chaque individu de la population simulée, la PM est calculée comme suit :
$$ \text{PM} = \text{Patrimoine Brut} \times \text{Part du Patrimoine Financier} \times 40\% $$
Cette règle permet de transformer la distribution de patrimoine en une distribution de PM, en tenant compte des non-linéarités observées dans la composition du patrimoine.

Après avoir appliqué ce processus à toute la population simulée, nous obtenons un échantillon réaliste de Provisions Mathématiques. Une dernière calibration a montré que la distribution de ces PM pouvait être modélisée de manière très satisfaisante par une \textbf{loi Lognormale}. Bien que les tests d'adéquation classiques rejettent formellement l'hypothèse nulle en raison de la très grande taille de l'échantillon (rendant le test extrêmement sensible aux infimes déviations), l'analyse comparative des critères d'information et de la distance K-S confirme la supériorité de la loi Lognormale sur cet ajustement, comme l'indique le tableau \ref{tab:stats_pm}.

\begin{table}[H]
\centering
\begin{tabular}{@{}lccc@{}}
\toprule
\textbf{Distribution} & \textbf{AIC} & \textbf{BIC} & \textbf{K-S (D)} \\
\midrule
Lognorm & 7 142 088 & 7 142 120 & 0.017 \\
Gamma & 7 254 743 & 7 254 774 & 0.145 \\
\bottomrule
\end{tabular}
\caption{Pour le critère de Kolmogorov-Smirnov, la loi Lognormale est largement meilleure que la loi Gamma pour estimer cette distribution.}
\label{tab:stats_pm}
\end{table}


\subsection{Synthèse et Génération du Portefeuille Final}

L'étape finale de la modélisation consiste à assembler les différentes distributions marginales et conditionnelles calibrées précédemment pour générer le portefeuille de passifs complet. Contrairement à une approche purement multivariée (copule globale 3D) qui peinerait à capturer la nature conditionnelle de la richesse par rapport à l'âge, une approche hybride a été implémentée.

\subsubsection{Modélisation de la dépendance des âges par Copule}

L'analyse des données montre une corrélation mécanique et comportementale entre l'âge de l'assuré et son âge à la souscription. Pour reproduire cette structure de dépendance sans figer les distributions marginales, une \textbf{Copule Gaussienne bivariée} a été utilisée.

Cette copule permet de générer des paires de rangs corrélés $(u, v) \in [0, 1]^2$ avec un paramètre de corrélation de Spearman calibré à $\rho = 0.7$. Ces rangs sont ensuite transformés en valeurs réelles via les fonctions quantiles inverses (PPF) des lois marginales retenues :
\begin{itemize}
    \item $Age_{assure} = F^{-1}_{Beta}(u)$
    \item $Age_{souscription} = F^{-1}_{GMM}(v)$
\end{itemize}

Une contrainte métier stricte est appliquée post-génération : l'âge à la souscription doit nécessairement être inférieur ou égal à l'âge actuel de l'assuré ($Age_{souscription} \leq Age_{assure}$). Les paires ne respectant pas cette condition sont rejetées et régénérées, assurant ainsi la cohérence temporelle des contrats.

\subsubsection{Génération en cascade de la Provision Mathématique}

Une fois le profil d'âge de l'assuré fixé, la Provision Mathématique (PM) est générée de manière conditionnelle, reproduisant la chaîne causale identifiée lors de la calibration :
\begin{enumerate}
    \item \textbf{Génération du Patrimoine Global :} Pour un âge donné, un montant de patrimoine est tiré aléatoirement dans la loi Lognormale spécifique à la tranche d'âge de l'assuré (cf. section précédente).
    \item \textbf{Test de Détention :} Une épreuve de Bernoulli détermine si l'individu détient un contrat d'assurance-vie, avec une probabilité dépendant de son niveau de patrimoine.
    \item \textbf{Calcul de la PM :} Pour les détenteurs, la PM est calculée par application déterministe des ratios calibrés :
    $$ \text{PM} = \text{Patrimoine} \times \%_{\text{Financier}}(\text{Patrimoine}) \times 40\% $$
\end{enumerate}

\subsubsection{Historique des taux techniques et des TMG retenus pour la modélisation}

La détermination des \emph{taux minimums garantis} (TMG) par année de souscription s'appuie sur les plafonds réglementaires de taux techniques définis par le Code des assurances, tels qu'analysés dans la note \og Le taux technique en assurance vie \fg{} publiée par l'ACPR \cite{acpr66},et les travaux de l'ACPR sur l'assurance vie en environnement de taux bas \cite{acprTauxBas}, complétés par la veille de marché de Good Value for Money sur les caractéristiques des contrats d'assurance vie \cite{gvmTMG}.

La note \cite{acpr66} rappelle notamment que, depuis les années 1990, le taux technique maximal autorisable sur les contrats d'assurance vie est encadré par une formule fonction du taux moyen des emprunts d'État (TME), avec un plafond de l'ordre de 4{,}5\,\% au début des années 1990, puis une baisse progressive de ce plafond au fil de la diminution des taux d'intérêt. Les travaux ultérieurs de l'ACPR montrent qu'en environnement de taux bas, le taux technique moyen effectivement pratiqué sur le stock de contrats s'est progressivement rapproché de 0\,\%, avec un taux moyen de l'ordre de 0{,}35--0{,}5\,\% au milieu des années 2010 \cite{acprTauxBas}.

Sur la base de ces éléments réglementaires et de marché, nous construisons une série historique de TMG par année de souscription. Pour chaque année $t$, nous retenons :
\begin{itemize}
  \item un \emph{taux technique maximal réglementaire} $TT^{\max}_t$, issu des plafonds ACPR/Code des assurances ;
  \item un $TMG_t$ effectivement utilisé dans notre modèle, calibré en dessous de ce plafond, de façon à représenter un contrat \og typique \fg{} de l'année considérée.
\end{itemize}

Le tableau~\ref{tab:TMG_historique} présente la série retenue pour la période 1993--2024.

On observe alors que, jusqu'au milieu des années 1990, les plafonds de taux techniques autorisent des TMG jusqu'à 4{,}5\,\%, ce qui est cohérent avec les niveaux de TMG donnés dans contrats de l'époque tels que décrits dans \cite{acpr66,gvmTMG}. À partir de la fin des années 1990 et des années 2000, la baisse des taux d'intérêt se traduit par une réduction progressive des plafonds réglementaires et des TMG pratiqués, les valeurs retenues $TMG_t$ restant systématiquement inférieures à $TT^{\max}_t$ afin de tenir compte d'un certain conservatisme des assureurs.

À partir de 2015, nous faisons l'hypothèse de TMG nuls ($TMG_t = 0$) pour les nouveaux contrats, ce qui reflète l'environnement de taux durablement bas documenté par l'ACPR \cite{acprTauxBas} et la généralisation des contrats d'épargne en assurance vie sans garantie explicite de taux (garantie limitée au capital net de frais, la performance étant portée par la participation aux bénéfices). Cette hypothèse est cohérente avec les analyses de marché montrant que, sur la période récente, la quasi-totalité des nouveaux contrats d'épargne individuelle n'affichent plus de taux minimum garanti significatif \cite{gvmTMG}.

Dans le cadre de la modélisation ALM présentée dans la suite, cette série annuelle de TMG constitue la base pour la construction de profils discrets par génération de contrat : pour chaque année de souscription $t$, trois profils de contrat sont définis autour de $TMG_t$ (un profil \og généreux \fg{}, un profil médian et un profil conservateur), de façon à représenter la dispersion observée des TMG dans une même cohorte tout en conservant un cadre de simulation discret adapté à l’agrégation de passifs.

\subsubsection{Intégration du TAF et du TFGSE dans la modélisation des TMG}

Au-delà du seul $TMG_t$, la valorisation des engagements en assurance vie dépend fortement des \emph{frais de gestion prélevés sur les encours} et de la \emph{participation aux bénéfices} (PB) effectivement redistribuée aux assurés. Pour capturer ces deux dimensions dans un cadre de modélisation discret, nous introduisons deux paramètres complémentaires pour chaque contrat :
\begin{itemize}
  \item le taux de frais de gestion sur encours $TFGSE$, appliqué annuellement sur la provision mathématique ;
  \item le taux d'affectation des produits financiers $TAF$, représentant la part des produits
        financiers attribuée aux assurés (taux servi et dotation à la provision pour participation
        aux bénéfices).
\end{itemize}

Les données de marché publiées par France Assureurs sur l'assurance vie en unités de compte montrent qu'en 2024, les frais de gestion sur encours des contrats en unités de compte s'élèvent en moyenne à 0{,}88\,\% par an, avec une valeur de 0{,}83\,\% pour les gestions libres ou pilotées sans surcoût et un surcoût moyen de 0{,}36\,\% pour la gestion sous mandat \cite{faUC2024}. Pour les supports en euros, les mêmes travaux indiquent un taux de frais de gestion d'environ 0{,}66\,\% par an, ce qui situe la moyenne de marché entre 0{,}6\,\% et 0{,}7\,\% selon le type de contrat \cite{faUC2024}. Ces ordres de grandeur sont cohérents avec les comparatifs de frais publiés par différents acteurs de place, qui situent la plupart des contrats dans une fourchette de 0{,}5\,\% à 1{,}0\,\% de frais de gestion annuels sur encours.

Dans le modèle, afin de rester cohérent avec cette distribution observée tout en conservant une structure discrète compatible avec l'agrégation ALM, nous retenons trois niveaux de $TFGSE$ :
\[
  TFGSE \in \{0{,}50\%,\ 0{,}70\%,\ 0{,}90\%\}.
\]
Ces trois valeurs représentent respectivement :
\begin{itemize}
  \item un profil \og peu chargé \fg{} ($0{,}50\,$\%), typique de contrats compétitifs ou de gammes patrimoniales ;
  \item un profil \og moyen \fg{} ($0{,}70\,$\%), proche de la moyenne observée sur les fonds en euros ;
  \item un profil \og chargé \fg{} ($0{,}90\,$\%), représentant des contrats grand public avec une structure de frais élevée.
\end{itemize}
Dans chaque cohorte d'année $t$, la probabilité d'appartenance à l'un de ces trois profils est ajustée de manière à ce que la moyenne pondérée de $TFGSE$ reste compatible avec les statistiques agrégées de marché, tout en reflétant une amélioration progressive des grilles tarifaires sur les générations récentes de contrats.

Concernant la participation aux bénéfices, le cadre réglementaire impose une redistribution d'au moins 85\,\% des bénéfices financiers et 90\,\% des bénéfices techniques au profit des assurés, sous forme de PB et de dotation à la provision pour participation aux bénéfices (PPB). Les analyses de l'ACPR sur la revalorisation des contrats d'assurance vie mettent en évidence que, sur longue période, la part des produits financiers effectivement affectée aux assurés (taux servi + variation de PPB) se situe fréquemment au-delà de 80\,\% du résultat financier, les marges techniques des assureurs restant limitées \cite{acprTauxBas,acprRevalo2024}.

Nous modélisons cette réalité au moyen d'un taux d'affectation des produits financiers $TAF$ discret, qui représente la part des produits financiers attribuée aux assurés sur un exercice donné. Trois niveaux sont retenus :
\[
  TAF \in \{80\%,\ 90\%,\ 95\%\}.
\]
Ces niveaux correspondent à :
\begin{itemize}
  \item $80\,$\% : un comportement proche du plancher réglementaire, associé à des contrats peu généreux
        en termes de participation aux bénéfices ;
  \item $90\,$\% : un comportement \og moyen \fg{} de marché, cohérent avec les constats de l'ACPR
        sur des marges techniques modérées ;
  \item $95\,$\% : un profil particulièrement favorable aux assurés, typique de certains contrats
        haut de gamme ou de gammes associatives affichant des taux servis durablement supérieurs
        à la moyenne du marché.
\end{itemize}

Pour assurer la cohérence économique du modèle, une dépendance entre $TFGSE$ et $TAF$ sera introduite au travers de profils de contrat. Concrètement, pour chaque contrat synthétique, nous tirons d'abord un profil de frais parmi $\{0{,}50\%, 0{,}70\%, 0{,}90\%\}$, puis un niveau de $TAF$ parmi les trois valeurs $\{80\%, 90\%, 95\%\}$, avec une loi discrète conditionnelle. Par exemple, un contrat à faibles frais ($TFGSE = 0{,}50\,$\%) aura une probabilité plus forte d'être associé à un $TAF$ élevé (90--95\,\%), tandis qu'un contrat fortement chargé ($TFGSE = 0{,}90\,$\%) sera plus souvent associé à un $TAF$ dans la partie basse de la grille (80--85\,\%). Ce couplage reproduit qualitativement le fait que les contrats les plus compétitifs en frais sont également ceux qui, historiquement, affichent les meilleurs taux servis et une politique de PB plus favorable aux assurés.

Au total, pour une année de souscription $t$ donnée, chaque contrat du portefeuille synthétique est donc caractérisé par un triplet discret $(TMG_t, TFGSE, TAF)$, où :
\begin{itemize}
  \item $TMG_t$ est fixé par la table historique de la section précédente, construite à partir des plafonds de taux techniques ACPR ;
  \item $TFGSE$ prend l'une des trois valeurs $\{0{,}50\%, 0{,}70\%, 0{,}90\%\}$ ;
  \item $TAF$ prend l'une des trois valeurs $\{80\%, 90\%, 95\%\}$.
\end{itemize}
Ce choix de discrétisation permet de simplifier la génération du portfeuilles sur les aspects financiers et assurantiels tout en restant ancré sur des ordres de grandeur observés sur le marché français de l'assurance vie, tels que documentés par France Assureurs et l'ACPR \cite{acprTauxBas,acprRevalo2024,faUC2024}.

\subsubsection{Corrélation des caractéristiques contractuelles avec la provision mathématique}

Dans le modèle initial, les paramètres $(TMG_t, TFGSE, TAF)$ sont tirés indépendamment de
l'année de souscription $t$. Cependant, pour refléter fidèlement la structure du portefeuille
d'un assureur français, il est pertinent d'introduire une dépendance entre ces caractéristiques
contractuelles et la \emph{provision mathématique} du contrat $PM_k$, qui sert de proxy naturel
pour segmenter les contrats par gamme commerciale (Mass Market, Patrimonial, Gestion Privée).

Cette approche s'appuie sur deux observations de marché :
\begin{enumerate}
  \item Les frais de gestion sur encours $TFGSE$ sont généralement \emph{dégressifs} avec l'encours
        du contrat : les contrats à forte provision mathématique (typiquement $>$ 100k€,
        voire $>$ 500k€ pour la Gestion Privée) bénéficient de grilles tarifaires plus favorables
        \cite{faUC2024}.
  \item Les contrats à gros encours sont historiquement plus concentrés dans les gammes patrimoniales
        et de Gestion Privée, qui affichent des politiques de participation aux bénéfices (PB) et
        des taux minimums garantis (TMG) différenciés par rapport aux contrats retail
        (petits patrimoines).
\end{enumerate}

\title{Définition des seuils de provision mathématique}

Nous segmentons les contrats selon trois tranches de provision mathématique $PM_k$ :
\begin{center}
\begin{tabular}{lcc}
  \hline
  \textbf{Gamme} & \textbf{Tranche $PM_k$} & \textbf{Exemple typique} \\
  \hline
  Mass Market / Retail & $PM_k \leq 50\,000€$ & Contrats petit patrimoine \\
  Patrimonial & $50\,000€ < PM_k \leq 500\,000€$ & Clientèle moyenne/haute gamme \\
  Gestion Privée & $PM_k > 500\,000€$ & HNWI, family offices \\
  \hline
\end{tabular}
\end{center}

Ces seuils sont inspirés de la segmentation commerciale standard des acteurs français
(AXA, CNP Assurances, Generali) et cohérents avec les statistiques agrégées de France Assureurs
sur la structure par encours des contrats d'épargne individuelle \cite{faUC2024}.

\title{Lois conditionnelles par tranche de provision}

Pour chaque contrat généré, après tirage de l'année $t$ et de la provision mathématique $PM_k$,
les paramètres $(TFGSE, TAF)$ sont tirés selon la loi conditionnelle suivante :

\begin{table}[h!]
\centering
\caption{Lois discrètes conditionnelles de $(TFGSE, TAF)$ selon la tranche de provision mathématique}
\label{tab:TFGSE_TAF_PM}
\begin{tabular}{ll|ccc}
  \hline
  \multirow{2}{*}{\textbf{Gamme}} & \multirow{2}{*}{\textbf{Tranche $PM_k$}} &
    \multicolumn{3}{c}{\textbf{Probabilités $P(TFGSE, TAF|PM_k)$}} \\
  & & $(0{,}5\%,90\%)$ & $(0{,}7\%,90\%)$ & $(0{,}9\%,80\%)$ \\
  \hline
  Mass Market & $PM_k \leq 50k€$ & 0,20 & 0,40 & 0,40 \\
  Patrimonial & $50k€ < PM_k \leq 500k€$ & 0,40 & 0,40 & 0,20 \\
  Gestion Privée & $PM_k > 500k€$ & 0,60 & 0,30 & 0,10 \\
  \hline
\end{tabular}
\end{table}

Le TMG reste fonction de l'année $t$ uniquement ($TMG_t$ de la table~\ref{tab:TMG_historique}),
mais la pondération des générations anciennes (TMG élevés) sera naturellement plus forte dans les
tranches Gestion Privée, car ces contrats ont eu plus de temps pour accumuler des provisions élevées.

\title{Algorithme de génération mis à jour}

L'algorithme de génération de portefeuille devient :
\begin{enumerate}
  \item Tirer l'année de souscription $t \in [1993,2024]$ selon la structure d'âge du portefeuille ;
  \item Tirer la provision mathématique $PM_k$ (loi lognormale typique des encours assurance vie) ;
  \item Identifier la tranche de $PM_k$ et tirer le couple $(TFGSE, TAF)$ selon la table~\ref{tab:TFGSE_TAF_PM} ;
  \item Assigner $TMG_t$ selon la table historique ;
  \item Calculer le TMG net : $TMG_t^{net} = \max(TM G_t - TFGSE, 0)$.
\end{enumerate}

\title{Vérification de cohérence avec les données agrégées}

Cette structure assure plusieurs cohérences observables :
\begin{itemize}
  \item \textbf{TFGSE moyen} : la moyenne pondérée par les tranches de $PM_k$ retombe autour de
        0,65--0,70\,\%, cohérent avec France Assureurs (fonds euros : 0,66\,\%) \cite{faUC2024}.
  \item \textbf{TMG moyen} : les contrats anciens (TMG $>$ 2\,\%) sont naturellement plus présents
        dans les tranches à gros $PM_k$, ce qui est cohérent avec le taux technique moyen actuel
        du portefeuille (0,35--0,5\,\%) rapporté par l'ACPR \cite{acprTauxBas}.
  \item \textbf{Relation frais-rendement} : les gros contrats (GP) ont à la fois des frais plus bas
        et un TAF plus élevé, reproduisant le différentiel de performance observé entre gammes.
\end{itemize}

Cette approche par corrélation avec la provision mathématique permet ainsi de générer un
portefeuille synthétique dont les caractéristiques microéconomiques (par contrat) sont
cohérentes avec les statistiques macroéconomiques publiées, tout en conservant une structure
parfaitement discrète et donc compatible avec l'agrégation ALM.


\subsubsection{Ajout du sexe pour finaliser le portefeuille}

Pour finaliser le portefeuille, le sexe est une variable important à modéliser. Il est généré aléatoirement selon une loi de Bernoulli, calibrée sur la répartition hommes/femmes observée dans la population des assurés. En France, les données de l'INSEE indiquent que la répartition est légèrement en faveur des femmes, avec environ 52\,\% de femmes et 48\,\% d'hommes parmi les assurés en assurance vie \cite{insee_assurance_vie}. Cette répartition est donc utilisée pour générer le sexe de chaque individu du portefeuille synthétique, en tirant une variable binaire où $P(Sexe = Femme) = 0.52$ et $P(Sexe = Homme) = 0.48$.

Ce processus itératif permet de générer un portefeuille synthétique de 1 000 000 contrats présentant des caractéristiques statistiques fidèles à la réalité du marché français, prêt à être utilisé pour les étapes d'agrégation.

\subsection{Mise en correspondance avec le format final du modèle ALM}
Une fois les caractéristiques individuelles générées (âge, âge à la souscription, PM, TMG, TAF, TFGSE, sexe), il est nécessaire de les formater pour les rendre compatibles avec les exigences d'entrée du modèle ALM utilisé pour la projection des passifs. Cela implique notamment de :
\begin{itemize}
    \item Discrétiser les variables continues (âge, PM) en classes d'âge et de PM adaptées à la granularité du modèle ALM.
    \item Assigner à chaque contrat un profil de frais et de participation aux bénéfices en fonction des règles définies précédemment.
    \item Structurer les données dans un format tabulaire ou matriciel compatible avec les algorithmes d'agrégation et de projection utilisés dans le modèle ALM.
\end{itemize}


\section{Présentation du Portefeuille de Référence Généré}

\subsection{Analyse descriptive du portefeuille de référence}

\subsection{Limitations du portefeuille généré}




\chapter{Présentation des méthodes d'agrégation candidates}

Après avoir établi le cadre réglementaire et construit un générateur de passif capable de produire des données réalistes, il convient désormais d'aborder la problématique centrale de ce mémoire : la réduction de la dimension du portefeuille de passif. 

La projection individuelle de chaque contrat au sein des modèles ALM requiert des ressources computationnelles excessives ; l'agrégation du passif s'impose donc comme une nécessité technique pour les assureurs. Cette nécessité découle directement des exigences de réactivité lors des phases de clôture réglementaire Solvabilité 2, où la finesse de l'analyse des risques doit impérativement se concilier avec la performance d'exécution pour permettre le lancement de multiples sensibilités et scénarios économiques.

L'enjeu sous-jacent est de résoudre un problème complexe d'optimisation sous contraintes : comment réduire drastiquement la volumétrie des données en entrée du modèle ALM sans altérer la fidélité des indicateurs de risque prudentiels, à savoir le \textit{Best Estimate} (BE) et le \textit{Solvency Capital Requirement} (SCR) ?

Pour répondre à cette problématique, l'objectif de ce chapitre est de définir et d'exposer de manière exhaustive les fondements mathématiques et algorithmiques de différentes approches d'agrégation qui seront mises en concurrence dans ce mémoire. La démarche s'articulera autour de l'étude de trois grandes familles de méthodes :

\begin{itemize}
    \item \textbf{Les approches déterministes classiques :} Méthodes traditionnellement utilisées sur le marché, elles reposent sur une segmentation experte du portefeuille par caractéristiques communes ou par tranches (\textit{Banding}). Elles serviront de référence (ou \textit{benchmark}) pour évaluer l'apport des techniques plus complexes.
    \item \textbf{Les approches statistiques par apprentissage non supervisé :} L'apport des algorithmes de \textit{Machine Learning}, tels que les algorithmes de partitionnement (K-Means), basés sur la densité (DBSCAN, HDBSCAN) ou les arbres de décision (CART), sera exploré pour constituer des \textit{Model Points} fondés sur une similarité multidimensionnelle des profils de risque.
    \item \textbf{Les approches dynamiques basées sur les flux financiers :} En rupture avec le regroupement spatial statique, ces méthodes avancées (telles que le \textit{Cash-Flow Matching} et l'approche par calibration \textit{a posteriori}) proposent d'agréger les contrats en fonction de leur comportement projeté et de leur cinétique d'écoulement.
\end{itemize}

L'exposé détaillé de ces modèles nous permettra de poser le socle théorique nécessaire avant d'aborder, dans le chapitre suivant, le protocole de test standardisé et l'analyse comparative de leurs performances respectives.

\section{Les approches déterministes : Le standard de marché}
\subsection{Segmentation stricte par regroupement de caractéristiques}

La méthode par regroupement de caractéristiques repose sur une segmentation par classes de risque, cette approche constitue le standard historique en actuariat. Elle servira de ba
se de comparaison pour évaluer la performance des algorithmes de regroupement statistique plus avancés présentés dans les parties suivantes.

Le principe fondamental repose sur une partition déterministe du portefeuille. L'objectif est de regrouper les contrats partageant des caractéristiques de risque identiques. Dans 
un modèle de projection de gestion actif-passif, l'évolution de la Provision Mathématique (PM) est influencée par des caractéristiques spécifiques du contrat. La mortalité dépend 
de la génération, de l'âge et du sexe ; les rachats sont corrélés à l'âge de l'assuré et à l'ancienneté du contrat (fiscalité) ; la revalorisation dépend du Taux Minimum Garanti (
TMG).

Par conséquent, pour garantir que le Model Point agrégé se comporte comme la somme des contrats individuels, il est impératif de regrouper les polices selon ces axes discriminants
. Contrairement aux méthodes statistiques qui cherchent des similarités globales, cette approche applique une segmentation rigide. Les critères de regroupement retenus pour cette 
étude sont :
\begin{itemize}
    \item \textbf{Le Sexe} : Indispensable pour l'application des tables de mortalité différenciées.
    \item \textbf{Le Taux Minimum Garanti (TMG)} : Crucial pour la valorisation des garanties financières.
    \item \textbf{L'Âge de l'assuré} : Discrétisé à l'entier inférieur pour correspondre aux entrées des tables de mortalité.
    \item \textbf{L'Ancienneté du contrat} : Discrétisée à l'entier inférieur pour modéliser correctement la fiscalité et les rachats structurels.
\end{itemize}

Mathématiquement, cette méthode définit une relation d'équivalence stricte. Deux contrats $i$ et $j$ appartiennent au même Model Point $k$ si et seulement si leurs vecteurs de car
actéristiques discrétisées sont identiques :
\begin{equation}
(S_i, \lfloor A_i \rfloor, \lfloor Anc_i \rfloor, TMG_i) = (S_j, \lfloor A_j \rfloor, \lfloor Anc_j \rfloor, TMG_j)
\end{equation}

Où $S_i, A_i$ et $Anc_i$ désignent respectivement le sexe, l'âge et l'ancienneté du contrat $i$. Une fois les groupes constitués, l'agrégation s'opère par la somme des variables extensives. La Provision Mathématique du Model Point $k$ (associé à la classe d'équivalence $C_k$) est alors la somme des encours des contrats qui le constituent :
\begin{equation}
PM_{MP_k} = \sum_{i \in C_k} PM_i
\end{equation}
Les variables intensives du Model Point (dont la valeur est indépendante de la taille du groupe, comme l'Âge ou l'Ancienneté) prennent alors les valeurs définies par la segmentation (par exemple, 45 ans et 10 ans), et non une moyenne pondérée, ce qui est cohérent avec la logique de discrétisation à l'entier.

Cette approche présente l'avantage majeur de la simplicité et de la transparence. Elle conserve exactement la volumétrie financière du portefeuille et respecte scrupuleusement les garanties contractuelles (absence de dilution du TMG par moyenne). Elle garantit une homogénéité parfaite des assurés au sein d'un groupe.

Cependant, elle présente des limites intrinsèques :
\begin{itemize}
    \item \textbf{La rigidité de la structure de regroupement :} La définition des classes est fixée \textit{a priori} et de manière uniforme, sans tenir compte de la distribution réelle des capitaux. Cette approche ne permet pas de concentrer automatiquement la précision du regroupement sur les zones à fort enjeux financiers, traitant potentiellement avec la même granularité les segments marginaux et les segments prépondérants.
    \item \textbf{L'absence de contrôle sur la compression :} Le nombre de Model Points finaux n'est pas paramétrable. Il dépend exclusivement de la dispersion du portefeuille et de la finesse de la segmentation. Sur un portefeuille très hétérogène, cette méthode peut générer un nombre de groupes très élevé, dont certains ne contiendront que peu de contrats, limitant ainsi l'efficacité de la réduction de dimension (phénomène du « fléau de la dimension »).
\end{itemize}

\subsection{Approche par tranches (Banding) et pondération financière}

Cette méthode constitue une évolution de la segmentation par classes présentée précédemment. Elle vise à réduire davantage le nombre de Model Points en relâchant la contrainte d'égalité stricte sur les variables continues (âge et ancienneté) au profit d'une logique d'intervalles, ou discrétisation par paliers.

Le principe de regroupement reste déterministe. Les variables catégorielles ou contractuelles majeures (Sexe, TMG) conservent une discrimination stricte. En revanche, l'espace des variables temporelles est découpé en tranches. Dans le cadre de cette implémentation, les intervalles suivant ont été retenus :
\begin{itemize}
    \item Un pas de 2 ans pour l'âge de l'assuré ;
    \item Un pas de 5 ans pour l'ancienneté du contrat.
\end{itemize}

Un contrat $i$ appartient à un groupe $k$ si ses caractéristiques discrètes correspondent et si ses variables continues tombent dans les intervalles définis :

\begin{equation}
i \in C_k \iff 
\begin{cases} 
S_i = S_k \\ 
TMG_i = TMG_k \\ 
A_i \in [A_{min}^k, A_{max}^k[ \\ 
Anc_i \in [Anc_{min}^k, Anc_{max}^k[ 
\end{cases}
\end{equation}

La spécificité majeure de cette approche réside dans la détermination des caractéristiques du Model Point. Contrairement à une approche simpliste qui retiendrait le centre de l'intervalle, il convient de calculer le barycentre financier des contrats regroupés. Ainsi, l'âge ($A_{MP_k}$) et l'ancienneté ($Anc_{MP_k}$) du Model Point $k$ sont calculés comme suit :

\begin{equation}
A_{MP_k} = \frac{\sum_{i \in C_k} A_i \cdot PM_i}{\sum_{i \in C_k} PM_i} \quad ; \quad Anc_{MP_k} = \frac{\sum_{i \in C_k} Anc_i \cdot PM_i}{\sum_{i \in C_k} PM_i}
\end{equation}

Cette pondération par les encours permet de s'assurer que le Model Point est représentatif des contrats les plus significatifs financièrement au sein de la tranche, minimisant ainsi le biais d'agrégation sur les projections de flux futurs.

\subsubsection{Variante Smart Granular et limites (Biais d'agrégation et variance intra-classe)}
Dans le cadre de l'optimisation, une variante nommée \textbf{Smart Granular} a été développée. Contrairement au Banding classique à pas constant, cette approche adapte la finesse des mailles selon la sensibilité locale des risques. Plusieurs profils de découpage ont été implémentés pour répondre à des objectifs d'analyse différents :
\begin{itemize}
    \item \textbf{Smart Focus Active :} Priorise la précision sur la phase de constitution de l'épargne (20-65 ans) avec un pas de 2 ans, tout en compressant fortement les extrêmes (pas de 15 ans au-delà de 85 ans).
    \item \textbf{Smart Focus Young :} Stratégie ultra-fine sur les nouveaux entrants (pas de 1 an jusqu'à 50 ans) pour capturer la dynamique de croissance du portefeuille.       
    \item \textbf{Smart Focus Fiscal :} Découpage spécifique autour de l'ancienneté 8 ans (pas de 1 an entre 0 et 12 ans) pour modéliser avec précision le basculement de la fiscalité des rachats.
    \item \textbf{Smart Granular (Équilibré) :} Mélange des approches précédentes, resserrant les mailles à la fois sur l'âge et sur le cap fiscal des 8 ans.
\end{itemize}
\subsubsection{Limites de l'approche et variance intra-classe}
Bien que la pondération par la Provision Mathématique assure la conservation du barycentre financier, le regroupement par tranches (Banding) écrase inévitablement la dispersion au sein de chaque groupe. Mathématiquement, la variance intra-classe (ou inertie) d'une tranche $k$ pour une caractéristique $X$ (comme l'âge) s'exprime par :

\begin{equation}
 Var(X_k) = \frac{1}{\sum_{i \in MP_k} PM_i} \sum_{i \in MP_k} PM_i (X_i - X_{MP_k})^2 
\end{equation}
Plus la tranche est large, plus cette variance augmente, ce qui dilue la précision du profil de risque. Le modèle ALM traitant les flux de manière non linéaire (notamment via les probabilités de rachat ou de mortalité), cette perte de dispersion engendre le fameux « biais d'agrégation ». C'est cette limitation fondamentale qui justifie le recours à des algorithmes de partitionnement dynamique. Plutôt que de subir une grille d'agrégation statique et arbitraire qui contraint la donnée, il apparaît nécessaire d'adopter des approches où la structure intrinsèque du portefeuille dicte elle-même la formation des groupes. Les méthodes d'apprentissage non supervisé, explorées dans la section suivante, répondent précisément à cet impératif en cherchant à minimiser l'inertie intra-classe de manière algorithmique.


\section{Les approches statistiques spatiales (Apprentissage Non Supervisé)}

\subsection{Préparation de l'espace spatial : Techniques transverses}
Au-delà du choix de l'algorithme de clustering, la performance de l'agrégation dépend fortement de la préparation des données et des contraintes métier imposées lors du processus.

\subsubsection*{Stratification par Risques Majeurs (TMG et Sexe)}
Afin d'éviter des compensations de risques biologiquement ou financièrement absurdes, une stratégie de \textbf{pré-split} a été systématiquement appliquée. Le portefeuille est d'abord divisé en strates étanches selon :
\begin{itemize}
    \item \textbf{Le Sexe :} Pour garantir le respect des tables de mortalité différenciées.
    \item \textbf{Le Taux Minimum Garanti (TMG) :} Pour éviter de diluer la valeur des garanties financières.
\end{itemize}
L'algorithme de clustering (K-Means ou HDBSCAN) est ensuite lancé indépendamment au sein de chaque strate. Cette approche garantit une « pureté » minimale des Model Points finaux.

\subsubsection*{Ingénierie des caractéristiques : Log-PM et Poids Dimensionnels}
Pour orienter les algorithmes vers les zones à fort enjeux, deux techniques de \textit{Feature Engineering} ont été testées :
\begin{itemize}
    \item \textbf{Ajout de la dimension Log-PM :} Le logarithme de la Provision Mathématique est ajouté comme variable spatiale. Cela force l'algorithme à regrouper des contrats de taille financière similaire, évitant qu'un « gros » contrat ne soit noyé dans une masse de « petits » contrats.
    \item \textbf{Pondération des dimensions :} Lors du calcul de la distance euclidienne, des poids $w_d$ sont appliqués. Par exemple, une importance de 10 est donnée au TMG contre 1 pour l'Âge, pour s'assurer que la proximité financière prime sur la proximité démographique.
\end{itemize}

\subsection{Partitionnement par inertie : \textit{k}-moyennes (K-Means) pondéré}

Contrairement aux méthodes déterministes qui segmentent l'espace des risques selon une grille préétablie, les approches par apprentissage non supervisé (Machine Learning) visent à déterminer la structure des données. L'objectif n'est plus d'imposer un regroupement, mais de laisser l'algorithme identifier les zones de forte densité du portefeuille pour y placer les groupes semblables, dans notre cas les Model Points.

Parmi les algorithmes de clustering, la méthode des K-Means a été retenue car c'est une méthode robuste et fonctionne très bien pour minimiser l'inertie intra-classe, c'est-à-dire la dispersion des contrats autour de leur Model Point représentatif.

\subsubsection{Principe de l'algorithme}
L'algorithme des K-Means cherche à partitionner un ensemble de $N$ contrats en $K$ groupes (ou clusters) distincts, de manière à minimiser la distance entre chaque contrat et le centre de son groupe (le centroïde). Dans notre contexte, ce centroïde deviendra le Model Point.

Mathématiquement, pour un ensemble de contrats représentés par des vecteurs de caractéristiques $x_i \in \mathbb{R}^d$ (où $d$ est le nombre de dimensions : Âge, Ancienneté, TMG...), l'algorithme cherche à déterminer les $K$ centroïdes $\mu_1, ..., \mu_K$ qui minimisent la fonction objectif $J$ (l'inertie) :


\begin{equation}
J = \sum_{j=1}^{K} \sum_{x_i \in C_j} w_i \left\lVert x_i - \mu_j \right\rVert^2
\end{equation}

Où :
\begin{itemize}
    \item $C_j$ est l'ensemble des contrats assignés au cluster $j$.
    \item $|| . ||$ est la distance/norme utilisée (ici norme euclidienne).
    \item $w_i$ est le poids du contrat $i$, ici cela correspond à la Provision Mathématique ($PM_i$). Ainsi, l'algorithme est forcé de minimiser l'erreur de regroupement prioritairement pour les contrats à fort enjeux financiers.
\end{itemize}

\subsubsection{Application aux données du portefeuille}
La mise en œuvre de cet algorithme sur un portefeuille d'épargne nécessite plusieurs étapes pour garantir la pertinence des regroupements : le choix des variables, la standardisation des données, et l'adaptation de l'algorithme pour intégrer la pondération par la PM.

Les variables retenues pour le calcul de la distance sont celles qui impactent directement le profil de risque et les flux futurs :
\begin{itemize}
    \item \textbf{L'Âge de l'assuré} ;
    \item \textbf{L'Ancienneté fiscale} ;
    \item \textbf{Le Taux Minimum Garanti (TMG)}.
\end{itemize}

Ces variables ayant des échelles très différentes (un âge varie de 0 à 100, un TMG de 0 à 0.04), un calcul de distance brut donnerait un poids disproportionné à l'âge. Il est donc impératif de procéder à une standardisation (centrage-réduction) des données avant le clustering :

\begin{equation}
 \tilde{x}_{i, d} = \frac{x_{i, d} - \mu_d}{\sigma_d} 
\end{equation}
Cette transformation place toutes les variables sur une échelle comparable, permettant à l'algorithme de traiter équitablement les différentes dimensions du risque.

L'algorithme utilisé est une variante pondérée des K-Means. Contrairement à une approche standard où chaque point a une importance égale, ici chaque contrat « attire » le centroïde proportionnellement à sa PM. Le processus itératif est le suivant :
\begin{enumerate}
    \item \textbf{Initialisation :} Sélection des $K$ centroïdes initiaux via la méthode \textbf{k-means++}. Contrairement à une initialisation totalement aléatoire qui risque de conduire à des optimums locaux de mauvaise qualité, cet algorithme répartit les centres initiaux de manière espacée. Le premier centroïde est choisi au hasard, puis chaque centroïde suivant est sélectionné avec une probabilité proportionnelle au carré de la distance qui le sépare du centroïde le plus proche déjà choisi ($D(x)^2$). Cela accélère considérablement la convergence.
    \item \textbf{Affectation :} Chaque contrat est assigné au centroïde le plus proche en termes de distance euclidienne pondérée.
    \item \textbf{Mise à jour :} Les nouveaux centroïdes sont recalculés comme le barycentre pondéré des contrats de leur cluster.
    \item \textbf{Convergence :} Répétition des étapes 2 et 3 jusqu'à stabilisation des centroïdes.
\end{enumerate}

\subsubsection{Constitution des Model Points finaux}
Une fois la convergence atteinte, chaque cluster $j$ est transformé en un Model Point unique.
\begin{itemize}
    \item Les variables extensives (PM, Nombre de contrats) sont sommées : $PM_{MP_j} = \sum_{i \in S_j} PM_i$.
    \item Les variables intensives (Âge, Ancienneté, TMG) sont définies par les coordonnées du centroïde final, qui correspondent naturellement à la moyenne pondérée des caractéristiques des contrats du cluster.
\end{itemize}

Cette méthode permet de définir automatiquement des Model Points situés au cœur des masses financières du portefeuille, comme illustré dans la Figure \ref{fig:kmeans_clusters}, offrant ainsi une représentation optimale de la distribution des risques.

\begin{figure}[H]
    \centering
    \includegraphics[width=1\textwidth]{images/2_chapitres/chapitre4/kmeans_explication.png}
    \caption{Visualisation de la méthode K-Means sur un jeu de données fictif \cite{kmeans}}
    \label{fig:kmeans_clusters}
\end{figure}

Bien que l'algorithme des K-Means s'avère performant pour minimiser la variance financière, il repose sur une hypothèse géométrique forte : il tend à construire des partitions sphériques (ou convexes) et de tailles comparables. Or, la réalité d'un portefeuille d'assurance vie se caractérise souvent par des densités très hétérogènes et des formes irrégulières, liées par exemple à la commercialisation massive d'une génération spécifique de produits. Pour capter ces niches sans forcer l'homogénéité des volumes, il convient de s'affranchir de la notion de distance pure au profit d'une analyse de la densité locale.

\subsection{Partitionnement par densité : DBSCAN (introduction du Dithering et gestion du bruit par k-NN)}

Si l'algorithme des K-Means est performant pour des données réparties de manière homogène et sphérique, il montre ses limites lorsque les structures sous-jacentes du portefeuille sont complexes ou de densités variables. En effet, les données d'un portefeuille d'assurance vie présentent souvent des formes allongées ou irrégulières. Par exemple, une génération de produits vendue massivement sur une courte période crée une concentration spécifique de forme non convexe.

Pour capter ces formes complexes sans fixer \textit{a priori} le nombre de Model Points comme dans les K-Means, des méthodes basées sur la densité peuvent être mises en place : DBSCAN (\textit{Density-Based Spatial Clustering of Applications with Noise} ou \textit{Regroupement spatial basé sur la densité d'applications avec bruit}) et son extension hiérarchique HDBSCAN.

\subsubsection{Principes fondamentaux de DBSCAN}
L'algorithme DBSCAN définit un cluster comme une zone de forte densité séparée par des zones de faible densité. Il repose sur deux paramètres clés : un rayon de voisinage $\varepsilon$ (\textit{epsilon}) et un nombre minimum de points $MinPts$.

Mathématiquement, la notion de densité est formalisée par le voisinage $\varepsilon$ d'un point $x$, noté $N_\varepsilon(x)$ :
$ N_\varepsilon(x) = \{ y \in D \mid d(x,y) \le \varepsilon \} $

Un point $x$ est qualifié de \textbf{point cœur} (\textit{core point}) si son voisinage contient au moins $MinPts$ points : $|N_\varepsilon(x)| \ge MinPts$.

À partir de cette définition, les clusters sont construits par propagation de la propriété de \textit{densité-accessibilité} :
\begin{itemize}
    \item Un point $p$ est directement densité-accessible depuis $q$ si $q$ est un point cœur et $p \in N_\varepsilon(q)$.
    \item Un cluster est alors l'ensemble maximal de points connectés par cette relation de densité.
    \item Tout point n'appartenant à aucun cluster est considéré comme du \textbf{bruit} (\textit{outlier}).
\end{itemize}

\begin{figure}[H]
    \centering
    \includegraphics[width=1\textwidth]{images/2_chapitres/chapitre4/dbscan.png}
    \caption{Explication de l'algorithme DBSCAN (TODO : A refaire)}
    \label{fig:dbscan}
\end{figure}

\subsubsection{Adaptation aux contraintes assurantielles}
L'application directe de DBSCAN aux données brutes du portefeuille d'assurance vie utilisé se heurte à deux obstacles majeurs nécessitant des adaptations spécifiques implémentées dans le protocole : Le problème de la discrétisation et la gestion du bruit.

Les variables de gestion (Âge, Ancienneté) sont traditionnellement stockées de manière discrète. Cette structure crée une grille artificielle où plusieurs contrats se superposent exactement sur les mêmes coordonnées $(x,y)$. Cela fausse le calcul de densité locale : un point isolé sur la grille peut artificiellement paraître très dense simplement parce qu'il superpose plusieurs contrats identiques au mêmes caractéristiques.

Pour y remédier, une technique de \textbf{Dithering} (bruitage uniforme) peut être utilisée. Avant l'étape de clustering, une perturbation aléatoire $u_i$ est ajoutée aux variables temporelles de chaque contrat $x_i$ :

$ \tilde{x}_{i, d} = x_{i, d} + u_{i, d} \quad \text{avec} \quad u_{i, d} \sim \mathcal{U}([-0.5, 0.5]) $

Cette transformation permet alors de fluidifier l'espace et d'éviter une trop grande densité des contrats, sans altérer les propriétés statistiques globales du portefeuille (l'espérance de la perturbation étant nulle).

\subsubsection{Gestion du bruit et réallocation des contrats}
Contrairement à une analyse de données classique exploratoire où le bruit peut être écarté, dans un modèle épargne, la complétude des engagements est une contrainte absolue :     

$\sum PM_{MP} = \sum PM_{contrats}$.

Les contrats classés comme \og bruit \fg{} par DBSCAN (zones de faible densité ou points isolés) ne peuvent être ignorés.

Il a donc fallu implémenter une étape de post-traitement systématique : une réallocation via un algorithme des \textbf{$k$-plus proches voisins ($k$-NN)} avec $k=1$. Chaque contrat identifié comme bruit $x_{noise}$ est réaffecté au cluster validé $C_j$ le plus proche :
$ Class(x_{noise}) = Class(\underset{y \in \text{Clustered}}{\text{argmin}} \ d(x_{noise}, y)) $
Cela garantit qu'aucun contrat n'est perdu tout en rattachant les profils atypiques aux segments les plus ressemblants.

\begin{figure}[H]
    \centering
    \includegraphics[width=1\textwidth]{images/2_chapitres/chapitre4/dbscan.png}
    \caption{Explication du reclassement du bruit (TODO : A refaire)}
    \label{fig:dbscan_bruit}
\end{figure}


\subsection{L'extension hiérarchique adaptative : HDBSCAN (Distance d'accessibilité, extraction EOM vs Leaf)}

Bien que l'algorithme DBSCAN permette de créer des clusters sur des données non sphériques comme le fait l'algorithme K-Means, il conserve une limite majeure : l'utilisation d'un seuil de densité global ($\varepsilon$) unique. Dans un portefeuille d'assurance vie, la densité des données est hétérogène. Certaines zones, correspondant aux produits récemment commercialisés, présentent une très forte concentration de contrats, tandis que d'autres, regroupant des générations anciennes ou des produits de niche, sont beaucoup plus diffuses. L'application d'un $\varepsilon$ unique conduit inévitablement à un compromis insatisfaisant : un seuil strict fragmente les zones diffuses en bruit, tandis qu'un seuil lâche fusionne des clusters distincts dans les zones denses.

Pour répondre à cette problématique, l'algorithme \textbf{HDBSCAN} (\textit{Hierarchical Density-Based Spatial Clustering of Applications with Noise}) propose une approche hiérarchique permettant de détecter des clusters de densités variables. Son fonctionnement se décompose en cinq étapes clés \cite{hdbscan} :

\subsubsection{Transformation de l'espace : la distance d'accessibilité mutuelle}
Afin de rendre l'algorithme plus robuste aux points aberrants (bruit), HDBSCAN ne travaille pas directement sur la distance euclidienne brute, mais définit une nouvelle métrique : la « distance d'accessibilité mutuelle ».

\begin{equation}
 d_{mreach}(a, b) = \max \{ \text{core}_k(a), \text{core}_k(b), d(a, b) \} 
\end{equation}

Où $\text{core}_k(x)$ est la distance du point $x$ à son $k$-ième voisin le plus proche. Cette métrique permet de pondérer la distance entre deux points par leur densité locale : si un point se trouve dans une zone de faible densité, sa distance de cœur sera élevée, augmentant ainsi sa distance mutuelle $d_{mreach}$ avec les autres points. Cette transformation a pour effet d'isoler les points aberrants en les « repoussant » hors des zones de forte concentration, ce qui stabilise la formation des clusters.

\begin{figure}[H]
    \centering
    \includegraphics[width=0.7\textwidth]{images/2_chapitres/chapitre4/hdbscan1.png}
    \caption{Illustration de la distance d'accessibilité mutuelle \cite{hdbscan}}
    \label{fig:mutual_reachability}
\end{figure}

\subsubsection{Construction de l'Arbre Couvrant Minimum (MST)}
L'algorithme construit ensuite un graphe où chaque contrat est un sommet, relié aux autres par des arêtes pondérées par la distance $d_{mreach}$. Un Arbre Couvrant Minimum (Minimum Spanning Tree - MST) est généré pour connecter l'ensemble des points en minimisant le poids total des arêtes.

\begin{figure}[H]
    \centering
    \includegraphics[width=0.7\textwidth]{images/2_chapitres/chapitre4/hdbscan2.png}
    \caption{Construction de l'Arbre Couvrant Minimum \cite{hdbscan}}
    \label{fig:mst}
\end{figure}

\subsubsection{Construction de la hiérarchie des clusters}
En supprimant itérativement les arêtes du MST par ordre décroissant de poids, l'algorithme déconnecte progressivement le graphe. Cela crée une structure dendrogrammatique (arbre hiérarchique) représentant l'ensemble des regroupements possibles, du plus global (un seul cluster) au plus fin (chaque point est un cluster).

\begin{figure}[H]
    \centering
    \includegraphics[width=0.7\textwidth]{images/2_chapitres/chapitre4/hdbscan3.png}
    \caption{Construction de la hiérarchie des clusters \cite{hdbscan}}
    \label{fig:hierarchy_clusters}
\end{figure}

\subsubsection{Condensation de l'arbre}
L'arbre hiérarchique complet étant trop complexe, il est condensé. À chaque séparation (split), la taille minimale des nouveaux groupes formés est vérifiée si les nouveaux groupes formés atteignent une taille minimale ($MinPts$). Si ce n'est pas le cas, les points sont considérés comme du bruit détaché du cluster principal. Si les deux branches sont suffisamment grandes, il est considéré qu'il y a naissance de deux vrais clusters. Cette étape simplifie drastiquement l'arbre en ne conservant que les branches significatives.

\begin{figure}[H]
    \centering
    \includegraphics[width=0.7\textwidth]{images/2_chapitres/chapitre4/hdbscan4.png}
    \caption{Condensation de l'arbre hiérarchique \cite{hdbscan}}
    \label{fig:condensed_tree}
\end{figure}

\subsubsection{Extraction des clusters stables}
Contrairement aux méthodes hiérarchiques classiques qui coupent l'arbre à un niveau fixe, HDBSCAN sélectionne les clusters en maximisant une mesure de stabilité appelée \textit{Excess of Mass} (eom). La stabilité d'un cluster est définie par la somme, pour tous ses points, de la différence entre la densité à laquelle le point quitte le cluster ($\lambda_{d
eath}$) et celle où le cluster est apparu ($\lambda_{birth}$) :

\begin{equation}
 \mathcal{S}(C) = \sum_{x \in C} (\lambda_{death}(x) - \lambda_{birth}(C)) 
\end{equation}
L'algorithme remonte l'arbre condensé et sélectionne l'ensemble de clusters disjoints qui maximise cette stabilité globale.

\subsubsection{Stratégies d'extraction des clusters : EOM vs Leaf}
Deux stratégies de sélection des nœuds dans la hiérarchie ont été mises en concurrence :
\begin{itemize}
    \item \textbf{Excess of Mass (EOM) :} C'est la stratégie par défaut de HDBSCAN. Elle cherche à maximiser la stabilité globale en sélectionnant les clusters les plus persistants dans l'arbre. Elle a tendance à produire des macro-clusters, ce qui favorise un fort taux de compression mais peut lisser des micro-segments atypiques.
    \item \textbf{Leaf (Feuilles) :} Cette variante force l'algorithme à sélectionner uniquement les nœuds terminaux de l'arbre condensé (les « feuilles »). Elle garantit l'homogénéité maximale au sein de chaque Model Point en évitant toute fusion, au prix d'un nombre de MP plus important.
\end{itemize}

Cette méthodologie permet d'identifier simultanément des micro-clusters très compacts et des macro-clusters plus étendus, offrant une segmentation optimale et adaptative de la population des assurés sans paramétrage complexe d'un rayon de voisinage.

\begin{figure}[H]
    \centering
    \includegraphics[width=0.7\textwidth]{images/2_chapitres/chapitre4/hdbscan6.png}
    \caption{Résultat final du clustering HDBSCAN \cite{hdbscan}}
    \label{fig:hdbscan_final}
\end{figure}

\begin{figure}[H]
    \centering
    \includegraphics[width=0.8\textwidth]{images/2_chapitres/chapitre4/dbscan_hdbscan.jpg}
    \caption{Comparaison des approches DBSCAN et HDBSCAN \cite{dbscan_hdbscan}}
    \label{fig:hdbscan_comparison}
\end{figure}

\subsubsection{Constitution finale des Model Points}
Une fois la partition optimale obtenue (via DBSCAN ou HDBSCAN) et le bruit réalloué, la construction des Model Points suit la logique de conservation des flux financiers. Pour chaque cluster $j$, le Model Point est défini par le barycentre pondéré des contrats :

$ \mu_j = \frac{\sum_{i \in C_j} PM_i \cdot x_i}{\sum_{i \in C_j} PM_i} $

Cette méthode assure que le Model Point se situe au centre de gravité financier de son groupe, minimisant ainsi le biais d'agrégation sur les projections futures.

\section{L'approche par segmentation supervisée}

Alors que les méthodes d'apprentissage non supervisé (K-Means, HDBSCAN) cherchent à regrouper les contrats selon une similarité globale de leurs caractéristiques, elles ne garantissent pas structurellement l'homogénéité d'un indicateur financier précis. Pour pallier cette limite et forcer la modélisation à se concentrer sur les enjeux de rentabilité (comme le TMG), il convient de se tourner vers des algorithmes d'apprentissage supervisé.

\subsection{Principe des Arbres de Décision (CART) appliqués au passif}

L'approche par arbres de décision, et plus précisément l'algorithme CART (\textit{Classification and Regression Trees}), introduit une logique d'apprentissage supervisé dans le processus d'agrégation. Contrairement aux K-Means qui cherchent une similarité globale, le CART vise à partitionner le portefeuille en minimisant la variance d'une variable cible d'
intérêt.

Dans le cadre de cette étude, deux variantes de ciblage ont été implémentées :
\begin{itemize}
    \item \textbf{Refined :} La cible est une combinaison linéaire de l'encours et du taux garanti ($PM \times (1 + 10 \times TMG)$), afin de forcer l'arbre à isoler les masses fi
nancières à fort enjeux de revalorisation.
    \item \textbf{Spread Focus :} La cible est le TMG pur, l'objectif étant d'obtenir des clusters d'une grande pureté financière pour limiter la dilution des garanties.
\end{itemize}

L'algorithme procède par divisions successives (splits) binaires de l'espace des caractéristiques (Âge, Ancienneté, TMG). À chaque étape, il choisit la variable et le seuil qui maximisent l'homogénéité de la cible au sein des deux groupes formés. Le processus s'arrête lorsqu'un nombre maximal de feuilles (correspondant au nombre de Model Points souhaité) est atteint.

\subsection{Définition des variables cibles (Refined vs Spread Focus)}
La puissance de l'arbre CART réside dans le choix de la métrique à optimiser :
\begin{itemize}
    \item \textbf{Variante Refined :} Une variable cible d'intérêt $Y_i$ est définie par $Y_i = PM_i \times (1 + \alpha \cdot TMG_i)$. Cette pondération force l'arbre à accorder plus d'importance aux contrats qui pèsent lourd financièrement tout en ayant un coût de garantie élevé. C'est une approche visant l'équilibre entre masse et risque.
    \item \textbf{Variante Spread Focus :} La cible est le TMG pur. L'algorithme cherche alors à créer des groupes où l'hétérogénéité des taux garantis est minimale. Cette approche est privilégiée pour les modèles ALM très sensibles au spread entre taux de marché et taux techniques.
\end{itemize}

Cette méthode présente l'avantage d'identifier automatiquement les interactions non linéaires entre les variables (par exemple, un seuil d'âge qui ne devient critique que pour une certaine ancienneté). Cependant, elle produit une segmentation par « paliers » qui peut s'avérer trop brutale pour modéliser des phénomènes continus comme la mortalité.

\section{Les approches dynamiques basées sur l'écoulement des flux}

Toutes les méthodes de partitionnement spatial abordées jusqu'ici, qu'elles soient supervisées ou non, partagent une limite conceptuelle commune : elles évaluent la similarité des profils de risque de manière purement statique, à la date d'arrêté. Pour franchir un nouveau cap de fidélité financière, une rupture méthodologique s'impose : agréger les contrats non plus sur ce qu'ils sont, mais sur ce qu'ils vont produire.

\subsection{Appariement comportemental (Cash-Flow Matching) et niveaux de fidélité (Proxy vs Real OPS)}

La méthode de \textit{Cash-Flow Matching} constitue l'approche la plus avancée de ce protocole. Elle repose sur un changement de paradigme : les contrats sont regroupés selon leur \textbf{profil de risque dynamique}.

Le principe s'articule en deux phases :
\begin{enumerate}
    \item \textbf{Phase de Projection Proxy :} Chaque contrat individuel $i$ est projeté au sein du moteur métier sous un scénario déterministe central sur un horizon de $T=20$ ans. Un vecteur de flux financiers est extrait de cette projection $\mathbf{CF}_i = [f_{i,1}, f_{i,2}, ..., f_{i,T}]$, où $f_{i,t}$ représente la somme des prestations (rachats, décès, termes) et de la marge financière à l'année $t$.
    \item \textbf{Phase de Clustering :} L'algorithme des K-Means est ensuite appliqué non pas sur les données de gestion, mais sur ces vecteurs de flux $\mathbf{CF}_i$. La distan
ce entre deux contrats est alors définie par leur proximité comportementale dans le temps :
    
\begin{equation}
 d(i, j) = \sqrt{\sum_{t=1}^{T} w_t (f_{i,t} - f_{j,t})^2} 
\end{equation}
\end{enumerate}

\subsubsection{Niveaux de fidélité du moteur proxy}
Cette approche a été déclinée en deux versions selon la complexité du moteur de projection utilisé pour générer les vecteurs $\mathbf{CF}_i$ :
\begin{itemize}
    \item \textbf{Mode Proxy (Simple) :} Utilise un modèle simplifié, souvent linéaire ou déterministe de premier ordre, pour projeter les flux. C'est une méthode rapide mais qui peut négliger les effets de second ordre comme l'épuisement de la PPE ou les options de rachat dynamique.
    \item \textbf{Mode Real OPS (Moteur Métier) :} Utilise le véritable moteur de calcul Polars développé pour ce mémoire. Bien que plus coûteux en temps de préparation (env. 30s), il garantit que le clustering s'opère sur la réalité économique exacte du contrat, capturant l'intégralité des non-linéarités (participation aux bénéfices, rachats structurels, fiscalité).
\end{itemize}

Cette approche, désignée sous le terme \textbf{Real OPS} dans cette étude, permet de capturer intrinsèquement toutes les non-linéarités du modèle ALM. Deux contrats présentant des caractéristiques très différentes (par exemple, un assuré jeune sur un contrat ancien et un assuré âgé sur un contrat récent) peuvent être regroupés s'ils génèrent des flux de sortie statistiquement équivalents.

L'avantage majeur est la réduction drastique du « biais d'agrégation » puisque le critère de regroupement est précisément l'indicateur calculé \textit{in fine} (le Best Estimate). L'inconvénient réside dans le coût computationnel de la pré-projection initiale, qui doit être compensé par le gain de temps lors des lancements ALM ultérieurs.

\subsection{Calibration de lois a posteriori (Méthode "Biais Zéro" et clés financières)}

L'approche par calibration \textit{a posteriori}, dénommée \og Biais Zéro \fg{} dans cette étude, marque une rupture méthodologique avec les techniques de clustering spatial. Là où les méthodes classiques tentent d'ajuster des contrats à des lois de mortalité existantes, cette approche propose d'adapter les lois de sortie pour qu'elles correspondent parfaitement à un regroupement arbitraire de contrats.

\subsubsection{Mécanique des décréments et bases de calibration}

Pour atteindre une erreur nulle sur le Best Estimate, la calibration doit respecter scrupuleusement l'ordonnancement des flux au sein du moteur de projection Polars. Le moteur applique la séquence suivante sur une période $[t, t+1]$ :
\begin{enumerate}
    \item \textbf{Revalorisation technique} : Application du TMG au prorata temporis sur la PM de début d'année.
    \item \textbf{Calcul des rachats} : Les rachats s'appliquent sur la PM revalorisée à mi-période.
    \item \textbf{Calcul des décès} : Les décès s'appliquent sur la PM nette des rachats.
\end{enumerate}

Ainsi, les lois induites $q_{k,t}$ ne sont pas calculées sur la PM de début d'année, mais sur les \textbf{bases d'application exactes} (mi-période). Pour un groupe de contrats $k$, les probabilités de rachat ($q_{k,t}^{rt}$) et de décès ($q_{k,t}^{dc}$) sont définies par :

\begin{equation}
    q_{k,t}^{rt} = \frac{\sum_{i \in C_k} PR_{i,t}}{\sum_{i \in C_k} BR_{i,t}} \quad ; \quad q_{k,t}^{dc} = \frac{\sum_{i \in C_k} PD_{i,t}}{\sum_{i \in C_k} BD_{i,t}}
\end{equation}

Où $PR_{i,t}$ et $PD_{i,t}$ représentent respectivement les prestations pour rachats et pour décès du contrat $i$ à l'instant $t$. Les termes $BR_{i,t}$ et $BD_{i,t}$ désignent les bases d'application exactes associées.

Où la base de rachat intègre les intérêts techniques acquis à mi-période, et la base de décès est nette des rachats déjà effectués. Cette précision mathématique permet de capturer l'intégralité de la cinétique des flux, rendant le Model Point agnostique de la démographie réelle ($Age, Sexe$).

\subsubsection{Définition des clés d'agrégation financières}

Le portefeuille est partitionné selon un triplet de variables financières qui régissent la mécanique de revalorisation :
\begin{itemize}
    \item \textbf{Le TMG net} : Le taux minimum garanti servi à l'assuré.
    \item \textbf{Le TFGSE} : Le taux de frais de gestion sur encours prélevé par l'assureur.
    \item \textbf{Le TAF} : Le taux d'affectation des produits financiers.
\end{itemize}

Cette segmentation garantit que le \textbf{Model Point Financier (MPF)} possède un seuil de rentabilité brut identique à celui des contrats individuels :
\begin{equation}
    TMG_{brut} = \frac{TMG_{net} + TFGSE}{TAF}
\end{equation}

L'agrégation sur ces clés, combinée aux lois induites, permet d'atteindre le \og Biais Zéro \fg{} opérationnel, l'erreur résiduelle ne provenant plus que des arrondis numériques (estimée à $0,006\%$ dans nos tests).

\section{Conclusion}

Ce chapitre a permis d'exposer le socle théorique des différentes méthodes d'agrégation envisagées, de la rigidité des approches déterministes classiques à la flexibilité des algorithmes d'apprentissage automatique spatial, jusqu'à la complexité des modèles d'appariement de flux financiers. Chaque méthode présente des compromis distincts entre la précision de la modélisation, le respect des garanties financières et le coût computationnel. 

Cependant, l'évaluation purement théorique de ces approches demeure insuffisante pour statuer sur leur applicabilité. Il est désormais impératif de confronter ces méthodes à la réalité d'un moteur de projection stochastique. Le chapitre suivant s'attachera ainsi à définir un protocole de test standardisé et à mener une analyse comparative rigoureuse, afin d'identifier la méthodologie optimale pour l'évaluation des passifs sous le référentiel Solvabilité 2.
\chapter{Protocole d'Analyse et Résultats de l'Agrégation}

Ce chapitre constitue le cœur expérimental de cette étude. Après avoir exposé les fondements théoriques des méthodes d'agrégation, il s'agit désormais de mettre en œuvre un cadre rigoureux pour comparer leurs performances. L'objectif est double : quantifier précisément le risque de modèle induit par la réduction de dimension sur le passif seul, puis valider la robustesse de la méthode sélectionnée à travers des analyses de sensibilité.

\section{Définition du Protocole de Test Comparatif}

La sélection de la méthode d'agrégation optimale ne peut reposer sur une simple intuition statistique. Elle nécessite un cadre expérimental capable de simuler les conditions réelles d'une clôture prudentielle, en mettant en concurrence la simplicité des méthodes déterministes et la puissance des approches par apprentissage.

\subsection{Constitution des portefeuilles de test}

L'étude s'appuie sur un portefeuille de référence, désigné par la suite comme le « Portefeuille Full », composé de \textbf{50 000 contrats individuels}. Ce volume a été choisi pour représenter une taille critique permettant d'observer les phénomènes de compensation statistique tout en restant techniquement projetable en un temps raisonnable pour établir une base de comparaison exacte.

Chaque contrat est défini par un vecteur de caractéristiques multidimensionnel $\mathbf{x}_i \in \mathbb{R}^{12}$, comprenant notamment :
\begin{itemize}
    \item \textbf{Variables de risques biométriques :} Âge de l'assuré (de 18 à 95 ans) et Sexe.
    \item \textbf{Variables de structure fiscale :} Ancienneté du contrat (cruciale pour les lois de rachat et la fiscalité en cas de décès).
    \item \textbf{Variables financières :} Provision Mathématique (PM), Taux Minimum Garanti (TMG) variant de 0\% à 4,5\%, et taux de chargement.
\end{itemize}

Le processus de mise en œuvre suit une architecture en trois étapes distinctes :
\begin{enumerate}
    \item \textbf{Phase de pré-traitement :} Standardisation (z-score) des variables continues et application du \textit{dithering} pour les méthodes de densité afin de fluidifier l'espace des données discrétisées.
    \item \textbf{Génération des Model Points :} Application de chaque algorithme cible pour obtenir des portefeuilles compressés. Pour chaque méthode, nous avons fait varier les hyperparamètres afin de générer une famille de portefeuilles allant de 25 à 5 000 lignes.
    \item \textbf{Projection ALM Massive :} Chaque portefeuille compressé est injecté dans le moteur de projection complet (projection sur 50 ans, scénario central Best Estimate) pour mesurer l'impact réel sur les flux.
\end{enumerate}

\begin{figure}[H]
    \centering
    \fbox{\begin{minipage}{0.8\textwidth}
        \centering
        \vspace{2cm}
        \textbf{[GRAPHIQUE : Workflow du Protocole de Test]} \\
        \textit{Schéma montrant le flux : Portefeuille Initial $\rightarrow$ Clustering $\rightarrow$ Vecteurs de flux $\rightarrow$ Moteur ALM $\rightarrow$ Comparaison BE}
        \vspace{2cm}
    \end{minipage}}
    \caption{Schéma de mise en œuvre du protocole de test comparatif}
    \label{fig:workflow_test}
\end{figure}

\subsection{Définition des critères de sélection}

Le choix final de la méthode repose sur un arbitrage multicritère, principalement axé sur la recherche d'un point optimal sur la « Courbe de Pareto » entre précision et taux de compression.

\begin{itemize}
    \item \textbf{La fidélité sur le Best Estimate (BE) :} Mesurée par l'erreur relative $\Delta_{BE} \%$. Une erreur supérieure à $0,05\%$ est considérée comme significative.
    $$ \Delta_{BE} \% = \frac{BE_{agr\acute{e}g\acute{e}} - BE_{r\acute{e}f\acute{e}rence}}{BE_{r\acute{e}f\acute{e}rence}} $$
    \item \textbf{La préservation de la "Pureté Financière" :} Capacité de la méthode à ne pas diluer les taux garantis (TMG) au sein des groupes.
    \item \textbf{Le taux de compression :} On cherche à descendre en dessous de 1 000 Model Points pour garantir la fluidité des calculs stochastiques futurs.
    \item \textbf{Le temps de constitution (Overhead) :} Le coût de préparation des données doit rester marginal par rapport au gain de temps de projection.
\end{itemize}

\section{Analyse Comparative et Choix de la Méthode Optimale}

\subsection{Synthèse des performances sur le passif seul}

L'analyse massive réalisée (plus de 150 simulations individuelles) a permis de dresser une matrice de performance exhaustive permettant d'isoler le comportement de chaque algorithme face à la compression.

\begin{table}[H]
\centering
\small
\begin{tabular}{llcccc}
\toprule
\textbf{Méthode} & \textbf{Variante / Paramètres} & \textbf{Nb MP} & \textbf{Erreur BE \%} & \textbf{Écart (M€)} & \textbf{Robustesse} \\ 
\midrule
\textit{Référence} & \textit{Portefeuille 50k lignes} & \textit{50 000} & \textit{0,000 \%} & \textit{0,0} & \textit{Absolue} \\
\midrule
\rowcolor{blue!10} \textbf{Cash-Flow} & \textbf{Real OPS (n=2000)} & \textbf{2 060} & \textbf{< 0,00001 \%} & \textbf{0,001} & \textbf{Maximale} \\
\rowcolor{blue!5} \textbf{Cash-Flow} & \textbf{Real OPS (n=500)} & \textbf{679} & \textbf{0,00017 \%} & \textbf{0,022} & \textbf{Excellente} \\
\midrule
\textbf{HDBSCAN} & Selection: Leaf (Fin) & 1 632 & 0,00068 \% & 0,086 & Très sensible \\
\textbf{Banding} & Smart Granular (Âge/Anc) & 708 & 0,00068 \% & 0,086 & Très stable \\
\textbf{K-Means} & Pondéré PM (n=2500) & 2 499 & 0,00116 \% & 0,146 & Constante \\
\textbf{K-Means} & Pondéré PM (n=500) & 499 & 0,00143 \% & 0,180 & Bonne \\
\midrule
\textbf{CART} & Refined (Cible PM/TMG) & 499 & -0,14940 \% & -18,76 & Biaisée \\
\textbf{HDBSCAN} & Selection: EOM (Densité) & 86 & 0,11671 \% & 14,65 & Instable \\
\textbf{CART} & Spread Focus (Compression) & 26 & 0,13197 \% & 16,57 & Trop grossière \\
\bottomrule
\end{tabular}
\caption{Matrice comparative des performances d'agrégation sur le Best Estimate}
\label{tab:giga_matrice_resultats}
\end{table}

\begin{figure}[H]
    \centering
    \fbox{\begin{minipage}{0.8\textwidth}
        \centering
        \vspace{2cm}
        \textbf{[GRAPHIQUE : Courbe de Pareto Précision/Compression]} \\
        \textit{Axes : X = Nb Model Points (log), Y = Erreur BE \% (log). \\ Courbes pour : Banding, K-Means, Cash-Flow.}
        \vspace{2cm}
    \end{minipage}}
    \caption{Arbitrage Précision vs Nombre de Model Points par méthode}
    \label{fig:pareto_agreg}
\end{figure}

Plusieurs enseignements majeurs découlent de ces résultats :
\begin{enumerate}
    \item \textbf{La supériorité du Cash-Flow Matching :} Cette méthode surpasse l'ensemble des approches classiques. En regroupant les contrats sur leur comportement projeté plutôt que sur leurs caractéristiques statiques, elle capture toutes les non-linéarités métier.
    \item \textbf{L'efficacité du Banding « Smart » :} Le Banding avec une discrétisation fine reste extrêmement robuste (erreur < 0,001\%), constituant un excellent standard de marché.
    \item \textbf{L'instabilité des méthodes de densité :} HDBSCAN est extrêmement sensible au paramétrage, ce qui rend son usage industriel délicat.
    \item \textbf{Le biais des Arbres de Décision (CART) :} Bien qu'offrant de forts taux de compression, le CART présente un biais systématique trop élevé pour une étude de précision.
\end{enumerate}

\subsection{Justification du choix de la méthode retenue}

Au terme de cette analyse, la méthode \textbf{Cash-Flow Clustering (Real OPS)} est retenue pour la suite de ce mémoire. Elle offre le meilleur compromis : une erreur relative négligeable ($10^{-7}$) et une division par 75 de la taille du portefeuille. Cela garantit que les variations de BE mesurées dans les sections suivantes seront exclusivement dues aux chocs appliqués, et non à un bruit d'agrégation.

\section{Analyse de Sensibilité des Indicateurs S2}

Le but de cette section est de tester la robustesse de la méthode d'agrégation sélectionnée face à des variations de l'environnement ou du portefeuille.

\subsection{Définition des Scénarios de Sensibilité}
    \subsubsection{Création des portefeuilles de test via le générateur}
    % Votre texte ici...
    \subsubsection{Description des chocs sur les variables clés (âge, montant de la PM, etc.)}
    % Votre texte ici...
    \subsubsection{Scénario d'intégration d'un nouveau produit dans le portefeuille}
    % Votre texte ici...

\subsection{Analyse de l'Impact de l'Agrégation sur la Mesure des Chocs}
    \subsubsection{Comparaison des indicateurs S2 sur portefeuilles choqués granulaires et agrégés}
    % Votre texte ici...
    \subsubsection{Analyse de la fidélité de la méthode d'agrégation à retranscrire la sensibilité}
    % Votre texte ici...

\section{Interprétation des Résultats et Validation de l'Approche}
    \subsection{Validation de la performance de la chaîne de modélisation}
    % Votre texte ici...
    \subsection{Enseignements sur la sensibilité des portefeuilles aux modifications du passif}
    % Votre texte ici...

\chapter{Conclusion}
\section{Résumé des résultats}
\subsection{Synthèse des principaux résultats obtenus}
\subsection{Impact des méthodes d'agrégation et des contraintes réglementaires sur les portefeuilles de passifs}

\section{Perspectives d'amélioration}
\subsection{Axes d'amélioration pour les générateurs de portefeuilles de passifs}
\subsection{Évolutions possibles des méthodes d'agrégation et de modélisation ALM}  
\subsection{Autres domaines d'application des générateurs de portefeuilles de passifs}
\section{Conclusion générale}

\chapter*{Annexes}
\addcontentsline{toc}{chapter}{Annexes}

\section*{Chapitre 2}


\begin{figure}[H]
\centering
\includegraphics[width=0.9\textwidth]{images/2_chapitres/chapitre3/distribution_age_assures.png}
\caption{Distribution empirique de l'âge des assurés (Hommes et Femmes).}
\label{fig:empirique}
\end{figure}

\begin{figure}[H]
\centering
\includegraphics[width=0.9\textwidth]{images/2_chapitres/chapitre3/estimation_loi_gamma.png}
\caption{Ajustement de la loi Gamma sur la distribution empirique.}
\label{fig:gamma}
\end{figure}

\begin{figure}[H]
\centering
\includegraphics[width=0.9\textwidth]{images/2_chapitres/chapitre3/pyramide_age.png}
\caption{Pyramide des âges de la population française en 2024 \cite{pyramide_age}.}
\label{fig:pyramide_age}
\end{figure}


\begin{table}[H]
  \centering
  \caption{Historique du taux technique maximal et du TMG utilisé dans le modèle (1993--2024)}
  \label{tab:TMG_historique}
  \begin{tabular}{rcc}
    \hline
    Année & $TT^{\max}_t$ & $TMG_t$ \\
          & (taux technique max) & (taux min. garanti retenu) \\
    \hline
    1993 & 0{,}045  & 0{,}045 \\
    1994 & 0{,}045  & 0{,}045 \\
    1995 & 0{,}035  & 0{,}035 \\
    1996 & 0{,}035  & 0{,}035 \\
    1997 & 0{,}0339 & 0{,}0325 \\
    1998 & 0{,}0272 & 0{,}025  \\
    1999 & 0{,}0265 & 0{,}025  \\
    2000 & 0{,}0329 & 0{,}0325 \\
    2001 & 0{,}0301 & 0{,}030  \\
    2002 & 0{,}0292 & 0{,}0275 \\
    2003 & 0{,}0247 & 0{,}0225 \\
    2004 & 0{,}0243 & 0{,}0225 \\
    2005 & 0{,}0207 & 0{,}020  \\
    2006 & 0{,}0229 & 0{,}0225 \\
    2007 & 0{,}0260 & 0{,}025  \\
    2008 & 0{,}0261 & 0{,}025  \\
    2009 & 0{,}0219 & 0{,}020  \\
    2010 & 0{,}0189 & 0{,}0175 \\
    2011 & 0{,}0213 & 0{,}020  \\
    2012 & 0{,}0153 & 0{,}015  \\
    2013 & 0{,}0141 & 0{,}0125 \\
    2014 & 0{,}0099 & 0{,}0075 \\
    2015 & 0{,}0051 & 0{,}0000 \\
    2016 & 0{,}0027 & 0{,}0000 \\
    2017 & 0{,}0045 & 0{,}0000 \\
    2018 & 0{,}0048 & 0{,}0000 \\
    2019 & 0{,}0009 & 0{,}0000 \\
    2020 & 0{,}0000 & 0{,}0000 \\
    2021 & 0{,}0003 & 0{,}0000 \\
    2022 & 0{,}0090 & 0{,}0000 \\
    2023 & 0{,}0186 & 0{,}0000 \\
    2024 & 0{,}0180 & 0{,}0000 \\
    \hline
  \end{tabular}
\end{table}
\newpage % Pour commencer la bibliographie sur une nouvelle page

% Ajoute une entrée "Bibliographie" dans la table des matières
\addcontentsline{toc}{chapter}{Bibliographie} 

\begin{thebibliography}{9}

% \bibitem{auteur_livre}
% NOM Prénom. \textit{Titre du livre en italique}. Lieu d'édition : Nom de l'éditeur, Année.

% \bibitem{auteur_article}
% NOM Prénom. « Titre de l'article entre guillemets ». \textit{Titre de la revue en italique}, volume X, numéro Y, saison Année, p. 123-456.
\bibitem{france_assureurs}
FRANCE ASSUREURS. \textit{L'assurance vie en 2024}. (23 septembre 2025). Consulté le 28 Octobre 2025, sur \url{https://www.franceassureurs.fr/nos-chiffres-cles/assurance-vie/lassurance-vie-en-2024/}


\bibitem{clustering_book}
DORNAIKA Fadi, HAMAD Denis, CONSTANTIN Joseph, TRONG HOANG Vinh. \textit{Advances in Data Clustering}. Lieu d'édition : Springer, 2024.

\bibitem{goffard_guerrault}
GOFFARD Pierre-Olivier, GUERRAULT Xavier. « Is it optimal to group policyholders by age, gender, and seniority for BEL computations based on model points? ». \textit{European Actuariel Journal}, volume 5, 17 Avril 2015, p. 165-180.


\bibitem{memoire_ben_fadhel}
BEN FADHEL Amine. « Accéleration de l'évaluation de la solvabilité prospective d'un assureur épargne ». \textit{Mémoire pour l'Institut des Actuaires }, 2022.

\bibitem{insee_prop_av_age}
INSEE. \textit{Taux de détention des produits de patrimoine selon l'âge de la personne de référence du ménage en 2021}. (2021). Consulté le 18 Septembre 2025, sur \url{https://www.insee.fr/fr/outil-interactif/5367857/details/30_RPC/34_PAT/34A_Figure1}

\bibitem{pyramide_age}
INSEE. \textit{Pyramide des âges de la population française en 2024}. (2024). Consulté le 1er Septembre 2025, sur \url{https://www.insee.fr/fr/outil-interactif/5014911/pyramide.htm}

\bibitem{insee_patrimoine_age}
INSEE. \textit{Enquête Histoire de vie et Patrimoine 2020-2021}. 2021. Consulté le 24 Janvier 2026, sur \url{https://www.insee.fr/fr/statistiques/7941419?sommaire=7941491#tableau-figure1}

\bibitem{repartition_hommes_femmes}
IFOP. \textit{Enquête sur les 60 ans de l'indépendance financière des femmes : chiffres clés sur la répartition hommes-femmes en assurance vie}. (16 Juillet 2025). Consulté le 16 Novembre 2025, sur \url{https://www.ifop.com/article/enquete-sur-les-60-ans-de-lindependance-financiere-des-femmes/}

\bibitem{repartition_hommes_femmes_euro_uc}
IFOP. \textit{Baromètre - les femmes et l'argent : chiffres clés sur la répartition hommes-femmes sur les fonds EURO et UC en assurance vie}. (Janvier 2024). Consulté le 16 Novembre 2025, sur \url{https://www.ifop.com/wp-content/uploads/2024/03/barometre-vives-2024ok-1.pdf}

\bibitem{kmeans}
geeksforgeeks. \textit{K-Means vs K-Means++ Clustering Algorithm}. Consulté le 1 Février 2026, sur \url{https://www.geeksforgeeks.org/machine-learning/k-means-vs-k-means-clustering-algorithm/}

% \bibitem{rapport_technique}
% NOM Prénom et NOM Prénom. \textit{Titre du rapport}. Type de rapport (e.g. Rapport de recherche), Organisme/Université, Année.
\bibitem{dbscan_hdbscan}
Daily Dose of Data Science. \textit{HDBSCAN vs. DBSCAN}. Consulté le 1 Février 2026, sur \url{https://blog.dailydoseofds.com/p/hdbscan-vs-dbscan}

\bibitem{hdbscan}
HDBSCAN Developers. \textit{How HDBSCAN Works}. Consulté le 3 Février 2026, sur \url{https://hdbscan.readthedocs.io/en/latest/how_hdbscan_works.html}


\bibitem{acpr66}
ACPR. \textit{Le taux technique en assurance vie (Code des assurances)}. Consulté le 14 Février 2026, sur \url{https://acpr.banque-france.fr/sites/default/files/medias/documents/201606_as66_le_taux_technique_en_assurance_vie.pdf}

\bibitem{acprTauxBas}
ACPR. \textit{Assurance vie en France et environnement de taux bas}. Consulté le 14 Février 2026, sur \url{https://acpr.banque-france.fr/sites/default/files/medias/documents/201705-as78-taux-bas-version-3_0.pdf}

\bibitem{acprRevalo2024}
ACPR. \textit{Revalorisation 2024 des contrats d’assurance-vie et de capitalisation}. Consulté le 14 Février 2026, sur \url{https://acpr.banque-france.fr/fr/publications-et-statistiques/publications/ndeg-175-revalorisation-2024-des-contrats-dassurance-vie-et-de-capitalisation}

\bibitem{acprMarche2024}
ACPR. \textit{Le marché de l’assurance-vie en 2024}. Consulté le 14 Février 2026, sur \url{https://acpr.banque-france.fr/fr/publications-et-statistiques/publications/ndeg-170-le-marche-de-lassurance-vie-en-2024}

\bibitem{gvmTMG}
Good Value for Money. \textit{Taux Minimum Garanti (TMG)}. Consulté le 14 Février 2026, sur \url{https://www.goodvalueformoney.eu/documentation/taux-minimum-garanti-tag}

\bibitem{faUC2024}
France Assureurs. \textit{L’assurance vie en unités de compte en 2024}. Consulté le 14 Février 2026, sur \url{https://www.franceassureurs.fr/nos-chiffres-cles/assurance-vie/lassurance-vie-en-unites-de-compte-en-2024/}

\end{thebibliography}

\end{document}
