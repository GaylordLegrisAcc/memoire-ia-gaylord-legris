\chapter*{Introduction}
% Ajouter manuellement cette section à la table des matières
\addcontentsline{toc}{chapter}{Introduction}

Placement privilégié des épargnants français, l'assurance vie a atteint un encours record de 1 908 milliards d'euros à fin 2023 (France Assureurs) \cite{site_web}, confirmant ainsi son rôle prépondérant dans le patrimoine financier national. Toutefois, ce secteur fait face à une rupture structurelle marquée par la fin du cycle de taux bas et la remontée brutale des taux d'intérêt observée depuis 2022. Le taux de revalorisation moyen des fonds en euros a ainsi atteint 2,6 \% pour l'année 2023 (estimation ACPR). Cette mutation rend d'autres produits d'épargne plus attractifs et exerce une pression concurrentielle inédite sur les contrats d'assurance vie, notamment sur les fonds en euros qui deviennent de moins en moins attractifs. Pour les assureurs, le défi est de taille : leurs portefeuilles d'actifs, majoritairement constitués d'obligations acquises durant la longue période de taux bas, présentent une forte inertie. Cet héritage obligataire freine leur capacité à servir des rendements compétitifs et place la gestion actif-passif (ALM) au cœur des enjeux stratégiques.

Pour piloter leur bilan, les assureurs s'appuient sur des modèles ALM sophistiqués, essentiels pour simuler l'impact de différentes stratégies dans le cadre réglementaire de Solvabilité II. Cependant, la complexité de ces modèles et la nécessité de réaliser un grand nombre de simulations se heurtent à une contrainte opérationnelle majeure : le temps de calcul. Cette contrainte limite la capacité des assureurs à explorer en profondeur l'ensemble des risques et des opportunités. Face à cette réalité, une question centrale émerge : dans quelle mesure peut-on réduire le temps de calcul des modèles tout en préservant la fidélité des indicateurs de risque ? Ce mémoire se propose d'investiguer cette problématique en étudiant l'impact d'une méthode de réduction de lignes dans des portefeuilles de passifs. L'enjeu est de déterminer si une représentation plus grossière du passif peut suffire pour représenter correctement le portefeuille et gérer le pilotage stratégique et sous quelles conditions une telle simplification est valide, sans masquer des dynamiques de risque essentielles.

Pour répondre à cette problématique, ce mémoire adoptera une double approche. Premièrement, il s'agira de développer un générateur de portefeuilles de passif puis le reste de l'analyse portera sur les effets de l'agrégation sur un portefeuille représentatif du marché français. L'objectif est de comprendre comment les risques évoluent à travers une agrégation. Ce mémoire ne se contentera pas d'analyser l'impact d'une seule méthode d'agrégation ; au contraire, plusieurs méthodes et approches seront testées. Le critère de sélection de la méthode la plus pertinente reposera sur un triple objectif : minimiser l'écart des indicateurs clés (notamment le Best Estimate et le SCR), optimiser la rapidité des calculs et atteindre le plus haut niveau d'agrégation possible qui garderait une significaitivité économique pour l'assureur suffisante.

L'axe principal de ce mémoire consistera donc à mener une analyse de sensibilités approfondie sur ces portefeuilles, qu'ils soient granulaires ou agrégés. Notre étude s'appuiera sur des indicateurs quantitatifs clés issus de la norme Solvabilité II, en évaluant notamment l'impact des chocs économiques sur le Best Estimate, le Solvency Capital Requirement (SCR) et la Present Value of Future Profits (PVFP). Ces métriques permettront de mesurer rigoureusement comment l'agrégation modifie la perception du risque et la valeur économique du portefeuille.

Ce mémoire s'articulera en quatre parties distinctes, chacune conçue pour apporter une réponse progressive et rigoureuse à notre problématique.


La \textbf{première partie} sera consacrée au cadre conceptuel de notre étude. Nous y détaillerons le contexte réglementaire de Solvabilité II, qui définit les exigences de capital et les métriques de risque, ainsi que les principes fondamentaux de la modélisation actif-passif (ALM). L'avantage de cette section est de fournir au lecteur les clés de compréhension essentielles pour appréhender les enjeux techniques et stratégiques du pilotage d'un bilan assurantiel.


La \textbf{deuxième partie} adoptera une approche pratique en se concentrant sur la mise en œuvre de notre environnement de simulation. Nous y décrirons la méthodologie de création des portefeuilles de passifs synthétiques, représentatifs de différentes générations de contrats, ainsi que les outils développés pour leur projection. Cette étape est cruciale car elle garantit la robustesse et la pertinence des analyses qui suivront, en créant un laboratoire d'expérimentation fiable.


La \textbf{troisième partie} constituera le cœur méthodologique de ce mémoire. Elle explorera et comparera de manière systématique plusieurs techniques d'agrégation des engagements de passif. L'objectif sera d'identifier les approches les plus prometteuses, en évaluant leur capacité à simplifier la structure du portefeuille sans dénaturer ses caractéristiques fondamentales. Cette analyse comparative permettra de mettre en lumière les forces et faiblesses de chaque méthode.


Enfin, la \textbf{quatrième partie} présentera et analysera les résultats de nos simulations. À travers des tests de sensibilité approfondis sur les indicateurs clés (SCR, PVFP), nous quantifierons l'impact de chaque méthode d'agrégation sur la perception du risque et la valeur économique. Cette analyse empirique nous permettra de conclure sur la validité des approches testées et de formuler des recommandations concrètes sur les conditions d'utilisation d'un passif agrégé pour un pilotage ALM à la fois efficace et optimisé.

Le secteur de l'assurance vie en France est confronté à une mutation structurelle induite par la remontée des taux d'intérêt, qui met sous pression la compétitivité des fonds en euros et complexifie la gestion actif-passif (ALM). Dans ce contexte, les assureurs s'appuient sur des modèles de projection sophistiqués pour piloter leur bilan et respecter les exigences du cadre réglementaire Solvabilité II. C'est de la rencontre entre ces exigences et les contraintes opérationnelles observées en entreprise qu'est née la problématique de ce mémoire.

\section{Contexte professionnel et genèse de la problématique chez Accenture}

Mon année en alternance au sein des équipes actuarielles d'Accenture m'a plongé au cœur de ces enjeux stratégiques. Ma mission principale a été de participer activement au développement et à l'industrialisation d'un modèle ALM propriétaire, entièrement programmé en Python. Cet outil, destiné à réaliser des projections et des simulations de bilans assurantiels, est conçu pour être à la fois robuste, flexible et performant.

Au cours de ce développement, nous avons été directement confrontés à énormément de contraintes opérationnelles dont le temps de calcul. L'exploration de stratégies multiples, la réalisation de sensibilités aux chocs réglementaires ou l'évaluation de nouvelles offres produits nécessitent un grand nombre de simulations. Or, la granularité fine des portefeuilles de passifs, bien que garantissant une précision maximale, engendre des temps de calcul prohibitifs qui peuvent freiner l'agilité et la prise de décision stratégique. Ce défi technique, vécu au quotidien, a été le point de départ de ma réflexion et constitue le fondement de ce mémoire. Les outils développés durant cette expérience, notamment le modèle ALM et un générateur de portefeuilles synthétiques, seront le socle technique sur lequel reposeront toutes les analyses de cette étude.

\section{Problématique et objectifs du mémoire}

L'optimisation des temps de calcul des modèles ALM est un enjeu de premier plan pour les compagnies d'assurance. Une des pistes explorées pour parvenir à améliorer les temps de calcul est la simplification ou l'agrégation des portefeuilles de passifs. Cependant, une telle démarche soulève une question fondamentale, qui constitue la problématique centrale de ce mémoire :

\begin{center}
\textit{Dans quelle mesure est-il possible d'agréger des portefeuilles de passifs en assurance vie pour optimiser les temps de simulation, sans dénaturer la mesure des indicateurs de risque et de valeur prudentiels issus du cadre Solvabilité II ?}
\end{center}

Pour répondre à cette problématique, ce mémoire poursuit un double objectif. D'une part, il s'agira de tester et de comparer rigoureusement plusieurs méthodes d'agrégation, en partant de portefeuilles de contrats générés grâce à un générateur de contrats développé dans le cadre de ce mémoire. D'autre part, nous chercherons à quantifier précisément l'impact de ces agrégations sur des indicateurs clés tels que le Best Estimate (BE), le Solvency Capital Requirement (SCR) et la Present Value of Future Profits (PVFP). L'enjeu est de déterminer s'il existe un niveau d'agrégation optimal qui préserve la signification économique des résultats tout en offrant un gain de performance substantiel.

Pour mener à bien cette analyse, ce mémoire s'articulera en quatre parties. La \textbf{première} posera le cadre conceptuel de Solvabilité II et de la modélisation ALM. La \textbf{deuxième} décrira l'environnement de simulation et les outils développés. La \textbf{troisième} se concentrera sur l'étude des différentes techniques d'agrégation. Enfin, la \textbf{quatrième} partie présentera et analysera les résultats des sensibilités menées sur les portefeuilles agrégés afin de répondre à notre problématique.