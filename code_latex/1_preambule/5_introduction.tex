\chapter*{Introduction}

% Ajouter manuellement cette section à la table des matières
\addcontentsline{toc}{chapter}{Introduction}

Confirmant son statut de placement privilégié des français, l'assurance vie a vu son encours (provisions mathématiques + provisions pour participation aux bénéfices) atteindre 1 986 milliards d'euros à fin 2024 (France Assureurs) \cite{france_assureurs}. Son attractivité dans un contexte économique incertain est soulignée par une collecte nette de +28,5 milliards d'euros, alimentée par une forte hausse des cotisations (+14,7 \%). Cette dynamique globale masque toutefois une réorientation marquée des flux : alors que les supports en euros enregistrent une collecte nette négative de 4,7 milliards d'euros, les unités de compte captent l'essentiel des nouveaux versements (+33,2 milliards d'euros). Conjuguée à un recul des prestations versées aux assurés (-3,1 \%), cette situation place la gestion actif-passif (ALM) au cœur des enjeux stratégiques pour les assureurs. Ceux-ci doivent piloter leur résultats tout en maintenant l'équilibre entre sécurité et rendement pour les épargnants.

Le secteur de l'assurance vie en France est ainsi confronté à un besoin important de pilotage via la gestion actif-passif (ALM). Pour cela, les assureurs s'appuient sur l'utilisation de modèles ALM. Ces modèles simulent l'impact de différentes stratégies ce qui nécessite cependant un grand nombre de projections stochastiques, cela engendre une contrainte opérationnelle majeure : le temps de calcul. Cette contrainte limite non seulement la capacité à explorer en profondeur l'ensemble des risques et des opportunités, mais freine également l'agilité stratégique et la réactivité des prises de décision. C'est de la rencontre entre ces exigences et des contraintes opérationnelles qu'ont les assureurs qu'est née la problématique de ce mémoire : comment concilier la nécessité de rapidité des calculs avec l'impératif de fidélité des indicateurs de risque ?

Ce mémoire se propose d'investiguer cette problématique en étudiant l'impact des techniques d'agrégation du passif, notamment par la création de \textit{model points}, qui consiste en une représentation simplifiée des contrats d'assurance. L'enjeu est de déterminer si cette modélisation simplifiée des portefeuilles de contrats peut constituer une solution fiable pour le pilotage stratégique, et d'évaluer sous quelles conditions une telle simplification est valide sans masquer des dynamiques de risque essentielles au pilotage de l'entreprise.

Pour répondre à cette problématique, ce mémoire adoptera une double approche. Premièrement, il s'agira de développer un générateur de portefeuilles de passif puis le reste de l'analyse portera sur les effets de l'agrégation sur un portefeuille représentatif du marché français. L'objectif est de comprendre comment les risques évoluent à travers l'agrégation. Pour cela, plusieurs méthodes et approches seront testées et comparées. Le critère de sélection de la méthode la plus pertinente reposera sur un double objectif : minimiser l'écart sur des indicateurs clés de la norme Solvabilité II, optimiser la rapidité des calculs en atteignant le plus haut niveau d'agrégation possible. Une fois la méthode cible sélectionnée, une analyse approfondie de sensibilités sera réalisée sur le portefeuille avant et après agrégation. Cette étude s'appuiera également sur des indicateurs quantitatifs clés issus de la norme Solvabilité II, en évaluant notamment l'impact de chocs économiques sur le Best Estimate, le Solvency Capital Requirement (SCR) et la Present Value of Future Profits (PVFP). Ces métriques permettront de mesurer rigoureusement comment l'agrégation modifie la perception du risque et la valeur économique du portefeuille.

Ce mémoire s'articulera en cinq temps. La première partie posera le cadre conceptuel de l'étude en explorant le contexte réglementaire de Solvabilité II utilisé dans ce mémoire, les produits d'epargne en assurance vie et les fondements des Générateurs de Scénarios Économiques (GSE) en univers risque neutre. Après l'explication du socle théorique, la deuxième partie abordera les principes de la Gestion Actif-Passif (ALM), en détaillant l'architecture du modèle de projection qui servira de base aux travaux de ce mémoire. La troisième partie permettra de poser les bases de l'analyse, avec l'élaboration d'un générateur de portefeuilles de passifs réalistes destiné à produire les données synthétiques cohérentes. Le cœur méthodologique sera présenté en quatrième partie, à travers un protocole d'analyse comparant diverses méthodes d'agrégation afin de choisir celle qui sera utilisée. Enfin, la cinquième partie sera consacrée à l'interprétation des résultats d'agrégations sur des chocs économiques où, par le biais d'analyses de sensibilité approfondies, l'impact de la méthode d'agrégation retenue sur la mesure du risque sera quantifié, validant ou non la pertinence de l'approche adoptée.