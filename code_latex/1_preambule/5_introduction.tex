\chapter*{Introduction}

% Ajouter manuellement cette section à la table des matières
\addcontentsline{toc}{chapter}{Introduction}

Placement privilégié des épargnants français, l'assurance vie a atteint un encours record de 1 989 milliards d'euros à fin 2024 (France Assureurs) \cite{france_assureurs}, confirmant son rôle prépondérant dans le patrimoine financier national. Cette performance est portée par une collecte nette annuelle de +29,4 milliards d'euros, une hausse de 28,2 milliards par rapport à 2023, ce qui témoigne d'une forte attractivité du produit dans un contexte économique incertain. La dynamique de l'année 2024 est marquée par une hausse globale des cotisations de +14 \%, bénéficiant tant aux supports en euros (+17 \%) qu'aux unités de compte (+8 \%). Simultanément, les prestations versées aux assurés sont en recul de 5 \%. Cette combinaison d'une collecte dynamique et de prestations maîtrisées place la gestion actif-passif (ALM) au cœur des enjeux stratégiques pour les assureurs, qui doivent piloter cet afflux de capitaux tout en maintenant l'équilibre entre sécurité et rendement pour les épargnants.

Le secteur de l'assurance vie en France est ainsi confronté à un besoin important de pilotage via la gestion actif-passif (ALM). Pour cela, les assureurs s'appuient sur l'utilisation de modèles ALM. Ces modèles simulent l'impact de différentes stratégies ce qui nécessite cependant un grand nombre de projections stochastiques, cela engendre une contrainte opérationnelle majeure : le temps de calcul. Cette contrainte limite non seulement la capacité à explorer en profondeur l'ensemble des risques et des opportunités, mais freine également l'agilité stratégique et la réactivité des prises de décision. C'est de la rencontre entre ces exigences et des contraintes opérationnelles qu'ont les assureurs qu'est née la problématique de ce mémoire : comment concilier la nécessité de rapidité des calculs avec l'impératif de fidélité des indicateurs de risque ? Ce mémoire se propose d'investiguer cette problématique en étudiant l'impact des techniques d'agrégation du passif, notamment par la création de \textit{model points}, c'est-à-dire une manière simplifiée de représenter les contrats d'assurance. L'enjeu est de déterminer si cette modélisation simplifiée des portefeuilles de contrats peut constituer une approximation fiable pour le pilotage stratégique, et d'évaluer sous quelles conditions une telle simplification est valide sans masquer des dynamiques de risque essentielles au pilotage de l'entreprise.

Pour répondre à cette problématique, ce mémoire adoptera une double approche. Premièrement, il s'agira de développer un générateur de portefeuilles de passif puis le reste de l'analyse portera sur les effets de l'agrégation sur un portefeuille représentatif du marché français. L'objectif est de comprendre comment les risques évoluent à travers une agrégation. Ce mémoire ne se contentera pas d'analyser l'impact d'une seule méthode d'agrégation ; au contraire, plusieurs méthodes et approches seront testées. Le critère de sélection de la méthode la plus pertinente reposera sur un triple objectif : minimiser l'écart des indicateurs clés de Solvabilité II (notamment le Best Estimate et le SCR), optimiser la rapidité des calculs et atteindre le plus haut niveau d'agrégation possible qui garderait une significativité économique pour l'assureur suffisante. L'axe principal de ce mémoire consistera donc à mener une analyse de sensibilités approfondie sur ces portefeuilles, qu'ils soient granulaires ou agrégés. Notre étude s'appuiera sur des indicateurs quantitatifs clés issus de la norme Solvabilité II, en évaluant notamment l'impact des chocs économiques sur le Best Estimate, le Solvency Capital Requirement (SCR) et la Present Value of Future Profits (PVFP). Ces métriques permettront de mesurer rigoureusement comment l'agrégation modifie la perception du risque et la valeur économique du portefeuille.

Ce mémoire s'articulera en un parcours logique et progressif en cinq temps. La première partie posera le cadre conceptuel de l'étude en explorant le contexte réglementaire de Solvabilité II, les produits d'epargne en assurance vie et les fondements des Générateurs de Scénarios Économiques (GSE). Après l'explication du socle théorique, la deuxième partie abordera les principes de la Gestion Actif-Passif (ALM), en détaillant l'architecture du modèle de projection qui servira de base aux travaux de ce mémoire. La troisième partie permettra de poser les bases de l'analyse, avec l'élaboration d'un générateur de portefeuilles de passifs réalistes destiné à produire les données synthétiques cohérentes. Le cœur méthodologique sera présenté en quatrième partie, à travers un protocole d'analyse comparant diverses méthodes d'agrégation afin de choisir celle qui sera utilisée. Enfin, la cinquième partie sera consacrée à l'interprétation des résultats d'agrégations sur des chocs économiques où, par le biais d'analyses de sensibilité approfondies, l'impact de la méthode d'agrégation retenue sur la mesure du risque sera quantifié, validant ainsi la pertinence de l'approche.