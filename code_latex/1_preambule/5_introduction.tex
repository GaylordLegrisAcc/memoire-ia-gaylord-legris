\chapter*{Introduction}
% Ajouter manuellement cette section à la table des matières
\addcontentsline{toc}{chapter}{Introduction}
Le secteur de l'assurance vie est soumis à des réglementations strictes et en constante évolution, notamment avec l'introduction des normes Solvabilité II. Ces réglementations visent à garantir la stabilité financière des compagnies d'assurance tout en protégeant les intérêts des assurés. Dans ce contexte, la modélisation Actif-Passif (ALM) joue un rôle crucial pour évaluer la solvabilité et la performance des portefeuilles d'assurance vie.

L'objectif de ce mémoire est d'analyser les sensibilités des portefeuilles de passifs en assurance vie selon les indicateurs de Solvabilité II, en adoptant une approche par générations et agrégations en Model Points. Cette approche permet de simplifier la complexité des portefeuilles tout en conservant leur représentativité, facilitant ainsi les tests de sensibilité et l'analyse des impacts réglementaires.

Nous commencerons par une introduction au contexte réglementaire et à la modélisation ALM, en mettant l'accent sur les spécificités de l'assurance vie et les différents piliers de Solvabilité II. Nous aborderons également les enjeux de l'ALM et l'importance des générateurs de scénarios économiques.

Ensuite, nous examinerons les contraintes techniques et la création d'outils nécessaires pour la génération et l'agrégation des portefeuilles de passifs. Nous présenterons le modèle ALM développé en Python, ainsi que les méthodes d'agrégation utilisées, telles que le K-means et le DBSCAN.

Nous procéderons ensuite à des tests de sensibilité sur différents portefeuilles de passifs, en analysant les résultats obtenus et en évaluant l'impact des modifications apportées. Enfin, nous conclurons par une synthèse des principaux résultats et des perspectives futures pour améliorer les méthodes et outils utilisés.

Ce mémoire vise à fournir une analyse approfondie et des recommandations pratiques pour optimiser la gestion des portefeuilles de passifs en assurance vie, tout en respectant les contraintes réglementaires et techniques.