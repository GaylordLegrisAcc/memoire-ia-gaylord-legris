\chapter*{Introduction}
% Ajouter manuellement cette section à la table des matières
\addcontentsline{toc}{chapter}{Introduction}

Placement privilégié des épargnants français, l'assurance vie a atteint un encours record de 1 923 milliards d'euros à fin 2023 (France Assureurs), confirmant ainsi son rôle prépondérant dans le patrimoine financier national. Toutefois, ce secteur fait face à une rupture structurelle marquée par la fin du cycle de taux bas et la remontée brutale des taux d'intérêt observée depuis 2022. Le taux de revalorisation moyen des fonds en euros a ainsi atteint 2,6 \% pour l'année 2023 (estimation ACPR), créant un paradigme nouveau. Cette mutation rend d'autres produits d'épargne plus attractifs et exerce une pression concurrentielle inédite sur les contrats d'assurance vie, notamment sur les fonds en euros qui subissent une décollecte nette significative (\textbf{pas sûr, faudra que je regarde la replay de l'analyse SFCR fait ya pas longtemps}). Pour les assureurs, le défi est de taille : leurs portefeuilles d'actifs, majoritairement constitués d'obligations acquises durant la longue période de taux bas, présentent une forte inertie. Cet héritage obligataire freine leur capacité à servir des rendements compétitifs et place la gestion actif-passif (ALM) au cœur des enjeux stratégiques.

Pour piloter leur bilan, les assureurs s'appuient sur des modèles ALM sophistiqués, essentiels pour simuler l'impact de différentes stratégies dans le cadre réglementaire de Solvabilité II. Cependant, la complexité de ces modèles et la nécessité de réaliser un grand nombre de simulations se heurtent à une contrainte opérationnelle majeure : le temps de calcul. Cette contrainte limite la capacité des assureurs à explorer en profondeur l'ensemble des risques et des opportunités. Face à cette réalité, une question centrale émerge : dans quelle mesure une agrégation des engagements de passif permet-elle de préserver la fidélité des indicateurs de risque tout en optimisant les temps de calcul ? Ce mémoire se propose d'investiguer cette problématique en étudiant l'impact de l'agrégation des portefeuilles de passifs. L'enjeu est de déterminer si une représentation plus grossière du passif peut suffire pour le pilotage stratégique et sous quelles conditions une telle simplification est valide, sans masquer des dynamiques de risque essentielles.

Pour répondre à cette problématique, ce mémoire adoptera une double approche. Premièrement, nous reconnaissons l'hétérogénéité des engagements, qui ont été souscrits dans des contextes de taux variés. Une approche par générations de passifs est donc essentielle pour distinguer les dynamiques propres à chaque cohorte de contrats. Cette démarche de génération de portefeuilles synthétiques est fondamentale pour tester en amont l'attractivité et la résilience de nouvelles offres.

Deuxièmement, l'analyse portera sur les effets de l'agrégation de ces différentes générations de passifs. L'objectif est de comprendre comment les risques se combinent et si des effets de portefeuille permettent une mutualisation. Ce mémoire ne se contentera pas d'analyser l'impact d'une seule méthode d'agrégation ; au contraire, nous testerons et comparerons plusieurs approches. Le critère de sélection de la méthode la plus pertinente reposera sur un triple objectif : minimiser l'écart des indicateurs clés (notamment le Best Estimate et le SCR), optimiser la rapidité des calculs et atteindre le plus haut niveau d'agrégation possible qui garderait une significaitivité économique pour l'assureur suffisante.

L'axe principal de ce mémoire consistera donc à mener une analyse de sensibilités approfondie sur ces portefeuilles, qu'ils soient granulaires ou agrégés. Notre étude s'appuiera sur des indicateurs quantitatifs clés issus de la norme Solvabilité II, en évaluant notamment l'impact des chocs économiques sur le Best Estimate, le Solvency Capital Requirement (SCR) et la Present Value of Future Profits (PVFP). Ces métriques permettront de mesurer rigoureusement comment l'agrégation modifie la perception du risque et la valeur économique du portefeuille.

Ce mémoire s'articulera en quatre parties distinctes, chacune conçue pour apporter une réponse progressive et rigoureuse à notre problématique.


La \textbf{première partie} sera consacrée au cadre conceptuel de notre étude. Nous y détaillerons le contexte réglementaire de Solvabilité II, qui définit les exigences de capital et les métriques de risque, ainsi que les principes fondamentaux de la modélisation actif-passif (ALM). L'avantage de cette section est de fournir au lecteur les clés de compréhension essentielles pour appréhender les enjeux techniques et stratégiques du pilotage d'un bilan assurantiel.


La \textbf{deuxième partie} adoptera une approche pratique en se concentrant sur la mise en œuvre de notre environnement de simulation. Nous y décrirons la méthodologie de création des portefeuilles de passifs synthétiques, représentatifs de différentes générations de contrats, ainsi que les outils développés pour leur projection. Cette étape est cruciale car elle garantit la robustesse et la pertinence des analyses qui suivront, en créant un laboratoire d'expérimentation fiable.


La \textbf{troisième partie} constituera le cœur méthodologique de ce mémoire. Elle explorera et comparera de manière systématique plusieurs techniques d'agrégation des engagements de passif. L'objectif sera d'identifier les approches les plus prometteuses, en évaluant leur capacité à simplifier la structure du portefeuille sans dénaturer ses caractéristiques fondamentales. Cette analyse comparative permettra de mettre en lumière les forces et faiblesses de chaque méthode.


Enfin, la \textbf{quatrième partie} présentera et analysera les résultats de nos simulations. À travers des tests de sensibilité approfondis sur les indicateurs clés (SCR, PVFP), nous quantifierons l'impact de chaque méthode d'agrégation sur la perception du risque et la valeur économique. Cette analyse empirique nous permettra de conclure sur la validité des approches testées et de formuler des recommandations concrètes sur les conditions d'utilisation d'un passif agrégé pour un pilotage ALM à la fois efficace et optimisé.